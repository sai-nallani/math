\documentclass{article}
\usepackage{xcolor}
\usepackage{amsmath}
\usepackage{amsfonts}
\begin{document}
\textbf{1309. (\color{red}d0\color{black}, 1994 Norwegian MO, Round 2, P1 of 10)} A walnut salesman has a large supply of nuts. He knows that 20\% of these nuts are empty, and has devised a test for picking them out that discards 20\% of the nuts. However, when cracking the nuts that were discarded, he finds that a quarter of these were not empty after all. What proportion of the nuts that passed his test are empty?

\textbf{595. (\color{red}d0\color{black}, 2019 UK IMOK, M3)} Two numbers are such that the sum of their reciprocals is equal to 1. Each of these numbers is then reduced by 1 to give two new numbers.

Prove that these two new numbers are reciprocals of each other.

\begin{center}
    [\emph{The }reciprocal\emph{ of a non-zero number} $x$ \emph{is the number }$\frac1x$\emph{.}]
\end{center}

\textbf{1365. (\color{red}d1\color{black}, 2015 BMO1, P3 of 6)} Suppose that a sequence $t_0, t_1, t_2, \cdots$ is defined by a formula $t_n = An^2 + Bn + C$ for all integers $n \ge 0$. Here $A$, $B$ and $C$ are real constants with $A \ne 0$. Determine values of $A$, $B$ and $C$ which give the greatest possible number of consecutive terms of the sequence which are also consecutive terms of the Fibonacci sequence, The Fibonacci sequence is defined by $F_0 = 0$, $F_1 = 1$ and $F_m = F_{m-1} + F_{m-2}$ for $m \ge 2$.

\textbf{1296. (\color{red}d1\color{black}, Unknown)} Find all Pythagorean triangles whose areas are numerically equal to their respective perimeters.

\textbf{1071. (\color{red}d1\color{black}, 1997 Croatia MO, 1st Grade, P1 of 4)} Let $n$ be a natural number. Solve the equation \begin{equation*}\lvert\lvert\lvert\cdots\lvert\lvert\lvert x-1\rvert -2\rvert-3\rvert \cdots\rvert - (n-1)\rvert - n\rvert = 0\end{equation*}

\textbf{819. (\color{red}d1\color{black}, 2020 ASC, P1 of 5)} Given real numbers $a$ and $b$, prove that there exists a real number $x$ that satisfies at least one of the following three equations. $$ \begin{array}{r} x^{2}+2 a x+b=0 \\ a x^{2}+2 b x+1=0 \\ a x^{2}+2 x+b=0 \end{array} $$

\textbf{714. (\color{red}d1\color{black}, 2014 NZMO, P1 of 10)} Prove that for all positive real numbers \(a\) and \(b\): \[\frac{(a+b)^3}{4} \geq a^2b + b^2a.\]

\textbf{477. (\color{red}d1\color{black}, 2019 SMC, P24 of 25)} The numbers $x$, $y$ and $z$ are given by $x=\sqrt{12-3\sqrt7}-\sqrt{12+3\sqrt7}, y=\sqrt{7-4\sqrt3}-\sqrt{7+4\sqrt3}, z=\sqrt{2+\sqrt3}-\sqrt{2-\sqrt3}$.
What is the value of $xyz$?

\textbf{441. (\color{red}d1\color{black}, 2012 NIMO)} A number is called purple if it can be expressed in the form $\frac{1}{2^{a} 5^{b}}$ for positive integers $a>b$. Find the sum of all purple numbers.

\textbf{1330. (\color{red}d2\color{black}, 2014/5 BMO1, P4 of 6)} Let $x$ be a real number such that $t=x+x^{-1}$ is an integer greater than 2. Prove that $t_n=x^n+x^{-n}$ is an integer for all positive integers $n$. Determine the values of $n$ for which $t$ divides $t_n$.

\textbf{1324. (\color{red}d2\color{black}, 2008 Philippine MO, 60.2)} If $a$ and $b$ are positive real numbers, what is the minimum value of the expression$$\sqrt{a+b} ( \frac{1}{\sqrt{a}}+\frac{1}{\sqrt{b}} ) ?$$

\textbf{1295. (\color{red}d2\color{black}, 2015 Turkey Junior National Olympiad)} Given a non-constant function $f : \mathbb{R} \to \mathbb{R}$, show that there are real numbers $x$ and $y$ such that \begin{equation*}f(x+y) < f(xy).\end{equation*}

\textbf{1282. (\color{red}d2\color{black}, 2006 AIME I, P9 of 15)} The sequence $a_1, a_2, \ldots$ is geometric with $a_1=a$ and common ratio $r$, where $a$ and $r$ are positive integers. Given that $\log_8 a_1+\log_8 a_2+\cdots+\log_8 a_{12} = 2006,$ find the number of possible ordered pairs $(a,r)$.

\textbf{1268. (\color{red}d2\color{black}, 2022 AIME I, P1 of 15)} Quadratic polynomials $P(x)$ and $Q(x)$ have leading coefficients of $2$ and $-2$, respectively. The graphs of both polynomials pass through the two points $(16,54)$ and $(20,53)$. Find ${P(0) + Q(0)}$.

\textbf{1253. (\color{red}d2\color{black}, 2009 BMO1, P5 of 6)} Find all functions $f$, defined on the real numbers and taking real values, which satisfy the equation $f(x)f(y) = f(x+y) + xy$ for all real numbers $x$ and $y$.

\textbf{1218. (\color{red}d2\color{black}, Folklore)} Let $a, b, c$ be positive real numbers. Show that \begin{equation*}\frac{c}{a} + \frac{a}{b+c} + \frac{b}{c} \ge 2.\end{equation*}

\textbf{1191. (\color{red}d2\color{black}, Cauchy's Functional Equation)} Find all functions $f:\mathbb Qarrow\mathbb R$ which satisfy the following property: $$f(x+y)=f(x)+f(y)$$ for all $x,y\in\mathbb Q$.

\textbf{1183. (\color{red}d2\color{black}, 1975 Chisinau City MO, Grade 8, Day 2, P4 of 6)} Let $x, y$ be real numbers such that $xy = 1$. Show that \begin{equation*}x^2 + y^2 \ge 2\sqrt{2}(x-y).\end{equation*} For which $x, y$ do we have equality?

\textbf{1162. (\color{red}d2\color{black}, 1990 Romanian MO)} Find the least positive integer $m$ such that \begin{equation*}\begin{pmatrix}2n\\n\end{pmatrix}^{1/n} < m\end{equation*} for all $n \in \mathbb{N}$.

\textbf{1149. (\color{red}d2\color{black}, 2015 BMO2, P1 of 4)} The first term $x_1$ of a sequence is $2014$. Each subsequent term of the sequence is defined in terms of the previous term. The iterative formula is $$x_{n+1}=\frac{(\sqrt2+1)x_n-1}{(\sqrt2+1)+x_n}.$$ Find the $2015$th term $x_{2015}$.

\textbf{1134. (\color{red}d2\color{black}, 1996 BMO1, P2 of 5)} A function $f$ is defined over the set of all positive integers and satisfies \begin{equation*}f(1) = 1996\end{equation*} and \begin{equation*}f(1) + f(2) + \cdots + f(n) = n^2 f(n) \quad \text{for all } n > 1.\end{equation*}Calculate the exact value of $f(1996)$.

\textbf{1108. (\color{red}d2\color{black}, 2022 CHMMC Problem 3 of 6)} Let $F(x_1,..., x_n)$ be a polynomial with real coefficients in $n > 1$ “indeterminate” variables $x_1,..., x_n$. We say that $F$ is $n-alternating$ if for all integers $1 \leq i < j \leq n$,
\[
    F(x_1,\ldots, x_i,\ldots , x_j ,\ldots, x_n) = -F(x_1,\ldots, x_j ,\ldots, x_i ,\ldots, x_n),
\]
i.e. swapping the order of indeterminates $x_i, x_j$ flips the sign of the polynomial. For example, $x^2_1x_2 - x_2^2x_1$ is $2-alternating$, whereas $x_1x_2x_3 +2x_2x_3$ is not $3-alternating$.
Note: two polynomials $P(x_1,..., x_n)$ and $Q(x_1,..., x_n)$ are considered equal if and only if each monomial constituent $\alpha x_1^{k_1} \ldots x_n^{k_n}$  of $P$ appears in $Q$ with the same coefficient $\alpha$, and vice versa. This is equivalent to saying that $P(x_1,\ldots, x_n) = 0$ if and only if every possible monomial constituent of $P$ has coefficient $0$.
\begin{enumerate}
    \item Compute a $3-alternating$ polynomial of degree $3$.
    \item Prove that the degree of any nonzero $n-alternating$ polynomial is at least $\binom{n}{2}$
\end{enumerate}

\textbf{1092. (\color{red}d2\color{black}, 2017 Austria Regional Competition, P1 of 4)} Let \(x_1,\) \(x_2,\) \(\ldots,\) \(x_9\) be nonnegative real numbers subject to the condition \[x_1^2 + x_2^2 + \ldots + x_9^2 \geq 25.\]  Prove that there exist three of these numbers with a sum of at least \(5.\)

\textbf{1052. (\color{red}d2\color{black}, 2021 NICE Spring P 13 of 25)} Suppose $x$ and $y$ are nonzero real numbers satisfying the system of equations
\begin{align*}
    3x^2 + y^2 & = 13x, \\
    x^2 + 3y^2 & = 14y.
\end{align*}
Find $x+y$.

\textbf{1030. (\color{red}d2\color{black}, 2017/8 BMO1, P4 of 6)} Consider sequences $a_1,a_2,a_3,\ldots$ of positive real numbers with $a_1=1$ and such that $$a_{n+1}+a_n=(a_{n+1}-a_n)^2$$ for each positive integer $n$. How many possible values can $a_{2017}$ take?

\textbf{1015. (\color{red}d2\color{black}, 2001 BMO1, P3 of 5)} Find all real solutions to the equation \begin{equation*}x + \lfloor\frac{x}{6}\rfloor = \lfloor\frac{x}{2}\rfloor + \lfloor\frac{2x}{3}\rfloor,\end{equation*} where $\lfloor t\rfloor$ denotes the largest integer less than or equal to the real number $t$.

\textbf{987. (\color{red}d2\color{black}, 2009 Costa Rica Final Round, P5 of 6)} Suppose the polynomial \begin{equation*}x^n + a_{n-1}x^{n-1} + \dots + a_1x + a_0 \equiv (x+r_1)(x+r_2)\dots(x+r_n)\end{equation*} with $r_1, \dots, r_n$ real numbers. Show that \begin{equation*}(n-1)a_{n-1}^2 \ge 2na_{n-2}.\end{equation*}

\textbf{981. (\color{red}d2\color{black}, 2018 Cyprus TST, P1 of 5)} Determine all integers $n \geq 2$ for which the number 11111 in base $n$ is a perfect square.

\textbf{966. (\color{red}d2\color{black}, 2015 Austria Beginners' Competition,  P2 of 4)} Let $x, y$ be positive real numbers such that $xy = 4$. Prove that \begin{equation*}\frac{1}{x+3} + \frac{1}{y+3} \le \frac{2}{5}.\end{equation*} For which $x$ and $y$ does equality hold?

\textbf{938. (\color{red}d2\color{black}, 2001 Korean MO, P? of 8)} Let \begin{equation*}f(x) = \frac{2}{4^x + 2}\end{equation*} for real numbers $x$. Evaluate \begin{equation*}f(\frac{1}{2001}) + f(\frac{2}{2001}) + \cdots + f(\frac{2000}{2001}).\end{equation*}

\textbf{890. (\color{red}d2\color{black}, 2015 AMO, P1 of 8)} Define the sequence $a_{1}, a_{2}, a_{3}, \ldots$ by $a_{1}=4, a_{2}=7$, and $$ a_{n+1}=2 a_{n}-a_{n-1}+2, \quad \text { for } n \geq 2. $$ Prove that, for every positive integer $m$, the number $a_{m} a_{m+1}$ is a term of the sequence.

\textbf{791. (\color{red}d2\color{black}, 2020 Tournament of Towns Junior O-Level, P4 of 5)} For some integer n the equation $x^2 + y^2 + z^2 -xy -yz - zx = n$ has an integer solution $x, y, z$. Prove that the equation$ x^2 + y^2 - xy = n$ also has an integer solution $x, y$.

\textbf{770. (\color{red}d2\color{black}, 2018 Real IMOSL, A2)} Let $f: \mathbb{N} arrow \mathbb{N}$ be a function such that $$|f(n)-\frac{\sqrt{5}+1}{2} n|<\frac{\sqrt{5}-1}{2}$$ Prove that $f(f(n))=f(n)+n$.

\textbf{750. (\color{red}d2\color{black}, 2020 Cambridge Maths Tripos Exam, Paper 2, Q8 (Adapted))} Tony Wang has written an infinitely long sequence of numbers, where each of the numbers $0, 1, 2, 3, \dots$ appears exactly once and there are no other numbers.\\

\vspace{0.75em}

Brainy then gives Joe a number $n \geq 0$, which he writes down. Then Joe does the following task: if the last number he wrote down is $k$, he'll write down the $k$th term of Tony's sequence. Joe then repeats this task, stopping only if he writes down a number he has written before.\\

\vspace{0.75em}

Can Tony choose a sequence so that no matter what $n$ Brainy chooses, Joe keeps writing down numbers forever?

\textbf{672. (\color{red}d2\color{black}, RAZING THUNDER)} Find all positive real $x$ such that $$x^{-x^x}=2^{\sqrt2}.$$

\textbf{644. (\color{red}d2\color{black}, Folklore)} Consider functions $f, g: \mathbb{Z} \to \mathbb{Z}$ such that $f(g(x)) = x$ for all $x \in \mathbb{Z}.$ Is it necessarily true that $g(f(x)) = x$ for all $x \in \mathbb{Z}?$

\textbf{610. (\color{red}d2\color{black}, PST 10.0.1)} Find all functions \(f : \mathbb R \to \mathbb R\) such that \[f(x - f(y)) = 1 - x - y\] for all real numbers \(x\) and \(y\).

\textbf{602. (\color{red}d2\color{black}, 2017 Irish EGMO TST, P1 of 10)} Triangle $ABC$ has area $S$. Denote by $M$, $N$ and $P$ the midpoints of $BC$, $CA$, and $AB$ respectively. Prove that
$$
    2 S(\frac{1}{A B}+\frac{1}{B C}+\frac{1}{C A}) \leq A M+B N+C P<\frac{3}{2}(A B+B C+C A)
$$

\textbf{596. (\color{red}d2\color{black}, 2020 AMOC Senior Contest, P1 of 5)} Given real numbers $a$ and $b$, prove that there exists a real number $x$ that satisfies at least one of the following three equations.
\begin{align*}
    x^{2}+2 a x+b=0   \\
    a x^{2}+2 b x+1=0 \\
    a x^{2}+2 x+b=0
\end{align*}

\textbf{455. (\color{red}d2\color{black}, 2019 NZMO2, P3 of 5)} Let $a$,$b$ and $c$ be positive real numbers such that $a+b+c= 3$.  Prove that $$a^a+b^b+c^c \geq 3$$

\textbf{442. (\color{red}d2\color{black}, Project Euler, P1)} If we list all the natural numbers below 10 that are multiples of 3 or 5, we get 3, 5, 6 and 9. The sum of these multiples is 23.

Find the sum of all the multiples of 3 or 5 below 1000.


\textbf{392. (\color{red}d2\color{black}, 2018 Pan African MO, P1 of 6)} Find all functions $f\;:\; \mathbb{Z} \longrightarrow \mathbb{Z}$ such that $(f(x + y))^2 = f(x^2) + f(y^2)$ for all $x$, $y \in \mathbb{Z}$

\textbf{378. (\color{red}d2\color{black}, 2003 Canadian Buffet Contest Q7 of 30)} For any positive integer \(n,\) let \(S_{n}=1+2+\cdots+n .\) Determine the maximum value of \(\frac{S_{n}}{(n+32) S_{n+1}}\).

\textbf{358. (\color{red}d2\color{black}, 2019 NZ SMC P3 of 15, Generalised)} Find all polynomials $P(x)$ with real coefficients such that $$P(P(x)) = P(x)$$ for all real $x$.

\textbf{350. (\color{red}d2\color{black}, 2020 Maynooth Olympiad, P7 of 10 )} The sequence $a_n$, $n \geq 1$ has $a_1 = a_2 = 2$, $a_3 = 8$ and satisfies
$$
    a_{n + 1} = \frac{2020 + a_n a_{n - 1}}{a_{n-2}}
$$
for $n \geq 3$. Determine the value of
$$
    \frac{a_{2021} + a_{2019}}{a_{2020}}.
$$

\textbf{343. (\color{red}d2\color{black}, BMO1 2009, Q4 of 6)} Find all functions $f$, defined on the real numbers and taking real values, which satisfy the equation $f(x)f(y) = f(x + y) + xy$ for all real numbers $x$ and $y$.

\textbf{315. (\color{red}d2\color{black}, 2012 Irish MO, P1, Q3 of 5)} Find, with proof, all polynomials $f$ such that $f$ has nonnegative integer coefficients, $f(1) = 8$ and $f(2) = 2012$.

\textbf{308. (\color{red}d2\color{black}, 2016 Singapore MO, Jnr. Rnd. 2, P2 of 5)} Let $a_1, a_2, \cdots a_9$ be numbers such that $0 \leq p \leq a_i \leq q$ for each $i = 1, 2, \cdots 9$. Prove that \[\frac{a_1}{a_9} + \frac{a_2}{a_8} + \cdots + \frac{a_9}{a_1} \leq 1 + \frac{4(p^2+q^2)}{pq}.\]

\textbf{245. (\color{red}d2\color{black}, University of Indonesia Entrance Exam 2019 )} Given $a^2 - bc, b^2-ca , c^2 -ab$ is an arithmetic progression with $a+b+c = 12$, find the value of $a+c$

\textbf{224. (\color{red}d2\color{black}, 45th Austrian MO, Beginner Contest, Q3 of 4)} Let $a,b,c,d$ be real numbers with $a<b<c<d$.



\vspace{2mm}



Sort the number $x = ab+cd , y = bc+ad$ and $z = ca+bd$ in ascending order.

\textbf{190. (\color{red}d2\color{black}, 2019 New Zealand Senior MC, Q11 of 15)} A \textit{harmonic number} is a number of the form $\frac{1}{n}$ where $n \in \mathbb{N}$. Show that all fractions can be written as a sum of harmonic numbers, for example $\frac56 = \frac12 + \frac13$

\textbf{189. (\color{red}d2\color{black}, Mock 2017 INAMO Round 2 Part B, Problem 1 of 5)} Find all non-negative reals $(x,y)$ such that $x+y \leq 1$ and:
$$3xy = 2x(1-x) + 2y(1-y)$$

\textbf{147. (\color{red}d2\color{black}, 2018 German National Olympiad, Final Round P1
    )} Find all real numbers \(x,y,z\) satisfying the following system of equations:
\begin{align*}
    xy+z & =-30 \\
    yz+x & = 30 \\
    zx+y & =-18
\end{align*}

\textbf{133. (\color{red}d2\color{black}, 2012 BMO1 P3 of 6)} Find all real numbers \(x, y\) and \(z\) which satisfy the simultaneous equations \(x^2 - 4y + 7 = 0, y^2 - 6z + 14 = 0\) and \(z^2 - 2x - 7 = 0\).

\textbf{95. (\color{red}d2\color{black}, 2019 NZ Senior Math Comp Prelim Round, Q10)} Solve the following equation, giving exact solutions.

\[\sqrt[3]{5-x} + \sqrt[3]{5+x}=\sqrt[3]{25}.\]

\textbf{80. (\color{red}d2\color{black}, 2015 Australian Intermediate Math Olympiad, Q8)} Determine the number of non-negative integers \(x\) that satisfy the equation \[ \lfloor \frac{x}{44}  \rfloor  =  \lfloor \frac{x}{45}  \rfloor .\]

(\textit{Note: if \(r\) is any real number, then \(\lfloor r \rfloor\) denotes the largest integer less than or equal to \(r\)}.)

\textbf{78. (\color{red}d2\color{black}, 1983 Yugoslav Federal MC (24th), Grade 3/4, Q2 of 4)} A function \(f\) is defined on the integers and satisfies \[ f(x) = \begin{cases} x-10, & \text{if } x > 100,\\ f(f(x+11)), & \text{if } x \leq 100. \end{cases}\]

Prove that \(f(x) = 91\) for all \(x \leq 100\).

\textbf{57. (\color{red}d2\color{black}, 2011 BMO1, Q1 of 6)} Find all \emph{positive or negative} integers for which $n^2+20n+11$ is a perfect square. \textit{Remember that you must justify that you have found them all.}

\textbf{54. (\color{red}d2\color{black}, 2018 Putnam, Q1 of 12)} Find all ordered pairs of integers $(a,b)$ for which $$\frac{1}{a}+\frac{1}{b}= \frac{3}{2018}$$

\textbf{41. (\color{red}d2\color{black}, 2006 Romanian MO, 9th Form, Q1 of 4)} Find the maximum value of $(x^3+1)(y^3+1)$, where $x$ and $y$ are real numbers such that $x + y = 1$.

\textbf{1254. (\color{red}d3\color{black}, Unknown)} If $a,b,c>0$ and $a+b+c=6$, show that \[ ( a + \frac 1b )^2 + ( b + \frac 1c )^2 + ( c + \frac 1a )^2 \geq \frac{75}{4}.\]

\textbf{1114. (\color{red}d3\color{black}, 2022 EGMO, P4 of 6)} Given a positive integer $n \geq 2$, determine the largest positive integer $N$ for which there exist $N+1$ real numbers $a_{0}, a_{1}, \ldots, a_{N}$ such that
\begin{enumerate}
    \item $a_{0}+a_{1}=-\frac{1}{n}$, and
    \item $(a_{k}+a_{k-1})(a_{k}+a_{k+1})=a_{k-1}-a_{k+1}$ for $1 \leq k \leq N-1$
\end{enumerate}

\textbf{1107. (\color{red}d3\color{black}, 2017 European Mathematical Cup, Senior Problem 1 of 4)} Find all functions $f: \mathbb{N} arrow \mathbb{N}$ such that the inequality
$$
    f(x)+y f(f(x)) \leq x(1+f(y))
$$
holds for all positive integers $x, y$.

\textbf{1044. (\color{red}d3\color{black}, 2021 BMO2, P2 of 4)} Find all functions $f$ from the positive integers to the positive integers such that for all $x, y$ we have:
$$
    2 y f(f(x^{2})+x)=f(x+1) f(2 x y)
$$

\textbf{1009. (\color{red}d3\color{black}, 2016 NZ Camp Selection, P2 of 9)} We consider \(5 \times 5\) tables containing a real number in each of the 25 cells. The same number may occur in different cells, but no row or column contains five equal numbers. Such a table is \emph{balanced} if the number in the middle cell of every row and column is the average of the numbers in that row or column. A cell is called \emph{small} if the number in that cell is strictly smaller than the number in the cell in the very middle of the table. What is the least number of small cells that a balanced table can have?

\textbf{996. (\color{red}d3\color{black}, 2021 ITAMO, P4 of 6)} Given two fractions $a / b$ and $c / d$ we define their pirate sum as:
$$\frac{a}{b} \star \frac{c}{d}=\frac{a+c}{b+d},$$
where the two initial fractions are simplified the most possible, like the result.

For example, the pirate sum of $2 / 7$ and $4 / 5$ is $1 / 2$.

Given an integer $n \geq 3$, initially on a blackboard there are the fractions:
$\frac{1}{1}, \frac{1}{2}, \frac{1}{3}, \ldots, \frac{1}{n}$.
At each step we choose two fractions written on the blackboard, we delete them and write at their place their pirate sum.
Continue doing the same thing until on the blackboard there is only one fraction.
Determine, in function of $n$, the maximum and the minimum possible value for the last fraction.

\textbf{828. (\color{red}d3\color{black}, 2016 AMO, P3 of 8)} For a real number $x$, define $\lfloor x\rfloor$ to be the largest integer less than or equal to $x$, and define $\{x\}=x-\lfloor x\rfloor$.
\smallbreak
(a) Prove that there are infinitely many positive real numbers $x$ that satisfy the inequality
$$ \{x^{2}\}-\{x\}>\frac{2015}{2016} $$
(b) Prove that there is no positive real number $x$ less than $1000$ that satisfies this inequality.

\textbf{778. (\color{red}d3\color{black}, 2015 AMO, P6 of 8)} Determine the number of distinct real solutions of the equation $$(x-1)(x-3)(x-5) \cdots(x-2015)=(x-2)(x-4)(x-6) \cdots(x-2014)$$

\textbf{730. (\color{red}d3\color{black}, 2021 AMO P1)}
Find the solutions to
\begin{center}
    $\displaystyle a^{2n} + b^{2n} = 1 $ and $\displaystyle a^{2n+1} + b^{2n+1}$
\end{center}
for positive integer $n$ and real numbers $a,b$

\textbf{674. (\color{red}d3\color{black}, 2014 Taiwan TST1 Quiz 2, P1)} Find all increasing functions $f$ from the nonnegative integers to the integers satisfying $f(2)=7$ and\[ f(mn) = f(m) + f(n) + f(m)f(n) \]for all nonnegative integers $m$ and $n$.

\textbf{603. (\color{red}d3\color{black}, 2016 EGMO, P1 of 6)} Let $n$ be an odd positive integer, and let $x_1,x_2,\cdots ,x_n$ be non-negative real numbers. Show that\[ \min_{i=1,\ldots,n} (x_i^2+x_{i+1}^2) \leq \max_{j=1,\ldots,n} (2x_jx_{j+1}) \]where $x_{n+1}=x_1$.

\textbf{568. (\color{red}d3\color{black}, 2020 October Monday Maths Workshop, Q5 of 6)} If the numbers $2^n$ and $5^n$ ($n \in \mathbb{N}$) start with the same digit, what is this digit?

\textbf{525. (\color{red}d3\color{black}, 2015 Canadian MO, P1 of 5)} Find all functions $f: \mathbb{N} arrow \mathbb{N}$ such that for all $n \in \mathbb{N}$
$$
    (n-1)^{2}<f(n) f(f(n))<n^{2}+n.
$$

\textbf{434. (\color{red}d3\color{black}, 2008 Canadian Winter Camp Buffet Contest A1)} Find all functions \(f: \mathbb{R} arrow \mathbb{R}\) such that for all real numbers \(x\) and \(y\)
\[
    f(x f(y)+x)=xy + f(x)
\]

\textbf{365. (\color{red}d3\color{black}, 2015 CMO, P1 of 5)} Let $\mathbb{N} = \{1,2,3,\dots\}$ be the set of positive integers. Find all functions $f$, defined on $\mathbb{N}$ and taking values in $\mathbb{N}$, such that $(n-1)^2 < f(n)f(f(n)) < n^2 + n$ for every positive integer $n$.

\textbf{345. (\color{red}d3\color{black}, PST 11.0.6 (c))} Prove that \(a^{bc}b^{ca}c^{ab} \leq a^{ab}b^{bc}c^{ca}\), where \(a\), \(b\), and \(c\) are positive real numbers.

\textbf{322. (\color{red}d3\color{black}, 2008 BMO2, P1 of 4)} Find the minimum value of $x^2 + y^2 + z^2$ where $x$, $y$, $z$ are real numbers such that $x^3 + y^3 + z^3 - 3xyz = 1$.

\textbf{204. (\color{red}d3\color{black}, 2018 NZ Camp Selection Problems, Q5 )} Let $a$, $b$ and $c$ be positive real numbers satisfying \[\frac{1}{a+2019} + \frac1{b+2019} + \frac1{c+2019} = \frac1{2019}.\] Prove that $abc \geq 4038^3$.

\textbf{182. (\color{red}d3\color{black}, 2018 Irish MO Paper 1, Q3 of 5)} Find all functions $f(x) = ax^2 + bx + c$, with $a \ne 0$, such that $f(f(1)) = f(f(0)) = f(f(-1))$ .

\textbf{162. (\color{red}d3\color{black}, HDIGH 61, P2)} Find all functions $f:\mathbb{R}\to\mathbb{R}$ such that for all real $x,y$, $$f(f(x)+y)=x+f(f(y))$$

\textbf{127. (\color{red}d3\color{black}, 2014 USAMO,  P1 of 6    )} Let $a$, $b$, $c$, $d$ be real numbers such that $b-d \ge 5$ and all zeros $x_1, x_2, x_3,$ and $x_4$ of the polynomial $P(x)=x^4+ax^3+bx^2+cx+d$ are real. Find the smallest value the product $(x_1^2+1)(x_2^2+1)(x_3^2+1)(x_4^2+1)$ can take.

\textbf{123. (\color{red}d3\color{black}, Gaokao)} $x, y, z$ are positive reals. Let $S = \sqrt{x+2} + \sqrt{y+5} + \sqrt{z+10}$, and $T = \sqrt{x+1} + \sqrt{y+1} + \sqrt{z+1}$. Find the minimal possible value of $S^2 - T^2$.

\textbf{114. (\color{red}d3\color{black}, 2007 Italian MO, P6)} Let \(n \geq 2\) be a given integer. Determine
\begin{enumerate}
    \item the largest real \(c_n\) such that \[\frac{1}{1+a_1} + \frac{1}{1+a_2} + \cdots + \frac{1}{1+a_n} \geq c_n\] holds for any positive numbers \(a_1, a_2, \cdots, a_n\) with \(a_1a_2\cdots a_n = 1\),
    \item[(b)] the largest real \(d_n\) such that \[\frac{1}{1+2a_1} + \frac{1}{1+2a_2} + \cdots + \frac{1}{1+2a_n} \geq d_n\] holds for any positive numbers  \(a_1, a_2, \cdots, a_n\) with \(a_1a_2\cdots a_n = 1\).
\end{enumerate}

\textbf{107. (\color{red}d3\color{black}, 2004 British MO Round 2, Q3)} \begin{itemize}
    \item Given real numbers $a$, $b$ and $c$ with $a+b+c=0$, prove that $a^3+b^3+c^3>0$ iff $a^5+b^5+c^5>0$.
    \item[(b)] Given real numbers $a$, $b$, $c$ and $d$ with $a+b+c+d=0$, prove that $a^3+b^3+c^3+d^3>0$ iff $a^5+b^5+c^5+d^5>0$.
\end{itemize}

\textbf{98. (\color{red}d3\color{black}, 2009 Hungary-Israel Binational, Q2)} Denote the three real roots of the cubic \(x^3 - 3x - 1 = 0\) by \(x_1, x_2,\) and \(x_3,\) with \(x_1 < x_2 < x_3\). Prove that \(x_3^2 - x_2^2 = x_3 - x_1.\)

\textbf{92. (\color{red}d3\color{black}, 2009 Moldova TST2, Q2)} Determine all functions \(f\) from the non-negative reals to the non-negative reals such that
\[f(x+y-z) + f(2\sqrt{xz}) + f(2\sqrt{yz}) = f(x+y+z)\]
for all real \(x,y,z \geq 0\) such that \(x + y \geq z\).
\textbf{19. (\color{red}d3\color{black}, 2004 Swedish MO (44th), Final Round, Q3)} Find all functions $f$ satisfying $f(x)+x f(1-x) = x^2$ for all real $x$.
\textbf{1368. (\color{red}d4\color{black}, 2002 IMOSL, A1)} Find all functions from the reals to the reals such that $$f(f(x)+y) = 2x + f(f(y)-x)$$ for all real $x,y$.

\textbf{1332. (\color{red}d4\color{black}, 2007 Canadian MO, P3 of 5)} Suppose that $f$ is a real-valued function for which
\[
    f(xy) + f(y - x) \geq f(y + x)
\]
for all real numbers $x$ and $y$. Prove that $f(x) \geq 0$ for all real $x.$

\textbf{1256. (\color{red}d4\color{black}, 2022 NZMO1, P3 of 8)} Find all real numbers \(x\) and \(y\) such that
\begin{align*}
    x^2 + y^2                         & = 2, \\
    \frac{x^2}{2-y} + \frac{y^2}{2-x} & = 2.
\end{align*}

\textbf{1130. (\color{red}d4\color{black}, 2005 IMOSL, A2)} Find all functions $f: \mathbb{R}^{+} arrow \mathbb{R}^{+}$ which have the property:
$$
    f(x) f(y)=2 f(x+y f(x))
$$
for all positive real numbers $x$ and $y$.

\textbf{1088. (\color{red}d4\color{black}, 2013 SMT, A9)} Let $a=-\sqrt{3}+\sqrt{5}+\sqrt{7}$, $b=\sqrt{3}-\sqrt{5}+\sqrt{7}$ and $c=\sqrt{3}+\sqrt{5}-\sqrt{7}$. Evaluate $$\frac{a^4}{(a-b)(a-c)}+\frac{b^4}{(b-c)(b-a)}+\frac{c^4}{(c-a)(c-b)}.$$

\textbf{884. (\color{red}d4\color{black}, 2021 Taiwan "IMOC", A5)} Let $M$ be an arbitrary positive real number greater than 1 , and let $a_{1}, a_{2}, \ldots$ be an infinite sequence of real numbers with $a_{n} \in[1, M]$ for any $n \in \mathbb{N}$. Show that for any $\varepsilon>0$, there exists a positive integer $n$ such that
$$
    \frac{a_{n}}{a_{n+1}}+\frac{a_{n+1}}{a_{n+2}}+\cdots+\frac{a_{n+t-1}}{a_{n+t}} \geq t-\varepsilon
$$
holds for any positive integer $t$.

\textbf{878. (\color{red}d4\color{black}, 2019 EGMO, P1 of 6)} Find all triples $(a, b, c)$ of real numbers such that $ab + bc + ca = 1$ and

$$a^2b + c = b^2c + a = c^2a + b.$$

\textbf{877. (\color{red}d4\color{black}, 2021 Cambridge Maths Tripos Exam, Paper 4, Q5 (Adapted))} Let $n$ be a positive integer. Evaluate the sum
$$
    \sum_{r=0}^{n}(-1)^{r}\binom{n}{r}^2
$$

\textbf{843. (\color{red}d4\color{black}, 2000 IMO, P2 of 6)} Let $ a, b, c$ be positive real numbers so that $ abc = 1$. Prove that
\[ ( a - 1 + \frac 1b ) ( b - 1 + \frac 1c ) ( c - 1 + \frac 1a ) \leq 1.
\]

\textbf{708. (\color{red}d4\color{black}, 2010 USA TST, P1 of 9)} Let $P$ be a polynomial with integer coefficients such that $P(0)=0$ and
\[\gcd(P(0), P(1), P(2), \ldots ) = 1.\]
Show there are infinitely many $n$ such that
\[\gcd(P(n)- P(0), P(n+1)-P(1), P(n+2)-P(2), \ldots) = n.\]

\textbf{668. (\color{red}d4\color{black}, 2013/14 BMO1, P6 of 6)} The angles $A$, $B$ and $C$ of a triangle are measured in degrees, and the lengths of the opposite sides are $a$, $b$ and $c$ respectively. Prove that $$60\le \frac{aA+bB+cC}{a+b+c} < 90.$$

\textbf{667. (\color{red}d4\color{black}, 2014 Taiwan TST3 Quiz 3, P1)} Positive integers $x_1, x_2, \dots, x_n$ ($n \ge 4$) are arranged in a circle such that each $x_i$ divides the sum of the neighbors; that is\[ \frac{x_{i-1}+x_{i+1}}{x_i} = k_i \]is an integer for each $i$, where $x_0 = x_n$, $x_{n+1} = x_1$. Prove that\[ 2n \le k_1 + k_2 + \dots + k_n < 3n. \]

\textbf{653. (\color{red}d4\color{black}, 2003 Canada MO, P3 of 5)} Find all real positive solutions (if any) to
\begin{align*}
    x^3 + y^3 + z^3 & = x + y + z, \text{ and} \\
    x^2 + y^2 + z^2 & = xyz.
\end{align*}

\textbf{646. (\color{red}d4\color{black}, Mathematics Stack Exchange)} Suppose $S$ is a set of reals such that for all $a, b \in S, a-b \in S$ and for all $0 \neq a \in S, \frac{1}{a} \in S .$ If $1 \in S,$ prove that
$$ a, b \in S \Longrightarrow a b \in S $$

\textbf{645. (\color{red}d4\color{black}, 2002 Canadian MO, P5 of 5)} Let $\mathbb{N}=\{0,1,2, \ldots\}$. Determine all functions $f: \mathbb{N} arrow \mathbb{N}$ such that
$$
    x f(y)+y f(x)=(x+y) f(x^{2}+y^{2})
$$
for all $x$ and $y$ in $\mathbb{N}$.

\textbf{534. (\color{red}d4\color{black}, 1990 FIST, P2 of 5)} The integer part $[ x ]$ of a number $x$ is the greatest integer which is not greater than $x$. The \textit{fractional part} $f(x)$ is defined by $f(x) = x - [x ]$. \\
Find a positive number $x$ such that $$f(x) + f( \frac1x) = 1.$$\\
Are there any \textit{rational} solutions?

\textbf{506. (\color{red}d4\color{black}, 1996 USAMO, P6 of 6)} Determine (with proof) whether there is a subset $X$ of the integers with the following property: for any integer $n$ there is exactly one solution of $a + 2b = n$ with $a,b \in X$.

\textbf{471. (\color{red}d4\color{black}, 2009 Croatia TST, P1 of 4)} Find all real numbers $x, y, z$ such that the following two equations are satisfied: \begin{align*}3(x^2+y^2+z^2) &= 1,\\ x^2y^2 + y^2z^2 + z^2x^2 &= xyz(x+y+z)^3.\end{align*}

\textbf{395. (\color{red}d4\color{black}, 2019 ELMO Shortlist, A1)} Let $a$, $b$, $c$ be positive reals such that $\frac{1}{a}+\frac{1}{b}+\frac{1}{c}=1$. Show that $$a^abc+b^bca+c^cab\ge 27bc+27ca+27ab.$$

\textbf{388. (\color{red}d4\color{black}, 2005 BMO1, P5 of 5)} Let $S$ be a set of rational numbers with the following properties:
\begin{enumerate}
    \item[i)] $\frac12\in S$;
    \item[ii)] If $x\in S$, then both $\frac{1}{x+1}\in S$ and $\frac{x}{x+1}\in S$.
\end{enumerate}
Prove that $S$ contains all rational numbers in the interval $0<x<1$.

\textbf{283. (\color{red}d4\color{black}, 2020 ICMC, Round 1, P2 of 6)} Find integers \( a \) and \( b \) such that

\[a^b = 3^0\binom{2020}{0} - 3^1\binom{2020}{2} + 3^2\binom{2020}{4} - \dots + 3^{1010}\binom{2020}{2020}.\]

\textbf{240. (\color{red}d4\color{black}, 1997 Brazil MO, P5 of 6)} Let $f(x) = x^2 + c$ with $c \in \mathbb{Q}$. Define $f^0(x) = x$ and $f^{n+1}(x) = f(f^n(x))$ for each $n \in \mathbb{N}$. We say that $x \in \mathbb{R}$ is \emph{pre-periodic} if the set $\{f^n(x) \, | \, n \in \mathbb{N}\}$ is finite.\\

\textbf{206. (\color{red}d4\color{black}, PST 10.5)} Do there exist functions \(f : \mathbb{R} arrow \mathbb{R}\) and \(g : \mathbb{R} arrow \mathbb{R}\) such that \[f(g(x)) = x^2 \quad \text{ and } \quad g(f(x)) = x^3\] for all real numbers \(x\)?

\textbf{142. (\color{red}d4\color{black}, 2016 Canadian MO, P3 of 5)} Find all polynomials \(P(x)\) with integer coefficients such that \(P(P(n) + n)\) is a prime number for infinitely many integers \(n\).

\textbf{128. (\color{red}d4\color{black}, 2014 IMOSL, A2 )} Define the function $f:(0,1)\to (0,1)$ by \[\displaystyle f(x) = \{ \begin{array}{lr} x+\frac 12 & \text{if}\ \  x < \frac 12\\ x^2 & \text{if}\ \  x \ge \frac 12 \end{array} .\] Let $a$ and $b$ be two real numbers such that $0 < a < b < 1$. We define the sequences $a_n$ and $b_n$ by $a_0 = a, b_0 = b$, and $a_n = f( a_{n -1})$, $b_n = f (b_{n -1} )$ for $n > 0$. Show that there exists a positive integer $n$ such that \[(a_n - a_{n-1})(b_n-b_{n-1})<0.\]

\textbf{88. (\color{red}d4\color{black}, 2017 Singapore MO Senior Division R2, Q4)} Find all functions $f : \mathbb{Z}^{+} \to \mathbb{Z}^{+}$ such that $f(k+1) > f(f(k))$ for all $k \geq 1$, where $\mathbb{Z}^{+}$ is the set of positive integers.

\textbf{46. (\color{red}d4\color{black}, 2006 Flanders MO, Q4 of 4)} Find all functions $f:\mathbb{R}\setminus{0,1}\to\mathbb{R}$ such that \begin{equation}f(x) + f(\frac{1}{1-x})  = 1 + \frac{1}{x(1-x)}\end{equation} for all real $x$.

\textbf{38. (\color{red}d4\color{black}, 2007 Canadian MO (37th), Q2 of 5)} For two real numbers $a,b$ with $ab \neq 1$, define operation $\star$ by $a \star b = \frac{a + b - 2ab}{1 - ab}$. Start with a list of $n \geq 2$ real numbers that all satisfy $0 < x < 1$. Select any two numbers $a$ and $b$ in the list; remove them and put the number $a \star b$ at the end of the list, therefore reducing its length by 1. Repeat this procedure until a single number remains.\
(a) Prove that this single number is the same regardless of the choice of pair at each stage.\
(b) Suppose the condition on the numbers in the list is weakened to $0 < x \leq 1$. What happens if the list contains exactly one 1?

\textbf{31. (\color{red}d4\color{black}, 2005 Serbia and Montenegro TST, Test 1, Q3)} Find all polynomials $P(x)$ that satisfy $P(x^2+1) = P(x)^2 + 1$ for all $x$.

\textbf{3. (\color{red}d4\color{black}, 2014 BMO2, Q2)} Prove that it is impossible to have a cuboid for which the volume, the surface area and the perimeter are numerically equal. (The perimeter of a cuboid is the sum of the lengths of all its twelve edges.)

\textbf{1334. (\color{red}d5\color{black}, 2010 IMO, P1 of 6)} Find all functions $f:\mathbb{R}arrow\mathbb{R}$ such that for all $x,y\in\mathbb{R}$ the following equality holds \[ f(\lfloor x\rfloor y)=f(x)\lfloor f(y)\rfloor \] where $\lfloor a\rfloor $ is greatest integer not greater than $a$.

\textbf{1304. (\color{red}d5\color{black}, 2006 IMOSL, A1)} A sequence of real numbers $ a_{0},\ a_{1},\ a_{2},\dots$ is defined by the formula
\[ a_{i + 1} = \lfloor a_{i}\rfloor\cdot \langle a_{i}\rangle\qquad\text{for}\quad i\geq 0;
\]here $a_0$ is an arbitrary real number, $\lfloor a_i\rfloor$ denotes the greatest integer not exceeding $a_i$, and $\langle a_i\rangle=a_i-\lfloor a_i\rfloor$. Prove that $a_i=a_{i+2}$ for $i$ sufficiently large.

\textbf{1249. (\color{red}d5\color{black}, 2022 Malaysian IMOTST, P4 of 6)} Given a positive integer $n$, suppose that $P(x,y)$ is a real polynomial such that
\[P(x,y)=\frac{1}{1+x+y} \hspace{0.5cm} \text{for all $x,y\in\{0,1,2,\dots,n\}$}.\] What is the minimum degree of $P$?

\textbf{1192. (\color{red}d5\color{black}, 2019 IMO, P1 of 6)} Let $\mathbb{Z}$ be the set of integers. Determine all functions $f : \mathbb{Z} \to \mathbb{Z}$ such that, for all integers $a$ and $b$,\[f(2a) + 2f(b) = f(f(a + b)).\]

\textbf{1178. (\color{red}d5\color{black}, 2012 USAMO, P1 of 6)} Find all integers $n \ge 3$ such that among any $n$ positive real numbers $a_1$, $a_2$, $\dots$, $a_n$ with \[\max(a_1, a_2, \dots, a_n) \le n \cdot \min(a_1, a_2, \dots, a_n),\] there exist three that are the side lengths of an acute triangle.

\textbf{1109. (\color{red}d5\color{black}, 2022 India EGMO TST, Day 1, P1 of 3 )} Let $n\ge 3$ be an integer, and suppose $x_1,x_2,\cdots ,x_n$ are positive real numbers such that $x_1+x_2+\cdots +x_n=1.$ Prove that$$x_1^{1-x_2}+x_2^{1-x_3}\cdots+x_{n-1}^{1-x_n}+x_n^{1-x_1}<2.$$

\textbf{1074. (\color{red}d5\color{black}, 2021 Spain MO, P4 of 6)} Let $a,b,c,d$ real numbers such that:

$$a+b+c+d=0 \text{ and } a^2+b^2+c^2+d^2 = 12$$
Find the minimum and maximum possible values for $abcd$, and determine for which values of $a,b,c,d$ the minimum and maximum are attained.

\textbf{1046. (\color{red}d5\color{black}, 2010 USAJMO, P2 of 6)} Let $n > 1$ be an integer.  Find, with proof, all sequences $x_1$, $x_2$, \dots, $x_{n-1}$ of positive integers with the following three properties: \begin{enumerate}
    \item $x_1 < x_2 < \cdots < x_{n-1}$;
    \item $x_i + x_{n-i} = 2n$ for all $i = 1, 2, \ldots , n - 1$;
    \item given any two indices $i$ and $j$ (not necessarily distinct)
          for which $x_i + x_j < 2n$,  there is an index $k$ such that $x_i + x_j = x_k$. \end{enumerate}

\textbf{1039. (\color{red}d5\color{black}, 1968, Putnam, A6)} For each positive integer $n \geq 1$, determine all monic polynomials of degree $n$ whose roots are all real, for which every coefficient is either $1$ or $-1$

\textbf{921. (\color{red}d5\color{black}, 2020 APMO, P2 of 5)} Show that \(r = 2\) is the largest real number \(r\) which satisfies the following condition:\\
If a sequence \(a_1, a_2, \dots\) of positive integers fulfills the inequalities \[a_n \leq a_{n+2} \leq  \sqrt{a_n^2 + ra_{n+1}}\] for every positive integer \(n\), then there exists a positive integer \(M\) such that \(a_{n+2} = a_n\) for every \(n \geq M\).

\textbf{900. (\color{red}d5\color{black}, 1996 IMOSL, A1)} Suppose that $a, b, c > 0$ such that $abc = 1$. Prove that \[ \frac{ab}{ab + a^5 + b^5} + \frac{bc}{bc + b^5 + c^5} + \frac{ca}{ca + c^5 + a^5} \leq 1. \]

\textbf{772. (\color{red}d5\color{black}, UK Training)} Prove that there is no positive integer $n$ such that there is positive integers $a, b, c, d$ with $ad = bc$ and $n^2 < a < b< c< d < (n + 1)^2$.

\textbf{744. (\color{red}d5\color{black}, 2005 IMC, P4 of 12)} Find all polynomials of degree $n$ whose coefficients are a permutation of $[1, 2, \dots, n]$ where all of the polynomial's roots are rational.

\textbf{709. (\color{red}d5\color{black}, 2007 USA TST, P2 of 6)} Let $n$ be a positive integer and let $a_1 \le a_2 \le \dots \le a_n$ and $b_1 \le b_2 \le \dots \le b_n$ be two nondecreasing sequences of real numbers such that

\[ a_1 + \dots + a_i \le b_1 + \dots + b_i  \text{ for every } i = 1, \dots, n \]

and

\[ a_1 + \dots + a_n = b_1 + \dots + b_n. \]

Suppose that for every real number $m$, the number of pairs $(i,j)$ with $a_i-a_j=m$ equals the numbers of pairs $(k,\ell)$ with $b_k-b_\ell = m$. Prove that $a_i = b_i$ for $i=1,\dots,n$.

\textbf{562. (\color{red}d5\color{black}, 2019 HMMT, Algebra/Number Theory P7 of 10)} Find the value of
\[\sum_{a = 1}^{\infty} \sum_{b = 1}^{\infty} \sum_{c = 1}^{\infty} \frac{ab(3a + c)}{4^{a+b+c} (a+b)(b+c)(c+a)}.\]

\textbf{492. (\color{red}d5\color{black}, 2018 IMOSL A1)} Let $\mathbb{Q}_{>0}$ denote the set of all positive rational numbers. Determine all functions $f:\mathbb{Q}_{>0}\to\mathbb{Q}_{>0}$ satisfying \begin{equation*}f(x^2f(y)^2) = f(x)^2f(y)\end{equation*} for all $x,y \in \mathbb{Q}_{>0}$.

\textbf{480. (\color{red}d5\color{black}, 2013 IMOSL, A1)} Let $n$ be a positive integer and let $a_1, \ldots, a_{n-1} $ be arbitrary real numbers. Define the sequences $u_0, \ldots, u_n $ and $v_0, \ldots, v_n $ inductively by $u_0 = u_1  = v_0 = v_1 = 1$, and $u_{k+1} = u_k + a_k u_{k-1}$, $v_{k+1} = v_k + a_{n-k} v_{k-1}$ for $k=1, \ldots, n-1.$

Prove that $u_n = v_n.$

\textbf{463. (\color{red}d5\color{black}, 2015 IMOSL A1)} Suppose that a sequence $a_1,a_2,\ldots$ of positive real numbers satisfies \[a_{k+1}\geq\frac{ka_k}{a_k^2+(k-1)}\]for every positive integer $k$. Prove that $a_1+a_2+\ldots+a_n\geq n$ for every $n\geq2$.

\textbf{360. (\color{red}d5\color{black}, 2004 IMOSL, A3)} Does there exist a function $s\colon \mathbb{Q} arrow \{-1,1\}$ such that if $x$ and $y$ are distinct rational numbers satisfying ${xy=1}$ or ${x+y\in \{0,1\}}$, then ${s(x)s(y)=-1}$? Justify your answer.

\textbf{263. (\color{red}d5\color{black}, 1996 Vietnam MO, Day 2, Q3 of 3)} Suppose $a,b,c$ are positive reals such that $ab+bc+ca+abc=4$. Prove that $a+b+c \geq ab+bc+ca$.

\textbf{262. (\color{red}d5\color{black}, 2020 ICMC, Round 1, P6 of 6)} Let $ r < 1/2 $ be a positive real number and let $ U_r $ denote the set of real numbers that differ from their nearest integer by at most $ r $. Prove that there exists a positive integer $ m $ such that for any real number $ x $, the sets $ \{x, 2x, 3x, \dots, mx\} $ and $ U_r$ have at least one element in common.

\textbf{260. (\color{red}d5\color{black}, 2019 BMO2, P4 of 4)} Find all functions $f$ from the set of positive real numbers to the positive real numbers for which $f(x) \leq f(y)$ whenever $x \leq y$ and \[f(x^4)+f(x^2)+f(x)+f(1) = x^4+x^2+x+1\] for all $x > 0$.

\textbf{173. (\color{red}d5\color{black}, 2007 Hungary-Israel MO Q6 of 6)} Let $t \geq 3$ be a real number and assume the polynomial $P(x)$ satisfies $$ | P(k) - t^k  | < 1$$ for $k = 0, 1, 2, \dots, n.$ Prove that the degree of $P$ is at least $n.$

\textbf{170. (\color{red}d5\color{black}, USSR Olympiad 1990, 11th Grade, Problem 7 of 8)} The following equation with erased coefficients is written on a blackboard:

\[x^3 + \cdots x^2 + \cdots x + \cdots = 0\]



Two players are playing a game. In one move the first player chooses a number and the second player puts it instead of dots into one of the vacant places. After three moves the game is over. (Note that the first player chooses a number in each of the three moves.) Is it possible for the first player to choose three numbers that will secure three distinct integer roots for the equation, no matter how the second player plays?

\textbf{165. (\color{red}d5\color{black}, 2017 IMOSL A2)} Let $q$ be a real number. Gugu has a napkin with ten distinct real numbers written on it, and he writes the following three lines of real numbers on the blackboard: \begin{enumerate} \item In the first line, Gugu writes down every number of the form $a-b$, where $a$ and $b$ are two (not necessarily distinct) numbers on his napkin. \item In the second line, Gugu writes down every number of the form $qab$, where $a$ and $b$ are two (not necessarily distinct) numbers from the first line. \item In the third line, Gugu writes down every number of the form $a^2+b^2-c^2-d^2$, where $a, b, c, d$ are four (not necessarily distinct) numbers from the first line. \end{enumerate}Determine all values of $q$ such that, regardless of the numbers on Gugu's napkin, every number in the second line is also a number in the third line.

\textbf{155. (\color{red}d5\color{black}, 2019 New Zealand MO)} Suppose that $x_1, x_2, x_3, \dots x_n$ are real numbers between 0 and 1 with sum $s$. Prove that \[\sum_{i=1}^n \frac{x_i}{s+1-x_i} + \prod_{i=1}^n (1-x_i) \leq 1\]

\textbf{125. (\color{red}d5\color{black}, 2018 USATST, P2 of 6)} Find all functions $f: \mathbb{Z}^2 \to [0, 1]$ such that for any integers $x$ and $y$, $$2f(x, y) = f(x - 1, y) + f(x, y - 1)$$

\textbf{119. (\color{red}d5\color{black}, 2019 China National Olympaid, P1 of 6)} Let \(a,b,c,d,e \geq -1\) and \(a+b+c+d+e=5.\) Find the maximum and minimum value of \(S=(a+b)(b+c)(c+d)(d+e)(e+a).\)

\textbf{110. (\color{red}d5\color{black}, 2018 Mathematical Ashes, P2)} Let \(\mathbb{N}_0\) denote the set of non-negative integers. Find all functions \(f:\mathbb{N}_0\to\mathbb{N}_0\) such that
\[f^{f(m)}(n) = n + 2f(m)\]
for all \(m,n \in \mathbb{N}_0\) with \(m \leq n\).

\textbf{105. (\color{red}d5\color{black}, 2006 Indian MO, Q5)} Let \(ABC\) be a triangle with sides \(a, b, c\), circumradius \(R\), and inradius \(r\). Prove that
\[\frac{R}{2r} \geq  ( \frac{64a^2b^2c^2}{(4a^2-(b-c)^2)(4b^2-(c-a)^2)(4c^2-(a-b)^2)}  )^2.\]

\textbf{65. (\color{red}d5\color{black}, 1995 IMO Q2 of 6)} Let $a$, $b$, $c$ be positive real numbers such that $abc=1$. Prove that \[\frac{1}{a^3(b+c)} + \frac{1}{b^3(c+a)} + \frac{1}{c^3(a+b)} \geq \frac{3}{2}.\]

\textbf{36. (\color{red}d5\color{black}, 1997 Balkan MO (14th), Q4)} Determine all functions $f : \mathbb{R} \to \mathbb{R}$ that satisfies $f(xf(x) + f(y)) = (f(x))^2 + y$ for all $x,y \in \mathbb{R}$.

\textbf{25. (\color{red}d5\color{black}, 1993 IMO (34th), Q1)} Let $n > 1$ be an integer and let $f(x) = x^n + 5x^{n-1} + 3$. Prove that there do not exist polynomials $g(x),h(x)$, each having integer coefficients and degree at least one, such that $f(x) = g(x)h(x)$.

\textbf{9. (\color{red}d5\color{black}, 2017 Canadian MO, Q2)} Define a function $f(n)$ from the positive integers to the positive integers such that $f(f(n))$ is the number of positive integer divisors of $n$. Prove that if $p$ is prime, then $f(p)$ is prime.

\textbf{1354. (\color{red}d6\color{black}, 1986 IMO, P5 of 6)} Find all functions $f$, defined on the non-negative real numbers and taking non-negative real values, such that: \begin{enumerate} \item $f(xf(y))f(y) = f(x+y)$ for all $x,y\ge 0$, \item $f(2) = 0$, \item $f(x) \neq 0$ for $0\le x < 2$. \end{enumerate}

\textbf{1333. (\color{red}d6\color{black}, 2013 IMOSL, A2)} Prove that for any set of $2000$ distinct real numbers there exist two pairs $a>b$ and $c>d$ with $a\neq c$ or $b\neq d$, such that $$|\frac{a-b}{c-d} - 1 | < \frac{1}{100000}.$$

\textbf{1307. (\color{red}d6\color{black}, 2004 IMC, Day 1 P2 of 6)} Let $P(x)=x^2-1$. How many distinct real solutions does the following equation have: \[\underbrace{P(P(\cdots (P}_{2004}(x))\cdots ))=0\ ?\]

\textbf{1228. (\color{red}d6\color{black}, 2000 Putnam)} Let $f(x)$ be a polynomial with integer coefficients. Define a sequence $a_0 , a_1, \cdots$ of integers such that $a_0 = 0$ and $a_{n+1} = f(a_n)$ for all $n \geq 0$. Prove that if there exists a positive integer $m$ for which $a_m = 0$ then either $a_1 = 0$ or $a_2 = 0$.


\textbf{1143. (\color{red}d6\color{black}, 2008 IMO, P4 of 6)} Find all functions $ f: (0, \infty) arrow (0, \infty)$ (so, $ f$ is a function from the positive real numbers to the positive real numbers) such that \[ \frac {\Big( f(w) \Big)^2 + \Big( f(x) \Big)^2}{f(y^2) + f(z^2) } = \frac {w^2 + x^2}{y^2 + z^2}\] for all positive real numbers \(w\), \(x\), \(y\), \(z\), satisfying $ wx = yz$.

\textbf{1131. (\color{red}d6\color{black}, 2012 BMO2, P3 of 4)} The set of real numbers is split into two subsets that do not intersect. Prove that for each pair $(m,n)$ of positive integers, there are real numbers $x<y<z$ all in the same subset such that $m(z-y)=n(y-x)$.

\textbf{1117. (\color{red}d6\color{black}, 2012 IMO, P4 of 6)} Find all functions $f:\mathbb Zarrow \mathbb Z$ such that, for all integers $a,b,c$ that satisfy $a+b+c=0$, the following equality holds:
\[f(a)^2+f(b)^2+f(c)^2=2f(a)f(b)+2f(b)f(c)+2f(c)f(a).\]
(Here $\mathbb{Z}$ denotes the set of integers.)

\textbf{1055. (\color{red}d6\color{black}, 2021 China TST, Test 1, P4 of 6)} Let $f(x), g(x)$ be two polynomials with integer coefficients. It is known that there are infinitely many primes $p$ for which there exists an integer $m_{p}$ such that
$$
    f(a) \equiv g(a+m_{p}) \pmod p
$$
holds for all $a \in \mathbb{Z}$. Prove that there exists a rational number $r$ such that
$$
    f(x)=g(x+r)
$$

\textbf{1041. (\color{red}d6\color{black}, 2021 Balkan MO, P2 of 4)} Find all functions \(f : (0, +\infty) \to (0, +\infty)\) such that \[f(x+f(x) + f(y)) = 2f(x) + y\] holds for all \(x,y \in (0, +\infty)\).

\textbf{1038. (\color{red}d6\color{black}, 2002 USAMO P 4 of 6)} Let $\mathbb{R}$ be the set of real numbers. Determine all functions $f : \mathbb{R} arrow \mathbb{R}$ such that \[f(x^2 - y^2) = xf(x) - yf(y)\] for all pairs of real numbers $x$ and $y$.

\textbf{1026. (\color{red}d6\color{black}, 2014 Putnam, B4)} Show that for each positive integer $n$, all the roots of the polynomial
\[ \sum_{k=0}^n 2^{k(n-k)} x^k \]
are real numbers.

\textbf{1019. (\color{red}d6\color{black}, Folklore)} For a prime $p$ show that $x^{p-1}+x^{p-2}+\cdots+x+1$ is irreducible over $\mathbb{Q}$

\textbf{871. (\color{red}d6\color{black}, 2020 IMOSL, A3)} Suppose that $a, b, c, d$ are positive real numbers satisfying $(a+c)(b+d)=a c+b d$. Find the smallest possible value of
$$
    S=\frac{a}{b}+\frac{b}{c}+\frac{c}{d}+\frac{d}{a}.
$$

\textbf{844. (\color{red}d6\color{black}, Adapted from Miklós Schweitzer 2020)} Let $S$ be the set of all infinite positive integer sequences. We say that two sequences in $S$ are \emph{separated} if, for all positive integers $n$, the $n$th term in the first sequence is distinct from the $n$th term in the second sequence. \\

Find all functions $f:S\to \mathbb{Z}^+$ such that:
\begin{enumerate}
    \item $f(s_1)\neq f(s_2)$ for any two separated sequences $s_1,s_2\in S$;
    \item For any constant sequence $s=(c,c,c,\dots)\in S$, we have $f(s)=c$.
\end{enumerate}

\textbf{557. (\color{red}d6\color{black}, Banach fixed-point theorem)} Consider a function $f: \mathbb{R}^2 \to \mathbb{R}^2$ which sends points in the Euclidean plane to points in the Euclidean plane. Define by $d(X, Y)$ the Euclidean distance between points $X$ and $Y.$ Suppose that for each pair of points $X, Y \in \mathbb{R}^2$ the points $f(X)$ and $f(Y)$ satisfy the inequality $$d(f(X), f(Y)) \leq 0.999 \hspace{0.1cm} d(X, Y).$$ Show that there is a unique point $P \in \mathbb{R}^2$ that satisfies $f(P) = P.$

\textbf{479. (\color{red}d6\color{black}, 2009 IMO, P5 of 6)} Determine all functions $ f$ from the set of positive integers to the set of positive integers such that, for all positive integers $ a$ and $ b$, there exists a non-degenerate triangle with sides of lengths
\[ a, f(b) \text{ and } f(b + f(a) - 1).\]
(A triangle is non-degenerate if its vertices are not collinear.)

\textbf{458. (\color{red}d6\color{black}, 2018 Balkan MO, P2 of 4)} Let $q$ be a positive rational number. Two ants are initially at the same point $X$ in the plane. In the $n$-th minute $(n = 1,2,...)$ each of them chooses whether to walk due north, east, south or west and then walks the distance of $q^n$ metres. After a whole number of minutes, they are at the same point in the plane (not necessarily $X$), but have not taken exactly the same route within that time. Determine all possible values of $q$.

\textbf{423. (\color{red}d6\color{black}, 2007 IMO, P1 of 6)} Real numbers $ a_{1}, a_{2}, \ldots, a_{n}$ are given. For each $i$ $ (1 \leq i \leq n )$ define
\[d_{i} = \max \{ a_{j} : 1 \leq j \leq i \} - \min \{ a_{j} : i \leq j \leq n \}\]
and let \[d = \max \{d_{i} : 1 \leq i \leq n \}\].
\renewcommand{\labelenumi}{\theenumi}
\renewcommand{\theenumi}{(\alph{enumi})}
\begin{enumerate}
    \item Prove that, for any real numbers $ x_{1}\leq x_{2}\leq \cdots \leq x_{n}$,
          \[\max \{ |x_{i} - a_{i}| : 1 \leq i \leq n \}\geq \frac {d}{2} \qquad \qquad (*)\]
    \item Show that there are real numbers $ x_{1}\leq x_{2}\leq \cdots \leq x_{n}$ such that the equality holds in \((*)\).
\end{enumerate}

\textbf{422. (\color{red}d6\color{black}, 2002 France TST P3)} Let $n$ be a positive integer and let $(a_1, a_2, \dots , a_{2n})$ be a permutation of $1, 2, \dots , 2n$ such that the numbers $\lvert a_{i + 1} - a_i \rvert$ are pairwise distinct for $i = 1, 2, \dots , 2n-1.$ Prove that $\{ a_2, a_4, \dots , a_{2n} \}  = \{1, 2, \dots , n\}$ if and only if $a_1 - a_{2n} = n.$

\textbf{410. (\color{red}d6\color{black}, 2014 France TST P4)} Let $n$ be a positive integer and $x_1,x_2,\ldots,x_n$ be positive reals. Show that there are numbers $a_1,a_2,\ldots, a_n \in \{-1,1\}$ such that the following holds:
\[a_1x_1^2+a_2x_2^2+\cdots+a_nx_n^2 \ge (a_1x_1+a_2x_2 +\cdots+a_nx_n)^2\]

\textbf{409. (\color{red}d6\color{black}, 2019 BMO1, P6 of 6)} A function $f$ is called $good$ if it assigns an integer value $f(m, n)$ to every ordered pair of integers $(m, n)$ in such a way that for every pair of integers $(m, n)$ we have:
$$2f(m, n) = f(m-n, n-m) + m + n = f(m+1, n) + f(m, n+1) - 1.$$
Find all good functions.

\textbf{389. (\color{red}d6\color{black}, 2020 Hong Kong TST, P4 of 4 )} Find all real-valued functions $f$ defined on the set of real numbers such that$$f(f(x)+y)+f(x+f(y))=2f(xf(y))$$for any real numbers $x$ and $y$.

\textbf{381. (\color{red}d6\color{black}, 2015 IMOSL, A2)} Determine all functions $f:\mathbb{Z}arrow\mathbb{Z}$ with the property that \[f(x-f(y))=f(f(x))-f(y)-1\]holds for all $x,y\in\mathbb{Z}$.

\textbf{319. (\color{red}d6\color{black}, 1987 USAMO P3)} Construct a set $S$ of polynomials inductively by the rules:

(i) $x\in S$;
(ii) if $f(x)\in S$, then $xf(x)\in S$ and $x+(1-x)f(x)\in S$.

Prove that there are no two distinct polynomials in $S$ whose graphs intersect within the region $\{0 < x < 1\}$.

\textbf{285. (\color{red}d6\color{black}, 2005 China TST Day 2 Q3 of 3)} Let $\alpha$ be a given positive real number. Find all functions $f: \mathbb{N} \to \mathbb{R}$ such that $$f(k + m) = f(k) + f(m)$$ holds for any positive integers $k, m$ satisfying $$\alpha m \leq k \leq (\alpha + 1)m.$$

\textbf{284. (\color{red}d6\color{black}, 2019 Czech-Polish-Slovak, Q4 of 6)} Given a real number $\alpha$, find all pairs $(f,g)$ of functions $f,g :\mathbb{R} \to \mathbb{R}$ such that $\forall x, y \in \mathbb{R}$, $$xf(x+y)+\alpha \cdot yf(x-y)=g(x)+g(y)$$

\textbf{270. (\color{red}d6\color{black}, 1998 Romania TST, Day 3, Q3 of 3)} Show that for any positive integer $n$ the polynomial $f(x)=(x^2+x)^{2^n}+1$ cannot be decomposed into the product of two integer non-constant polynomials.

\textbf{207. (\color{red}d6\color{black}, 2003 IMOSL A3)} Consider pairs of the sequences of positive real numbers \[a_1\geq a_2\geq a_3\geq\cdots,\qquad b_1\geq b_2\geq b_3\geq\cdots\] and the sums \[A_n = a_1 + \cdots + a_n,\quad B_n = b_1 + \cdots + b_n;\qquad n = 1,2,\ldots.\] For any pair define $c_n = \min\{a_i,b_i\}$ and $C_n = c_1 + \cdots + c_n$, $n=1,2,\ldots$.
(1) Does there exist a pair $(a_i)_{i\geq 1}$, $(b_i)_{i\geq 1}$ such that the sequences $(A_n)_{n\geq 1}$ and $(B_n)_{n\geq 1}$ are unbounded while the sequence $(C_n)_{n\geq 1}$ is bounded?
(2) Does the answer to question (1) change by assuming additionally that $b_i = 1/i$, $i=1,2,\ldots$?
Justify your answer.

\textbf{191. (\color{red}d6\color{black}, 2014 IMO, P1 of 6)} Let \(a_0 < a_1 < a_2 < \cdots\) be an infinite sequence of positive integers. Prove that there exists a unique integer \(n \geq 1\) such that \[a_n < \frac{a_0 + a_1 + \cdots + a_n}{n} \leq a_{n+1}.\]

\textbf{136. (\color{red}d6\color{black}, 2019 Japan MO Finals P3 of 5)} Find all functions $f:\mathbb R^{+} arrow \mathbb R^{+}$ such that

$$f(\frac{f(y)}{f(x)}+1)=f(x+\frac{y}{x}+1)-f(x)$$
for all $x,\ y\in\mathbb{R^{+}}$.

\textbf{23. (\color{red}d6\color{black}, 2018 Euclid Contest, Q10 (adapted))} In an infinite grid with two rows, each row continues to the left and right without bound. Each cell contains a positive real number. Prove that if each cell is the average of its three neighbours, then all the numbers in the grid are equal.

\textbf{1363. (\color{red}d7\color{black}, 2023 Israel MO, P6 of 7)} Determine if there exists a set $S$ of $5783$ different real numbers with the following property:
For every $a,b\in S$ (not necessarily distinct) there are $c\neq d$ in $S$ so that $a\cdot b=c+d$.

\textbf{1320. (\color{red}d7\color{black}, 2012 IMO, P2 of 6)} Let $n\ge 3$ be an integer, and let $a_2,a_3,\ldots ,a_n$ be positive real numbers such that $a_{2}a_{3}\cdots a_{n}=1$. Prove that
\[(1 + a_2)^2 (1 + a_3)^3 \dotsm (1 + a_n)^n > n^n.\]

\textbf{1229. (\color{red}d7\color{black}, 2020 RMM, P4 of 6)} Let $\mathbb N$ be the set of all positive integers. A subset $A$ of $\mathbb N$ is sum-free if, whenever $x$ and $y$ are (not necessarily distinct) members of $A$, their sum $x+y$ does not belong to $A$. Determine all surjective functions $f:\mathbb N\to\mathbb N$ such that, for each sum-free subset $A$ of $\mathbb N$, the image $\{f(a):a\in A\}$ is also sum-free.

\textit{Note: a function $f:\mathbb N\to\mathbb N$ is surjective if, for every positive integer $n$, there exists a positive integer $m$ such that $f(m)=n$.}

\textbf{1195. (\color{red}d7\color{black}, 2022 IZHO, P6 of 6)} Do there exist two bounded sequences $a_1,a_2,\dots $ and $b_1,b_2,\dots$ such that for each positive integers $n$ and $m>n$ at least one of the two inequalities $|a_m-a_n|>\frac{1}{\sqrt{n}}, |b_m-b_n|>\frac{1}{\sqrt{n}}$ holds?

\textbf{1132. (\color{red}d7\color{black}, 2013 RMM, P2 of 6)} Does there exist a pair $(g,h)$ of functions $g,h: \bR\to \bR$ such that the only function $f:\bR\to \bR$ satisfying $f(g(x))=g(f(x))$ and $f(h(x))=h(f(x))$ for all $x\in \bR$ is the identity function $f(x)\equiv x$?

\textbf{1124. (\color{red}d7\color{black}, 2012 IMOSL, A2)} Let $\mathbb{Z}$ and $\mathbb{Q}$ be the sets of integers and rationals respectively.
\begin{enumerate}
    \item Does there exist a partition of $\mathbb{Z}$ into three non-empty subsets $A$, $B$, $C$ such that the sets $A+B$, $B+C$, $C+A$ are disjoint?
    \item Does there exist a partition of $\mathbb{Q}$ into three non-empty subsets $A$, $B$, $C$ such that the sets $A+B$, $B+C$, $C+A$ are disjoint?
\end{enumerate}

Here $X+Y$ denotes the set $\{ x+y \mid x \in X, y \in Y \}$, for $X,Y \subseteq \mathbb{Z}$ and $X,Y \subseteq \mathbb{Q}$.

\textbf{1054. (\color{red}d7\color{black}, 2011 IMOSL, A3)} Determine all pairs $(f,g)$ of functions from the set of real numbers to itself that satisfy \[g(f(x+y)) = f(x) + (2x + y)g(y)\] for all real numbers $x$ and $y$.

\textbf{1005. (\color{red}d7\color{black}, 2019 India TST, Day 4, P1 of 3)} Determine all non-constant monic polynomials $f(x)$ with integer coefficients for which there exists a natural number $M$ such that for all $n \geq M$, $f(n)$ divides $f(2^n) - 2^{f(n)}$.

\textbf{950. (\color{red}d7\color{black}, 2014 IMOSL, A4)} Determine all functions $f: \mathbb{Z}\to\mathbb{Z}$ satisfying
\[ f\big(f(m)+n\big)+f(m)=f(n)+f(3m)+2014 \]
for all integers $m$ and $n$.




\textbf{824. (\color{red}d7\color{black}, 2020 ASC, P5 of 5)} Let $\mathbb{R}^{+}$ be the set of positive real numbers. Determine all functions $f: \mathbb{R}^{+} arrow \mathbb{R}^{+}$ such that $$ f(x^{f(y)})=f(x)^{y} $$ for all positive real numbers $x$ and $y$.

\textbf{823. (\color{red}d7\color{black}, 2020 RMM, P2 of 6)} Let $N \geq 2$ be an integer, and let $\mathbf a$ $= (a_1, \ldots, a_N)$ and $\mathbf b$ $= (b_1, \ldots b_N)$ be sequences of non-negative integers. For each integer $i \not \in \{1, \ldots, N\}$, let $a_i = a_k$ and $b_i = b_k$, where $k \in \{1, \ldots, N\}$ is the integer such that $i-k$ is divisible by $n$. We say $\mathbf a$ is $\mathbf b$-harmonic if each $a_i$ equals the following arithmetic mean:\[a_i = \frac{1}{2b_i+1} \sum_{s=-b_i}^{b_i} a_{i+s}.\]Suppose that neither $\mathbf a $ nor $\mathbf b$ is a constant sequence, and that both $\mathbf a$ is $\mathbf b$-harmonic and $\mathbf b$ is $\mathbf a$-harmonic.

Prove that at least $N+1$ of the numbers $a_1, \ldots, a_N,b_1, \ldots, b_N$ are zero.

\textbf{817. (\color{red}d7\color{black}, 2020 USEMO, P4 of 6)} A function $f$ from the set of positive real numbers to itself satisfies $$f(x + f(y) + xy) = xf(y) + f(x + y)$$for all positive real numbers $x$ and $y$. Prove that $f(x) = x$ for all positive real numbers $x$.

\textbf{333. (\color{red}d7\color{black}, 2016 USA TSTST, Q3 of 3)} Decide whether or not there exists a nonconstant polynomial $Q(x)$ with integer coefficients with the following property: for every positive integer $n > 2$, the numbers\[ Q(0), \; Q(1), Q(2),  \; \dots, \; Q(n-1) \]produce at most $0.499n$ distinct residues when taken modulo $n$.

\textbf{326. (\color{red}d7\color{black}, 2013 IMO Q5)} Let $\mathbb Q_{>0}$ be the set of all positive rational numbers. Let $f:\mathbb Q_{>0}\to\mathbb R$ be a function satisfying the following three conditions:

(i) for all $x,y\in\mathbb Q_{>0}$, we have $f(x)f(y)\geq f(xy)$;
(ii) for all $x,y\in\mathbb Q_{>0}$, we have $f(x+y)\geq f(x)+f(y)$;
(iii) there exists a rational number $a>1$ such that $f(a)=a$.

Prove that $f(x)=x$ for all $x\in\mathbb Q_{>0}$.

\textbf{257. (\color{red}d7\color{black}, 2018 IMO, P2 of 6)} Find all integers \(n \geq 3\) for which there exist real numbers \(a_1, a_2, \dots, a_{n+2}\) such that \(a_{n+1} = a_1\) and \(a_{n+2} = a_2\), and \[a_ia_{i+1} + 1 = a_{i+2}\] for \(i = 1, 2, \dots, n\).

\textbf{255. (\color{red}d7\color{black}, 2017 IMOSL A4)} A sequence of real numbers \(a_1, a_2, \dots\) satisfies the relation \[a_n = - \max_{i+j=n} ( a_i + a_j ) \qquad \text{ for all } n > 2017.\]Prove that this sequence is bounded, i.e., there is a constant \(M\) such that \(  \lvert a_n  \rvert \leq M\) for all positive integers \(n\).

\textbf{214. (\color{red}d7\color{black}, Putnam B5)} Find all functions $f$ from the interval $(1,\infty)$ to $(1,\infty)$ with the following property: if $x,y\in(1,\infty)$ and $x^2\le y\le x^3,$ then $(f(x))^2\le f(y) \le (f(x))^3.$

\textbf{158. (\color{red}d7\color{black}, 2012 APMO, Q5)} Let $ n $ be an integer greater than or equal to $ 2 $. Prove that if the real numbers $ a_1 , a_2 , \cdots , a_n $ satisfy $ a_1 ^2 + a_2 ^2 + \cdots + a_n ^ 2 = n $, then

\[\sum_{1 \le i < j \le n} \frac{1}{n- a_i a_j}  \le \frac{n}{2} \]

must hold.

\textbf{1279. (\color{red}d8\color{black}, 2022 USAMO, P5 of 6)} A function $f: \mathbb{R}\to \mathbb{R}$ is \textit{essentially increasing} if $f(s)\leq f(t)$ holds whenever $s\leq t$ are real numbers such that $f(s)\neq 0$ and $f(t)\neq 0$. \\\\ Find the smallest integer $k$ such that for any 2022 real numbers $x_1,x_2,\ldots , x_{2022},$ there exist $k$ essentially increasing functions $f_1,\ldots, f_k$ such that \[ f_1(n) + f_2(n) + \cdots + f_k(n) = x_n\qquad \text{for every } n= 1,2,\ldots 2022. \]

\textbf{1251. (\color{red}d8\color{black}, 2022 Israel TST10 P2 of 3)} Let $f: \bZ^2\to \bR$ be a function.  It is known that for any integer $C$ the four functions of $x$ \[f(x,C), \quad f(C,x),\quad f(x, C+x), \quad f(x,C-x)\] are polynomials of degree at most $100$. Prove that $f$ is equal to a polynomial in two variables $x,y$, and find its maximum possible degree.\\ \\ \textit{Remark: The degree of a bivariate polynomial $P(x,y)$ is defined as the maximal value of $i+j$ over all monomials $x^iy^j$ appearing in $P$ with a non-zero coefficient.}

\textbf{1223. (\color{red}d8\color{black}, 2021 IMOSL, A5)} Let $n\geq 2$ be an integer and let $a_1, a_2, \ldots, a_n$ be positive real numbers with sum $1$. Prove that
\[ \sum_{k=1}^n \frac{a_k}{1-a_k}(a_1+a_2+\cdots+a_{k-1})^2 < \frac{1}{3}. \]

\textbf{1188. (\color{red}d8\color{black}, 2020 IMO, P2 of 6)} The real numbers $a,b,c,d$ satisfy $a\geq b\geq c\geq d>0$ and $a+b+c+d=1$. Prove that \[(a+2b+3c+4d)a^ab^bc^cd^d<1.\]

\textbf{1125. (\color{red}d8\color{black}, 2022 ToT Spring Round, Senior A-level P4 of 7)} A polynomial of degree 2022 with integer coefficients and leading coefficient $1$ is given. What is the maximum possible number of roots of the polynomial in the segment $(0,1)$?

\textbf{963. (\color{red}d8\color{black}, 2014 USAMO, P3 of 6)} Prove that there exists an infinte set of points \[ \cdots , \; P_{-3}, \; P_{-2},\; P_{-1},\; P_0,\; P_1,\; P_2,\; P_3,\; \dots \] in the plane with the following property : For any three distinct integers $a$, $b$ and $c$, points $P_a$, $P_b$ and $P_c$ are collinear if and only if $a+b+c=2014$.




\textbf{908. (\color{red}d8\color{black}, Folklore)} Consider two identical blocks of mass $1$kg, initially at $0^\circ$ and $100^\circ$ degrees respectively. You are allowed to cut the blocks however you want and touch whatever pieces against each other. Heat transfer happens instantenously without any heat loss. At the end of the process both blocks must be in their original configuration. How hot can you get the first block?

\textbf{880. (\color{red}d8\color{black}, 2021 Taiwan TST Round 2, P5 of 6)} Let $\|x\|_*=(|x|+|x-1|-1)/2$. Find all $f:\mathbb{N}\to\mathbb{N}$ such that \[f^{(\|f(x)-x\|_*)}(x)=x, \quad\forall x\in\mathbb{N}.\]

\textbf{873. (\color{red}d8\color{black}, 2020 IMOSL, A5)} A magician intends to perform the following trick. She announces a positive integer $n$, along with $2n$ real numbers $x_1 < \dots < x_{2n}$, to the audience. A member of the audience then secretly chooses a polynomial $P(x)$ of degree $n$ with real coefficients, computes the $2n$ values $P(x_1), \dots , P(x_{2n})$, and writes down these $2n$ values on the blackboard in non-decreasing order. After that the magician announces the secret polynomial to the audience. Can the magician find a strategy to perform such a trick?

\textbf{767. (\color{red}d8\color{black}, 2020 USA TSTST, P7 of 9)} Find all nonconstant polynomials $P(z)$ with complex coefficients for which all complex roots of the polynomials $P(z)$ and $P(z) - 1$ have absolute value 1.

\textbf{740. (\color{red}d8\color{black}, 2017 Putnam, A5)} Each of the integers from $1$ to $n$ is written on a separate card, and then the cards are combined into a deck and shuffled. Three players, $A,B,$ and $C,$ take turns in the order $A,B,C,A,\dots$ choosing one card at random from the deck. (Each card in the deck is equally likely to be chosen.) After a card is chosen, that card and all higher-numbered cards are removed from the deck, and the remaining cards are reshuffled before the next turn. Play continues until one of the three players wins the game by drawing the card numbered $1.$

Show that for each of the three players, there are arbitrarily large values of $n$ for which that player has the highest probability among the three players of winning the game.

\textbf{655. (\color{red}d8\color{black}, 2013 USA TSTST, P2 of 9)} A finite sequence of integers $a_{1}, a_{2}, \ldots, a_{n}$ is called regular if there exists a real number $x$ satisfying
$$\lfloor k x\rfloor=a_{k} \quad \text { for } 1 \leq k \leq n$$
Given a regular sequence $a_{1}, a_{2}, \ldots, a_{n}$, for $1 \leq k \leq n$ we say that the term $a_{k}$ is forced if the following condition is satisfied:
the sequence
$$a_{1}, a_{2}, \ldots, a_{k-1}, b$$
is regular if and only if $b=a_{k}$. Find the maximum possible number of forced terms in a regular sequence with 1000 terms.

\textbf{635. (\color{red}d8\color{black}, 2016 USA TST, P3 of 6)} Let $p$ be a prime number. Let $\mathbb F_p$ denote the integers modulo $p$, and let $\mathbb F_p[x]$ be the set of polynomials with coefficients in $\mathbb F_p$. Define $\Psi : \mathbb F_p[x] \to \mathbb F_p[x]$ by\[ \Psi( \sum_{i=0}^n a_i x^i ) = \sum_{i=0}^n a_i x^{p^i}. \]Prove that for nonzero polynomials $F,G \in \mathbb F_p[x]$,\[ \Psi(\gcd(F,G)) = \gcd(\Psi(F), \Psi(G)). \]Here, a polynomial $Q$ divides $P$ if there exists $R \in \mathbb F_p[x]$ such that $P(x) - Q(x) R(x)$ is the polynomial with all coefficients $0$ (with all addition and multiplication in the coefficients taken modulo $p$), and the gcd of two polynomials is the highest degree polynomial with leading coefficient $1$ which divides both of them. A non-zero polynomial is a polynomial with not all coefficients $0$. As an example of multiplication, $(x+1)(x+2)(x+3) = x^3+x^2+x+1$ in $\mathbb F_5[x]$.

\textbf{599. (\color{red}d8\color{black}, Folklore)} Suppose $f:\mathbb{R}^2 \to \mathbb{R}$ is a function with the property that for every fixed $y_0 \in \mathbb{R}$. $f(x, y_0)$ is a real polynomial in $x$, and for every fixed $x_0 \in \mathbb{R}$. $f(x_0, y)$ is a real polynomial in $y$. Must $f$ necessarily be a polynomial in $x$ and $y$?

\textbf{586. (\color{red}d8\color{black}, 2018 IMOSL, A6)} Let $m, n \geq 2$ be integers. Let $f(x_1,\dots, x_n)$ be a polynomial with real coefficients such that $$f(x_1,\dots, x_n)=\lfloor \frac{x_1+\dots + x_n}{m} \rfloor$$ $$\text{ for every } x_1,\dots, x_n\in \{0,\dots, m-1\}.$$ Prove that the total degree of $f$ is at least $n.$

\textbf{564. (\color{red}d8\color{black}, 2017 IMOSL, A5)} An integer $n \geq 3$ is given. We call an $n$-tuple of real numbers $(x_1, x_2, \dots, x_n)$ Shiny if for each permutation $y_1, y_2, \dots, y_n$ of these numbers, we have

\[\sum \limits_{i=1}^{n-1} y_i y_{i+1} = y_1y_2 + y_2y_3 + y_3y_4 + \cdots + y_{n-1}y_n \geq -1.\]Find the largest constant $K = K(n)$ such that

\[\sum \limits_{1 \leq i < j \leq n} x_i x_j \geq K\]holds for every Shiny $n$-tuple $(x_1, x_2, \dots, x_n)$.

\textbf{487. (\color{red}d8\color{black}, 2016 IMO, Q5 of 6)} The equation
$$(x-1)(x-2)\cdots(x-2016)=(x-1)(x-2)\cdots (x-2016)$$is written on the board, with $2016$ linear factors on each side. What is the least possible value of $k$ for which it is possible to erase exactly $k$ of these $4032$ linear factors so that at least one factor remains on each side and the resulting equation has no real solutions?

\textbf{417. (\color{red}d8\color{black}, 2018 APMO, Q5 of 5)} Find all polynomials $P(x)$ with integer coefficients such that for all real numbers $s$ and $t$, if $P(s)$ and $P(t)$ are both integers, then $P(st)$ is also an integer.

\textbf{368. (\color{red}d8\color{black}, 2003 IMOSL, A5)} Let $\mathbb{R}^+$ be the set of all positive real numbers. Find all functions $f: \mathbb{R}^+ \to \mathbb{R}^+$ that satisfy the following conditions:
\begin{enumerate}
    \item[(i)] $f(xyz)+f(x)+f(y)+f(z)=f(\sqrt{xy})f(\sqrt{yz})f(\sqrt{zx})$ for all $x,y,z\in\mathbb{R}^+$
    \item[(ii)] $f(x)<f(y)$ for all $1\le x<y$.
\end{enumerate}

\textbf{250. (\color{red}d8\color{black}, Kronecker)} Let P(x) be a monic integer polynomial such that all roots have absolute value 1. Prove that all roots are roots of unity.

\textbf{72. (\color{red}d8\color{black}, 2018 NZ Camp Selection Problems, Q10 of 10)} Find all functions $f : \mathbb{R} \to \mathbb{R}$ such that \[f(x)f(y) = f(xy+1)+f(x-y)-2\] for all $x, y \in \mathbb{R}.$

\textbf{1342. (\color{red}d9\color{black}, 9th EMC, Senior league, P4 of 4)} Let $\mathbb{R^+}$ denote the set of all positive real numbers. Find all functions $f: \mathbb{R^+}arrow \mathbb{R^+}$ such that
$$xf(x + y) + f(xf(y) + 1) = f(xf(x))$$for all $x, y \in\mathbb{R^+}.$

\textbf{1098. (\color{red}d9\color{black}, USA TST 2007 P3)} Let $ \theta$ be an angle in the interval $ (0,\pi/2)$. Given that $ \cos \theta$ is irrational, and that $ \cos k \theta$ and $ \cos[(k + 1)\theta ]$ are both rational for some positive integer $ k$, show that $ \theta = \pi/6$.

\textbf{1091. (\color{red}d9\color{black}, Space-filling curve)} Say a function $f : \mathbb{R} \to \mathbb{R}^2$ is \emph{continuous} if for every $\epsilon > 0$ and $x \in \mathbb{R}$ there exists $\delta > 0$ such that $|x - y| < \delta$ implies $|f(x) - f(y)| < \epsilon$. Does there exist a continuous surjection from $\mathbb{R}$ to $\mathbb{R}^2$?

\textbf{1076. (\color{red}d9\color{black}, 2021 APMO, P5 of 5)} Determine all functions $f:\mathbb{Z} \to \mathbb{Z}$ such that \[f(f(a)-b)+bf(2a)\] is a perfect square for all integers $a$ and $b$.

\textbf{1027. (\color{red}d9\color{black}, Gauss (1801))} Express $\cos(\frac{2\pi}{17})$ in terms of radicals.

\textbf{1006. (\color{red}d9\color{black}, 2018 China TST Test 2, P6 of 6)} Let $M,a,b,r$ be non-negative integers with $a,r\ge 2$, and suppose there exists a function $f:\mathbb{Z}arrow\mathbb{Z}$ satisfying the following conditions: \begin{enumerate} \item For all $n\in \mathbb{Z}$, $f^{(r)}(n)=an+b$ where $f^{(r)}$ denotes the composition of $r$ copies of $f$ \item For all $n\ge M$, $f(n)\ge 0$ \item For all $n>m>M$, $n-m \mid f(n)-f(m)$ \end{enumerate} Show that $a$ is a perfect $r$-th power.

\textbf{957. (\color{red}d9\color{black}, 2014 IMOSL, A5)} Consider all polynomials $P(x)$ with real coefficients that have the following property: for any two real numbers $x$ and $y$ one has\[|y^2-P(x)|\le 2|x|\quad\text{if and only if}\quad |x^2-P(y)|\le 2|y|.\]Determine all possible values of $P(0)$.

\textbf{916. (\color{red}d9\color{black}, 2019 IMOSL, A5)} Let $x_1, x_2, \dots, x_n$ be different real numbers. Prove that \[\sum_{1 \leq i \leq n} \prod_{j \neq i} \frac{1-x_{i}x_{j}}{x_{i}-x_{j}}=\{\begin{array}{ll} 0, & \text { if } n \text { is even; } \\ 1, & \text { if } n \text { is odd. } \end{array}.\]

\textbf{881. (\color{red}d9\color{black}, 2021 IMO, P6 of 6)} Let $m\ge 2$ be an integer, $A$ a finite set of integers (not necessarily positive) and $B_1,B_2,...,B_m$ subsets of $A$. Suppose that, for every $k=1,2,...,m$, the sum of the elements of $B_k$ is $m^k$. Prove that $A$ contains at least $\dfrac{m}{2}$ elements.

\textbf{860. (\color{red}d9\color{black}, 2020 IMOSL, A6)} Find all functions $f : \mathbb{Z}arrow \mathbb{Z}$ satisfying \[f^{a^{2} + b^{2}}(a+b) = af(a) +bf(b)\] for all integers $a$ and $b.$

\textbf{769. (\color{red}d9\color{black}, 2021 USA TST, P3)} Find all functions $f \colon \mathbb{R} \to \mathbb{R}$ that satisfy the inequality
\[ f(y) - (\frac{z-y}{z-x} f(x) + \frac{y-x}{z-x}f(z)) \leq f(\frac{x+z}{2}) - \frac{f(x)+f(z)}{2} \]
for all real numbers $x < y < z$.

\textbf{734. (\color{red}d9\color{black}, 2019 IMOSL, A7)} Let $\mathbb Z$ be the set of integers. We consider functions $f :\mathbb Z\to\mathbb Z$ satisfying
\[f(f(x+y)+y)=f(f(x)+y)\]for all integers $x$ and $y$. For such a function, we say that an integer $v$ is f-rare if the set
\[X_v=\{x\in\mathbb Z:f(x)=v\}\]is finite and nonempty.
(a) Prove that there exists such a function $f$ for which there is an $f$-rare integer.
(b) Prove that no such function $f$ can have more than one $f$-rare integer.

\textbf{712. (\color{red}d9\color{black}, 2020 CMC, P8 of 8)} Let $a_{1}, a_{2}, \ldots$ be an infinite sequence of positive real numbers such that for each positive integer $n$ we have $$\frac{a_{1}+a_{2}+\cdots+a_{n}}{n} \geq \sqrt{\frac{a_{1}^{2}+a_{2}^{2}+\cdots+a_{n+1}^{2}}{n+1}}$$ Prove that the sequence $a_{1}, a_{2}, \ldots$ is constant.

\textbf{663. (\color{red}d9\color{black}, 2010 IMO, P6 of 6)} Let $a_1, a_2, a_3, \ldots$ be a sequence of positive real numbers, and $s$ be a positive integer, such that
\[a_n = \max \{ a_k + a_{n-k} \mid 1 \leq k \leq n-1 \} \ \textrm{ for all } \ n > s.\]
Prove there exist positive integers $\ell \leq s$ and $N$, such that
\[a_n = a_{\ell} + a_{n - \ell} \ \textrm{ for all } \ n \geq N.\]


\textbf{642. (\color{red}d9\color{black}, 2011 USA TST, P6 of 9)} A polynomial $P(x)$ is called nice if $P(0) = 1$ and the nonzero coefficients of $P(x)$ alternate between $1$ and $-1$ when written in order. Suppose that $P(x)$ is nice, and let $m$ and $n$ be two relatively prime positive integers. Show that \[Q(x) = P(x^n) \cdot \frac{(x^{mn} - 1)(x-1)}{(x^m-1)(x^n-1)}\] is nice as well.

\textbf{614. (\color{red}d9\color{black}, 2019 IMOSL, A6)} A polynomial $P(x, y, z)$ in three variables with real coefficients satisfies the identities
\smallbreak
$$P(x, y, z)=P(x, y, xy-z)=P(x, zx-y, z)=P(yz-x, y, z).$$
Prove that there exists a polynomial $F(t)$ in one variable such that
\smallbreak
$$P(x,y,z)=F(x^2+y^2+z^2-xyz).$$

\textbf{593. (\color{red}d9\color{black}, 2012 IMOSL, A7)} We say that a function $f:\mathbb{R}^k arrow \mathbb{R}$ is a metapolynomial if, for some positive integers $m$ and $n$, it can be represented in the form
\[f(x_1,\cdots , x_k )=\max_{i=1,\cdots , m} \min_{j=1,\cdots , n}P_{i,j}(x_1,\cdots , x_k),\]
where $P_{i,j}$ are multivariate polynomials. Prove that the product of two metapolynomials is also a metapolynomial.

\textbf{522. (\color{red}d9\color{black}, 2013 IMOSL, A5)} Let $\mathbb{Z}_{\ge 0}$ be the set of all nonnegative integers. Find all the functions $f: \mathbb{Z}_{\ge 0} arrow \mathbb{Z}_{\ge 0} $ satisfying the relation
\[ f(f(f(n))) = f(n+1 ) +1 \]
for all $ n\in \mathbb{Z}_{\ge 0}$.

\textbf{502. (\color{red}d9\color{black}, 2013 ELMO, P6 of 6)} Consider a function $f: \mathbb Z \to \mathbb Z$ such that for every integer $n \ge 0$, there are at most $0.001n^2$ pairs of integers $(x,y)$ for which $f(x+y) \neq f(x)+f(y)$ and $\max\{ \lvert x \rvert, \lvert y \rvert \} \le n$. Is it possible that for some integer $n \ge 0$, there are more than $n$ integers $a$ such that $f(a) \neq a \cdot f(1)$ and $\lvert a \rvert \le n$?

\textbf{453. (\color{red}d9\color{black}, 2011 USA TST, P9 of 9)} Determine whether or not there exist two different sets $A,B$, each consisting of at most $2011^2$ positive integers, such that every $x$ with $0 < x < 1$ satisfies the following inequality:
\[| \sum_{a \in A} x^a - \sum_{b \in B} x^b | < (1-x)^{2011}.\]

\textbf{446. (\color{red}d9\color{black}, 2000 IMO, P3 of 6)} Let $n \geq 2$ be a positive integer and $\lambda$ a positive real number. Initially there are $n$ fleas on a horizontal line, not all at the same point. We define a move as choosing two fleas at some points $A$ and $B,$  with $A$ to the left of $B,$ and letting the flea from $A$ jump over the flea from $B$ to the point $C$ so that $\frac{BC}{AB} = \lambda.$
\smallbreak
Determine all values of $\lambda$ such that, for any point $M$ on the line and for any initial position of the $n$ fleas, there exists a sequence of moves that will take them all to a position to the right of $M.$

\textbf{194. (\color{red}d9\color{black}, 2012 RMM P3 of 6)} Each positive integer is coloured red or blue. A function $f$ from the set of positive integers to itself has the following two properties: \begin{enumerate} \item If $x \leq y$ the $f(x) \leq f(y).$ \item If $x, y, z$ are (not necessarily distinct) positive integers of the same colour with $x + y = z$ then $f(x) + f(y) = f(z).$ \end{enumerate} Prove that there exists a positive number $c$ such that $f(x) \leq cx$ for all positive integers $x.$

\textbf{11. (\color{red}d9\color{black}, 2017 IMO, Q2)} Let $\mathbb{R}$ be the set of real numbers. Determine all functions $f : \mathbb{R} \to \mathbb{R}$ such that, for all real numbers $x$ and $y$, \[f(f(x)f(y)) + f(x + y) = f(xy).\]

\textbf{1273. (\color{red}d10\color{black}, Kazakhstan National Olympiad Grades 10-11, P3)} Given $m\in\mathbb{N}$. Find all functions $f:\mathbb{R^{+}}arrow\mathbb{R^{+}}$ such that$$f(f(x)+y)-f(x)=( \frac{f(y)}{y}-1)x+f^m(y)$$holds for all $x,y\in\mathbb{R^{+}}.$\\

($f^m(x) =$ $f$ applied $m$ times.)

\textbf{1084. (\color{red}d10\color{black}, 2022 ICMC, Round 2, P5 of 5)} A robot on the number line starts at $1$.  During the first minute, the robot writes down the number $1$. Each minute thereafter, it moves by one, either left or right, with equal probability. It then multiplies the last number it wrote by $n/t$, where $n$ is the number it just moved to, and $t$ is the number of minutes elapsed.  It then writes this number down.  For example, if the robot moves right during the second minute, it would write down $2/2=1$.\\[6pt]

Find the expected sum of all numbers it writes down, given that it is finite.

\textbf{958. (\color{red}d10\color{black}, Folklore)} Let $X, Y$ be subsets of $\mathbb{R}^2$. Say a function $f : X \to Y$ is continuous if for any $\varepsilon > 0$, there exists a $\delta > 0$ such that for any $x_1, x_2 \in X$ with $|x_1 - x_2| < \delta$ we have $|f(x_1) - f(x_2)| < \varepsilon$. Say $X$ and $Y$ are homemorphic if there exists a continuous bijection $f : X \to Y$ such that $f^{-1}$ is continuous. Is $\mathbb{R}^2$ homeomorphic to $\mathbb{R}^2 \backslash \{(0,0)\}$?

\textbf{895. (\color{red}d10\color{black}, 2020 IMOSL, A8)} Let $\mathbb{R}^{+}$ be the set of positive real numbers. Determine all functions $f: \mathbb{R}^{+} arrow \mathbb{R}^{+}$ such that, for all positive real numbers $x$ and $y$,
$$
    f(x+f(x y))+y=f(x) f(y)+1 .
$$

\textbf{874. (\color{red}d10\color{black}, 2021 IMO, P2 of 6)} Show that the inequality
$$
    \sum_{i=1}^{n} \sum_{j=1}^{n} \sqrt{|x_{i}-x_{j}|} \leq \sum_{i=1}^{n} \sum_{j=1}^{n} \sqrt{|x_{i}+x_{j}|}
$$
holds for all real numbers $x_{1}, \ldots, x_{n}$.

\textbf{846. (\color{red}d10\color{black}, 2020 USAMO, P6 of 6)} Let $n \ge 2$ be an integer. Let $x_1 \ge x_2 \ge ... \ge x_n$ and $y_1 \ge y_2 \ge ... \ge y_n$ be $2n$ real numbers such that$$0 = x_1 + x_2 + ... + x_n = y_1 + y_2 + ... + y_n $$$$\text{and} \hspace{2mm} 1 =x_1^2 + x_2^2 + ... + x_n^2 = y_1^2 + y_2^2 + ... + y_n^2.$$Prove that$$\sum_{i = 1}^n (x_iy_i - x_iy_{n + 1 - i}) \ge \frac{2}{\sqrt{n-1}}.$$

\textbf{636. (\color{red}d10\color{black}, 2012 ELMO, P3 of 6)} Let $f,g$ be polynomials with complex coefficients such that $\gcd(\deg f,\deg g)=1$. Suppose that there exist polynomials $P(x,y)$ and $Q(x,y)$ with complex coefficients such that $f(x)+g(y)=P(x,y)Q(x,y)$. Show that one of $P$ and $Q$ must be constant.

\textbf{467. (\color{red}d10\color{black}, 2016 IMOSL A8)} Find the largest real constant $a$ such that for all $n \geq 1$ and for all real numbers $x_0, x_1, ... , x_n$ satisfying $0 = x_0 < x_1 < x_2 < \cdots < x_n$ we have
\[\frac{1}{x_1-x_0} + \frac{1}{x_2-x_1} + \dots + \frac{1}{x_n-x_{n-1}} \geq a ( \frac{2}{x_1} + \frac{3}{x_2} + \dots + \frac{n+1}{x_n} )\]

\textbf{503. (\color{red}d11\color{black}, 2017 IMOSL, A8)} A function $f:\mathbb{R} \to \mathbb{R}$ has the following property:

For every $x,y \in \mathbb{R}$ such that $(f(x)+y)(f(y)+x) > 0$, we have $f(x)+y = f(y)+x$.

Prove that $f(x)+y \leq f(y)+x$ whenever $x>y$.

\textbf{356. (\color{red}d11\color{black}, 2007 IMO P6 of 6)} Let $ n$ be a positive integer. Consider
\[ S = \{ (x,y,z) \mid x,y,z \in \{ 0, 1, \ldots, n\}, x + y + z > 0  \}
\]
as a set of $ (n + 1)^{3} - 1$ points in the three-dimensional space. Determine the smallest possible number of planes, the union of which contains $ S$ but does not include $ (0,0,0)$.

\textbf{216. (\color{red}d11\color{black}, 2012 China TST Day 2 P6 of 6 )} $n$ is a given integer. Find all functions $f\colon \mathbb{Z} \to \mathbb{Z}$, such that for all integers $x,y$ we have $f( {x + y + f(y)} ) = f(x) + ny$.

\textbf{373. (\color{red}dT\color{black}, 2020 AFMO, P1 of 4)} For which $n$ does there exist a sequence $a_1, a_2, a_3, \dots$ of 1's and \(-1\)'s, where \(a_i = a_{i-n}\) for \(i > n\), such that \[\sum_{i=1}^n a_ia_{i+k} = 0\] for all $k \in \{1, 2, \dots, n-1\}$?

\end{document}