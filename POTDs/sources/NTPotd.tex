\documentclass{article}
\usepackage{xcolor}
\usepackage{amsmath}
\usepackage{amsfonts}
\begin{document}
\textbf{1120. (\color{red}d0\color{black}, 2019 UK JMO, B2)} The product $8000\times K$ is a square, where $K$ is a positive integer.\vspace{8pt}

What is the smallest possible value of $K$?

\textbf{686. (\color{red}d0\color{black}, Folklore)} Alice picks 10 positive whole numbers and finds that their sum is 73. Is it true that one of the
numbers she picked must be odd?

\textbf{1337. (\color{red}d1\color{black}, F01K10R3)} F1ND 7H3 F0UR 5M411357 N47UR4L NUMB3R5 $n$ 5UCH 7H47 \begin{align*}n \equiv 2 \pmod E,\\n \equiv 1 \pmod A,\\n \equiv 2 \pmod S,\\n \equiv 0 \pmod T.\end{align*}

\textbf{1323. (\color{red}d1\color{black}, 2014/15 BMO1, P1 of 6)} Place the following numbers in increasing order of size, and justify your reasoning: $$3^{3^4},3^{4^3},3^{4^4},4^{3^3}\text{ and }4^{3^4}.$$

\textit{Note that $a^{b^c}$ means $a^{(b^c)}$.}

\textbf{1204. (\color{red}d1\color{black}, 2005/6 BMO1, P1 of 6)} Let $n$ be an integer greater than $6$. Prove that if $n - 1$ and $n + 1$ are both prime, then $n^2(n^2 + 16)$ is divisible by $720$. Is the converse true?

\textbf{1176. (\color{red}d1\color{black}, 2014 Kosovo MO, P1 of 5)} Prove that for any integer, the number $2n^3 + 3n^2 + 7n$ is divisible by 6.

\textbf{1029. (\color{red}d1\color{black}, 2001 BMO1, P1 of 5)} Find all two-digit integers $N$ for which the sum of the digits of $10^N - N$ is divisible by 170.

\textbf{840. (\color{red}d1\color{black}, 2021 UK JMO, B5)} In Sally's sequence, every term after the second is equal to the sum of the previous two terms. Also, every term is a positive integer. Her eighth term is 400.

Find the minimum value of the third term in Sally's sequence.

\textbf{812. (\color{red}d1\color{black}, 2021 NZ Monday Math Workshop, April, P1 of 6)} Find all pairs of positive integers $x$ and $y$ such that
\[x^2 + y^2 = 2048.\]

\textbf{798. (\color{red}d1\color{black}, 2012 AMO, P1 of 8)} Determine the largest $n \in \mathbb{N}$ such that $$ 4^{n}+2^{2012}+1 $$ is a perfect square.

\textbf{742. (\color{red}d1\color{black}, 2018 UK JMO, A3)} What is the largest integer for which each pair of consecutive digits is a square?

\textbf{728. (\color{red}d1\color{black}, 2017 UK MOG, P4 of 5)} Let $n$ be an odd integer greater than $3$ and let $M=n^2+2n-7$.

Prove that, for all such $n$, at least four different positive integers (excluding $1$ and $M$) divide $M$ exactly.

\textbf{721. (\color{red}d1\color{black}, 2018/9 UK STMC, Group Round, P10 of 10)} Find all pairs $(x,y)$ of positive integers such that $$x^2y-xy^2=286.$$

\textbf{665. (\color{red}d1\color{black}, 2007 Greece MO, P1 of 4)} Find all positive integers $n$ such that $4^n + 2007$ is a perfect square.

\textbf{115. (\color{red}d1\color{black}, 1999 Mexican MO (13th), P2)} Prove that there are no 1999 primes in an arithmetic progression that are all less than 12345.

\textbf{1380. (\color{red}d2\color{black}, 1994 Mexico MO, P1 of 6)} The sequence $1, 2, 4, 5, 7, 9 ,10, 12, 14, 16, 17, ... $ is formed as follows. First we take one odd number, then two even numbers, then three odd numbers, then four even numbers, and so on. Find the number in the sequence which is closest to $1994$.

\textbf{1372. (\color{red}d2\color{black}, 250 Problems in Elementary Number Theory (Sierpinski))} Prove that every positive integer $n > 7$ can be written as $n = a + b$ for integers $a, b > 1$ with $\gcd(a, b) = 1$.

\textbf{1302. (\color{red}d2\color{black}, 2007 BMO1, P1 of 6)} Let $S$ be a subset of the numbers $\{1, \cdots, 2008\}$ which consists of $756$ distinct numbers. Show that there are two distinct elements $a, b$ of $S$ such that $a+b$ is divisible by $8$.

\textbf{1288. (\color{red}d2\color{black}, 2002 Austria Regional Competition for Advanced Students, P1 of 4)} Find the smallest natural number $x > 0$ so that all of the following fractions are simplified:\begin{equation*}\frac{3x+9}{8},\, \frac{3x+10}{9},\, \frac{3x+11}{10},\, \cdots\!\!\,,\, \frac{3x+49}{48}\end{equation*}i.e. all numerators are relatively prime to their corresponding denominators.

\textbf{1274. (\color{red}d2\color{black}, Folklore)} Find all integer solutions to the equation $4x^3 + 2y^3 = z^3$.

\textbf{1240. (\color{red}d2\color{black}, 2010/11 BMO1, P2 of 6, modified)} Let $s$ be an integer greater than $6$. A solid cube of side $s$ has a square hole of side $x < s$ drilled directly through from one face to the opposite face (so the drill removes a cuboid). The volume of the remaining solid is numerically equal to the total surface area of the remaining solid. Determine all possible integer values of $x$.

\textbf{1232. (\color{red}d2\color{black}, 1999 BMO1, P3 of 5)} Determine a positive constant $c$ such that the equation \begin{equation*}xy^2 - y^2 - x + y = c\end{equation*} has precisely three solutions $(x, y)$ in positive integers.

\textbf{1197. (\color{red}d2\color{black}, 2004 APMO, P1 of 5)} Determine all finite, non-empty sets $S$ such that, if $i, j \in S$, we have \begin{equation*}\frac{i+j}{\gcd(i, j)} \in S.\end{equation*}

\textbf{1164. (\color{red}d2\color{black}, 2016 CMO, P1 of 5)} The integers \(1, 2, 3, \ldots , 2016\) are written on a board. You can choose any two numbers on the board and replace them with their average. For example, you can replace \(1\) and \(2\) with \(1.5\), or you can replace \(1\) and \(3\) with a second copy of \(2\). After \(2015\) replacements of this kind, the board will have only one number left on it.
\begin{enumerate}
    \item Prove that there is a sequence of replacements that will make the final number equal to \(2\).
    \item Prove that there is a sequence of replacements that will make the final number equal to \(1000\).
\end{enumerate}

\textbf{1156. (\color{red}d2\color{black}, PST 2.0.8)} Find all 5-digit natural numbers such that after deleting any one digit, the remaining number is a 4-digit number which is divisible by 7.

(Note that leading zeros do not count as digits, e.g.\ 00123 counts as a 3-digit number.)

\textbf{1148. (\color{red}d2\color{black}, Folklore)} Prove that there are infinitely many primes $p$ which leave a remainder of $3$ when divided by $4$.

\textbf{1141. (\color{red}d2\color{black}, 2002 Norwegian MO, P1a of 4)} Find all integers $k$ such that both $k + 1$ and $16k + 1$ are perfect squares.

\textbf{1121. (\color{red}d2\color{black}, Folklore)} Is it possible for the product of five consecutive positive integers to be a perfect square?

\textbf{1100. (\color{red}d2\color{black}, 2022 Irish EGMO TST, P3 of 4)} We say two prime numbers are cousins if they differ by no more than 20. For example, 3 and 23 are cousins, but 2 and 23 are not cousins.

Find all pairs of prime cousins whose product is two less than a perfect square.

\textbf{1099. (\color{red}d2\color{black}, 2015 BMO1, P1 of 6)} On Thursday 1st January 2015, Anna buys one book and one shelf. For the next two years, she buys one book every day and one shelf on alternate Thursdays, so she next buys a shelf on 15th January 2015. On how many days in the period Thursday 1st January 2015 until (and including) Saturday 31st December 2016 is it possible for Anna to put all her books on all her shelves, so that there is an equal number of books on each shelf?

\textbf{1080. (\color{red}d2\color{black}, 2019 PUMaC Team Problem 6 of 15)} Pavel and Sara roll two, fair six-sided dice (with faces labeled from 1 to 6) but do not look at the result. A third-party observer whispers the product of the face-up numbers to Pavel and the sum of the face-up numbers to Sara.

\makebox[1.5em]{}Pavel and Sara are perfectly rational and truth-telling, and they both know this.

\makebox[1.5em]{}Pavel says, ``With the information I have, I am unable to deduce the sum of the two numbers rolled.”

\makebox[1.5em]{}Sara responds, ``Interesting! With the information I have, I am unable to deduce the product of the two numbers rolled.”

\makebox[1.5em]{}Pavel responds, ``Wow! I still cannot deduce the sum. But I'm sure you know the product by now!”

\makebox[1.5em]{}What is the product?

\textbf{1065. (\color{red}d2\color{black}, 2015 UK IMOK, M1, adapted)} Consider the sequence $2, 22, 222, 2222, 22222, \ldots$.

Are any of the numbers in this sequence divisible by 2022? If so, what is the smallest such number?

\textbf{1016. (\color{red}d2\color{black}, Folklore — Gersonides (1343))} Find all non-negative integers $x$ and $y$ such that $3^x - 2^y = 1$.

\textbf{974. (\color{red}d2\color{black}, AIME 2013, P6 of 15)} Find the least positive integer $N$ such that the set of 1000 consecutive integers beginning with $1000 \cdot N$ contains no square of an integer.

\textbf{973. (\color{red}d2\color{black}, Typo)} Find all positive integers $n$ such that $2n-1$ has precisely $n$ positive factors.

\textbf{967. (\color{red}d2\color{black}, 1964 IMO, P1 of 6)}
\begin{enumerate}
    \item Find all positive integers $ n$ for which $ 2^n-1$ is divisible by $ 7$.
    \item Prove that there is no positive integer $ n$ for which $ 2^n+1$ is divisible by $ 7$.
\end{enumerate}

\textbf{946. (\color{red}d2\color{black}, PST 3.0.6)} Find all integers x and y such that \[x^3y + x + y = xy + 2xy^2.\]

\textbf{945. (\color{red}d2\color{black}, 2000 BMO1, P2 of 5)} Show that, for every positive integer $n$, \begin{equation*}121^n - 25^n + 1900^n - (-4)^n\end{equation*} is divisible by 2000.

\textbf{925. (\color{red}d2\color{black}, PST 1.0.11)} Prove that every positive integer can be uniquely expressed as a sum of different numbers,
where each number is of the form \(2^n\) for some non-negative integer \(n\).

\textbf{918. (\color{red}d2\color{black}, PST 3.0.4)} Find all ordered triples of positive integers such that each of them divides the sum of the other two.

\textbf{904. (\color{red}d2\color{black}, 1998 USAMO, P1 of 6)} Suppose that the set $\{1,2, \cdots, 1998\}$ has been partitioned into disjoint pairs $\left\{a_{i}, b_{i}\right\}$ $(1 \leq i \leq 999)$ so that for all $i,\left|a_{i}-b_{i}\right|$ equals 1 or 6. Prove that the sum
$$
    \left|a_{1}-b_{1}\right|+\left|a_{2}-b_{2}\right|+\cdots+\left|a_{999}-b_{999}\right|
$$
ends in the digit 9.

\textbf{897. (\color{red}d2\color{black}, 2016 Irish MO, P1 of 10)} If the three-digit number $ABC$ is divisible by 27, prove that the three-digit numbers $BCA$ and $CAB$ are also divisible by 27.

\textbf{896. (\color{red}d2\color{black}, 2001/02 BMO1, P1 of 5)} Find all positive integers $m, n$ where $n$ is odd, that satisfy \begin{equation*}\frac{1}{m} + \frac{4}{n} = \frac{1}{12}.\end{equation*}

\textbf{827. (\color{red}d2\color{black}, 2016 AMO, P6 of 8)} Let $a, b, c$ be positive integers such that $a^{3}+b^{3}=2^{c}$.
\smallbreak
Prove that $a=b$.

\textbf{805. (\color{red}d2\color{black}, 2019 Indonesia MO, P1 of 8)} Given that $n$ and $r$ are positive integers. Suppose that $$ 1 + 2 + \dots + (n - 1) = (n + 1) + (n + 2) + \dots + (n + r) $$ Prove that $n$ is a composite number.

\textbf{792. (\color{red}d2\color{black}, 2019 Tournament of Towns Junior O-Level, P3 of 5)} The product of two positive integers $m$ and $n$ is divisible by their sum. Prove that $m + n \le  n^2$.

\textbf{777. (\color{red}d2\color{black}, 2002 AMO, P1 of 8)} Let $m, n \in \mathbb{N}$ such that $2001 m^{2}+m=2002 n^{2}+n$.
Prove that $m-n$ is a perfect square.

\textbf{659. (\color{red}d2\color{black}, 2020 Canada MO, P1 of 5)} There are $n \ge 3$ distinct positive real numbers. Show that there are at most $n-2$ different integer power of three that can be written as the sum of three distinct elements from these $n$ numbers.

\textbf{651. (\color{red}d2\color{black}, 2014/15 BMO1, P2 of 6)} Positive integers $p$, $a$ and $b$ satisfy the equation $p^2+a^2=b^2$. Prove that if $p$ is a prime greater than $3$, then $a$ is a multiple of $12$ and $2(p+a+1)$ is a perfect square.

\textbf{609. (\color{red}d2\color{black}, Folklore)} Ramanujan is bored because he finished his JEE Advanced exam an hour early. He tries concatenating numbers by writing them down as a single series of digits and looking and the number formed; for example, 17 and 29 concatenated makes 1729. He then tries concatenating all the numbers between 1 and 2018 in all possible different orders; are any of the numbers he forms cubes?

\textbf{588. (\color{red}d2\color{black}, 2016 UK IMOK, H6)} Tony multiplies at least two consecutive positive integers. He obtains the six-digit number $N$. The left-hand digits of $N$ are `47' and the right-hand digits of $N$ are `74'.

What integers did Tony multiply together?

\textbf{574. (\color{red}d2\color{black}, Handout by Oleg Golberg)} Let $S$ be a set of numbers. If $a,b\in S$ then we can add $ab+a+b$ to $S$. Initially $S=\{2,3,4,5\}$. Is it possible that $3023\in S$?

\textbf{567. (\color{red}d2\color{black}, Folklore)} Prove that if all the coefficients of the quadratic equation
$$
    ax^2 + bx + c = 0
$$
are odd integers, then the roots of the equation cannot be rational.

\textbf{553. (\color{red}d2\color{black}, 2018 UK IMOK, M5)} For which integers $n$ is $\frac{16(n^2-n-1)^2}{2n-1}$ also an integer?

\textbf{483. (\color{red}d2\color{black}, 2000 Putnam B-2)} Prove that the expression
\[
    \frac{\gcd(m,n)}{m} \binom{n}{m}
\]
is an integer for all pairs of integers $n \geq m \geq 1$.

\textbf{420. (\color{red}d2\color{black}, 2015 Spain MO, P1 of 6)} Suppose you pick two points with integer coordinates on the graph of a polynomial with integer coefficients. Prove that if the distance between these points is an integer, then the line joining them is parallel to the $x$-axis.

\textbf{406. (\color{red}d2\color{black}, 2018 AMO, P1 of 8 )} Find all pairs of positive integers $(n, k)$ such that
$$
    n! + 8 = 2^k.
$$

\textbf{399. (\color{red}d2\color{black}, 2013 NZ Camp Selection, P8 of 12)} Suppose that $a$ and $b$ are positive integers such that.
$$
    c = a + \frac{b}{a} - \frac{1}{b}
$$
is an integer. Prove that $c$ is a perfect square

\textbf{393. (\color{red}d2\color{black}, 2019 NZ Senior Math Competition, P8 of 15)} Find the smallest natural number $n$ such that $n$ has exactly 24 divisors.

\textbf{364. (\color{red}d2\color{black}, 1975 IMO P4 of 6)} When \(4444^{4444}\) is written in decimal notation, the sum of its digits is \(A .\) Let
\(B\) be the sum of the digits of \(A .\) Find the sum of the digits of \(B\). ($A$ and $B$) are written in decimal notation.

\textbf{344. (\color{red}d2\color{black}, 2020 AMO, Q1 of 8)} Determine all pairs $(a, b)$ of non-negative integers such that $$\frac{a+b}{2} - \sqrt{ab} = 1.$$

\textbf{280. (\color{red}d2\color{black}, AMOC Senior Contest 2014, P2 of 5)} For which integers $n \geq 2$ is it possible to separate the numbers $1, 2, \cdots ,n$ into two sets such that the sum of the numbers in one of the sets is equal to the product of the numbers in the other set?

\textbf{259. (\color{red}d2\color{black}, 2016 BAMO-12, P1 of 5)} The ${\textit{distinct prime factors}}$ of an integer are its prime factors listed without repetition. For example, the distinct prime factors of $40$ are $2$ and $5$. Let $A=2^k - 2$ and $B= 2^k \cdot A$, where $k$ is an integer ($k \ge 2$).\\
\makebox[10pt]{} Show that, for every integer $k$ greater than or equal to $2$,
\begin{itemize}
    \item[(i)] $A$ and $B$ have the same set of distinct prime factors.
    \item[(ii)] $A+1$ and $B+1$ have the same set of distinct prime factors.
\end{itemize}

\textbf{212. (\color{red}d2\color{black}, 2006 Swedish MO, P1 of 6)} If positive integers \(a\) and \(b\) have 99 and 101 different positive divisors respectively, can the product \(ab\) have exactly 150 positive divisors?

\textbf{197. (\color{red}d2\color{black}, 2019 New Zealand Senior MC, Q2)} Suppose there is a positive integer $n$ such that $\frac{n}{2}$ is a perfect square, $\frac{n}{3}$ is a perfect cube, and $\frac{n}{5}$ is a perfect fifth power. Find an expression for the smallest value of $n$.

\textbf{175. (\color{red}d2\color{black}, 2012 INAMO Round 1)} Find all integers $n$ that satisfy



$(n-1)(n-3)(n-5)\cdots (n-2013) = (n+2)(n+4)(n+6) \cdots (n+2012)$

\textbf{140. (\color{red}d2\color{black}, 1997 Mexico MO, Day 1, P1 of 3)} Determine all prime numbers \(p\) for which \(8p^4 - 3003\) is a positive prime number.

\textbf{93. (\color{red}d2\color{black}, 2010 Spanish MO Day 1, Q1)} A \textit{pucelana} sequence is an increasing sequence of 16 consecutive odd numbers whose sum is a perfect cube. How many pucelana sequences are there with 3-digit numbers only?

\textbf{87. (\color{red}d2\color{black}, 1998 Asian Pacific Mathematical Olympiad, Q2)} Show that for any two positive integers \(a\) and \(b\), \((36a+b)(a+36b)\) cannot be a power of 2.

\textbf{67. (\color{red}d2\color{black}, Tournament of Towns Spring 2004 Junior O-level Q4)} A positive integer $a > 1$ is given (in decimal notation). We copy it twice and obtain a number $b = \overline{aa}$ which happens to be a multiple of $a^2$. Find all possible values of $\frac{b}{a^2}$.

\textbf{1359. (\color{red}d3\color{black}, 2011 USAJMO, P1 of 6)} Find, with proof, all positive integers $n$ for which $2^n + 12^n + 2011^n$ is a perfect square.

\textbf{1345. (\color{red}d3\color{black}, 2011 Albania MO, P4 of 5)} The sequence $(a_{n})$ is defined by $a_1=1$ and $a_n=n(a_1+a_2+\cdots+a_{n-1})$ , $\forall n>1$.
\begin{enumerate}
    \item Prove that for every even $n$, $a_{n}$ is divisible by $n!$.
    \item Find all odd numbers $n$ for the which $a_{n}$ is divisible by $n!$.
\end{enumerate}

\textbf{1339. (\color{red}d3\color{black}, 2023 BMO1, P5 of 6)} For each integer $n\ge1$, let $f(n)$ be the number of lists of different positive integers starting with $1$ and ending with $n$, in which each term except the last divides its successor. Prove that for each integer $N\ge1$ there is an integer $n\ge1$ such that $N$ divides $f(n)$.

\textit{(So $f(1)=1$, $f(2)=1$ and $f(6)=3.$)}

\textbf{1318. (\color{red}d3\color{black}, 2002 IMOSL, N1)} What is the smallest positive integer $t$ such that there exist integers $x_1,x_2,\ldots,x_t$ with\[x^3_1+x^3_2+\,\ldots\,+x^3_t=2002^{2002}\,?\]

\textbf{1261. (\color{red}d3\color{black}, 1978 Kurschak Competition)} Show that $n^4+4^n$ is composite for all $n \in \mathbb{N} \backslash \{1\}$

\textbf{1247. (\color{red}d3\color{black}, 1986 IMO, P1 of 6)} Let $d$ be any positive integer not equal to $2, 5$ or $13$. Show that one can find distinct $a,b$ in the set $\{2,5,13,d\}$ such that $ab-1$ is not a perfect square.

\textbf{1241. (\color{red}d3\color{black}, 2010 USAMTS R2, P1 of 5)} Jeremy has a magic scale, each side of which holds a positive integer. He plays the following game: each turn, he chooses a positive integer $n$. He then adds $n$ to the number on the left side of the scale, and multiplies by $n$ the number on the right side of the scale. (For example, if the turn starts with 4 on the left and 6 on the right, and Jeremy chooses $n = 3$, then the turn ends with 7 on the left and 18 on the right.) Jeremy wins if he can make both sides of the scale equal.

Prove that if the game starts with the right scale holding $b$, where $b \geq 2$, then Jeremy can win the game in $b - 1$ or fewer turns.


\textbf{1225. (\color{red}d3\color{black}, Folklore (result from cubic residues))} Let $p$ be a prime which leaves a remainder of \(2\) when divided by \(3\). Show that, for any integer \(a\), there is some integer \(x\) such that \(x^3\) and \(a\) have the same remainder when divided by \(p\).

\textbf{1115. (\color{red}d3\color{black}, 2013 PUMaC Div A NT, P7 of 8)} Suppose $P(x)$ is a degree $n$ monic polynomial with integer coefficients such that $2013$ divides $P(r)$ for exactly $1000$ values of $r$ between $1$ and $2013$ inclusive. Find the minimum value of $n$.

\textbf{1073. (\color{red}d3\color{black}, 2015 HMIC, P1 of 5)} Let $S$ be the set of positive integers n such that the inequality
$\phi(n) · \tau(n) \geq \sqrt{\frac{n^3}{3}}$
holds, where $\phi(n)$ is the number of positive integers $k \leq n$ that are relatively prime to $n$, and $\tau (n)$ is
the number of positive divisors of $n$. Prove that $S$ is finite.

\textbf{1066. (\color{red}d3\color{black}, 2022 AIME II P8 of 15)} Find the number of positive integers $n \le 600$ whose value can be uniquely determined when the values of $\left\lfloor \frac n4\right\rfloor$, $\left\lfloor\frac n5\right\rfloor$, and $\left\lfloor\frac n6\right\rfloor$ are given, where $\lfloor x \rfloor$ denotes the greatest integer less than or equal to the real number $x$.

\begin{center}
    {\it (Note that it is not given that $n \le 600$ when finding the answer. That is, you cannot assume that $n \le 600$ to determine the value of $n$)}
\end{center}


\textbf{1051. (\color{red}d3\color{black}, 2003 USAMO, P1 of 6)} Prove that for every positive integer $n$ there exists an $n$-digit number divisible by $5^{n}$ all of whose digits are odd.

\textbf{1045. (\color{red}d3\color{black}, 2021 AIME I, P 14 of 15)} For any positive integer $a,$ $\sigma(a)$ denotes the sum of the positive integer divisors of $a$. Let $n$ be the least positive integer such that $\sigma(a^n)-1$ is divisible by $2021$ for all positive integers $a$. Find $n$.

\textbf{1037. (\color{red}d3\color{black}, Folklore)} Find with proof all positive integers $k$ such that, for $n=2^{k}$, every prime number which divides $n!+1$ also divides $n+1$.

\textbf{995. (\color{red}d3\color{black}, AIME 2019, P14 of 15)} Find the least odd prime factor of $2019^{8}+1$.

\textbf{968. (\color{red}d3\color{black}, 2019 HMIC, P1 of 5)} Let $m > 1$ be a fixed positive integer. For a nonempty string of base-ten digits $S$, let $c(S)$ be the number of ways to split $S$ into contiguous nonempty strings of digits such that the base-ten number represented by each string is divisible by $m$. These strings are allowed to have leading zeroes.
In terms of m, what are the possible values that c(S) can take?

For example, if $m = 2$, then $c(1234) = 2$ as the splits $1234$ and $12|34$ are valid, while the other six splits are invalid.

\textbf{954. (\color{red}d3\color{black}, 2020 Tournament of Towns Senior A, P1 of 7)} There were $n$ positive integers. For each pair of those integers Boris wrote their

arithmetic mean onto a blackboard and their geometric mean onto a whiteboard. It so

happened that for each pair at least one of those means was integer. Prove that on at

least one of the boards all the numbers are integer.

\textbf{933. (\color{red}d3\color{black}, 2021 AOPS Practice AMC 12, P21 of 25 )} The number $1 + 2^{21} + 4^{21}$ is divisible by exactly one two-digit prime $p.$ What is the sum of the digits of $p?$

\textbf{910. (\color{red}d3\color{black}, AIME 2021, P7 of 15)} Find the number of pairs $(m,n)$ of positive integers with $1 \leq m < n \leq 30$ such that there exists a real number $x$ satisfying $\sin(mx) + \sin(nx) = 2$.

\textbf{905. (\color{red}d3\color{black}, 2017 USAMTS R3 P2 of 5)} Let $q$ be a real number. Suppose
there are three distinct positive integers $a, b, c$ such that $q + a, q + b, q + c$ is a geometric
progression. Show that q is rational

\textbf{876. (\color{red}d3\color{black}, Taiwan "IMOC", N3)} Define the function $f: \mathbb{N}_{>1} \rightarrow \mathbb{N}_{>1}$ such that $f(x)$ is the greatest prime factor of $x$. A sequence of positive integers $\left\{a_{n}\right\}$ satisfies $a_{1}=M>1$ and
$$
    a_{n+1}=\left\{\begin{array}{l}
        a_{n}-f\left(a_{n}\right), \quad \text { if } a_{n} \text { is composite. } \\
        a_{n}+k, \quad \text { otherwise }
    \end{array}\right.
$$
Show that for any positive integers $M, k$, the sequence $\left\{a_{n}\right\}$ is bounded.

\textbf{869. (\color{red}d3\color{black}, 2021 NZMO1, P4)} Find all triples \((x,p,n)\) of non-negative integers such that \(p\) is prime and \[2x(x+5) = p^n + 3(x-1).\]

\textbf{855. (\color{red}d3\color{black}, 2017 ASC, P3 of 5)} Let $a_{1}<a_{2}<\cdots<a_{2017}$ and $b_{1}<b_{2}<\cdots<b_{2017}$ be positive integers such that $$ \left(2^{a_{1}}+1\right)\left(2^{a_{2}}+1\right) \cdots\left(2^{a_{2017}}+1\right)=\left(2^{b_{1}}+1\right)\left(2^{b_{2}}+1\right) \cdots\left(2^{b_{2017}}+1\right) $$ Prove that $a_{i}=b_{i}$ for $i=1,2, \ldots, 2017$.

\textbf{834. (\color{red}d3\color{black}, 2021 MODSMO, P2 of 7)} Find all integers $n$ that can be written in the form $$ n = \frac{x^{2}}{x+\lfloor\sqrt{x}\rfloor+1} $$ for some positive integer $x$.

\textbf{821. (\color{red}d3\color{black}, 2020 ASC, P3 of 5)} Let $a_{1}$ be a given integer greater than 1. For $k=2,3,4, \ldots$, let $a_{k}$ be the smallest positive integer that satisfies the following conditions:
\begin{itemize}
    \item $a_{k}>a_{k-1}$
    \item $a_{k}$ is not divisible by $a_{r}$ for any $r<k$.
\end{itemize}
Prove that the number of composite numbers in the sequence $a_{1}, a_{2}, a_{3}, \ldots$ is finite.

\textbf{701. (\color{red}d3\color{black}, 2021 Irish EGMO TST, P4 of 5)} Let us say that two rational numbers $p=a/b$ and $q=c/d$ (written in reduced terms) are \emph{linked} if $|p-q|=1/b d$. We say that $p$ and $q$ are \emph{connected} if there is a sequence of rational numbers $q_{0}=p$, $q_{1}$, $\ldots$, $q_{n}=q$, with each pair $q_{i}, q_{i+1}$ linked.

Show that any pair of rational numbers is connected.

\textbf{639. (\color{red}d3\color{black}, Folklore)} Let $p$ be a prime and define $k = \left\lfloor \frac{p}{4}\right\rfloor$. Find necessary and sufficient conditions on $p$ so that $p \mid k^{2k} - 1$.

\textbf{624. (\color{red}d3\color{black}, Tony Wang's friend)} Suppose Tony Wang has $p$ bowls of Chinese noodle soup, where $p$ is prime. Suppose $p^2+2$ is also prime. Prove that $p^3+2$ is prime.

\textbf{618. (\color{red}d3\color{black}, 2011/2 BMO1, P5 of 6)} Prove that the product of four consecutive positive integers cannot be equal to the product of two consecutive positive integers.

\textbf{554. (\color{red}d3\color{black}, 2013/4 BMO1, P3 of 6)} A number written in base 10 is a string of $3^{2013}$ digit $3$s. No other digit appears. Find the highest power of 3 which divides this number.

\textbf{547. (\color{red}d3\color{black}, 2015 IrMO, P9 of 10)} Let $p(x)$ and $q(x)$ be non-constant polynomial functions with integer coefficients. It is known that the polynomial $$p(x)q(x) = 2015$$
has at least $33$ different integer roots. Prove that neither $p(x)$ nor $q(x)$ can be a polynomial of degree less than three.

\textbf{505. (\color{red}d3\color{black}, 2006 IMOSL, N2)} For $ x \in (0, 1)$ let $ y \in (0, 1)$ be the number whose $ n$-th digit after the decimal point is the $ 2^{n}$-th digit after the decimal point of $ x$. Show that if $ x$ is rational then so is $ y$.

\textbf{484. (\color{red}d3\color{black}, 2019 NZMO1, P4 of 8)} Show that the number $122^n - 102^n - 21^n$ is always one less than a multiple of 2020, for any positive integer $n$.

\textbf{476. (\color{red}d3\color{black}, 2000 Irish MO, P8 of 10)} For each positive integer $n$ determine with proof, all positive integers $m$ such that there exist positive integers $x_{1}<x_{2}<\cdots<x_{n}$ with
\[
    \frac{1}{x_{1}}+\frac{2}{x_{2}}+\frac{3}{x_{3}}+\cdots+\frac{n}{x_{n}}=m
\]

\textbf{469. (\color{red}d3\color{black}, 2010 Iran TST, P12 of 12)} Prove that for each natural number $m$, there is a natural number $N$ such that for each $b$ satisfying $2 \leq b \leq 1389,$ the sum of the digits of $N$ in base $b$ is larger than $m$.

\textbf{456. (\color{red}d3\color{black}, 2018 RMO, P3)} Show that there are infinitely many tuples $(a,b,c,d)$ of natural numbers such that $a^3 + b^4 + c^5 = d^7$.

\textbf{386. (\color{red}d3\color{black}, 2020 BMO2, P1 of 4)} A sequence $a_1, a_2, \ldots $ has $a_1 > 2$ and satisfies: $$a_{n+1} = \frac{a_n(a_n-1)}{2}$$ for all positive integers $n$. For which values of $a_1$ are all terms of the sequence odd integers?

\textbf{331. (\color{red}d3\color{black}, PST 3.9)} Find all integral solutions to \[x^3 + 3y^3 + 9z^3 = 9xyz\]

\textbf{294. (\color{red}d3\color{black}, 2008 Singapore MO, Jnr. Rnd 2, Q5 of 5)} Determine all primes $p$ such that $5^p + 4p^4$ is a perfect square.

\textbf{287. (\color{red}d3\color{black}, 2007 INMOTC )} Show that every integer $n > 10$ with all digits elements of the set $\{1, 3, 7, 9\}$ has a prime factor at least 11.

\textbf{282. (\color{red}d3\color{black}, 2019 Kosovo MO, 11th Grade, P2 of 5)} Find all positive integers $n$ such that every decimal digit of $6^n+1$ is the same.

\textbf{253. (\color{red}d3\color{black}, 2019 BMO1, Q4)} There are 2019 penguins waddling towards their favourite restraunt. As the penguins arrive, they are handed tickets numbered in ascending order from 1 to 2019, and told to join the queue. The first penguin starts the queue. For each $n > 1$ the penguin holding ticket number $n$ finds the greatest $m < n$ which divides $n$ and enters the queue directly behind the penguin holding ticket number $m$. This continues until all 2019 penguins are in the queue.

\begin{itemize}

    \item How many penguins are in front of the penguins with ticket number 2?

    \item[(b)] What numbers are held by the penguins just in front of and just behind the penguin holding ticket number 33?

\end{itemize}

\textbf{252. (\color{red}d3\color{black}, 2011 China Western MO, Day 2, Problem 1 of 4)} Does there exist any odd integer $n \geq 3$ and $n$ distinct prime numbers $p_1 , p_2, \cdots p_n$ such that all $p_i + p_{i+1} (i = 1,2,\cdots , n$ and $p_{n+1} = p_{1})$ are perfect squares?

\textbf{217. (\color{red}d3\color{black}, USAMTS Year 30 Round 1 Problem 3 of 5 )} Find, with proof, all pairs of positive integers $(n, d)$ with the following property: for every integer $S$, there exists a unique non-decreasing sequence of $n$ integers $a_1 , a_2 , a_3 , \cdots , a_n$ such that $a_1 + a_2 + a_3 + \cdots + a_n = S$ and $a_n - a_1 = d$

\textbf{183. (\color{red}d3\color{black}, 2017 BMO1, Q1 of 6 )} Helen divides $365$ by each of $1, 2, 3, \cdots, 365$ in turn, writing down a list of the $365$ remainders. Then Phil divides $366$ by each of $1, 2, 3, \cdots , 366$ in turn, writing down a list of the $366$ remainders. Whose list of remainders has the greater sum and by how much?

\textbf{169. (\color{red}d3\color{black}, 2001 Polish MO Round 1, Q9 of 12)} Prove that among any 12 consecutive integers there is one that cannot be written as the sum of 10 fourth powers.



\textbf{121. (\color{red}d3\color{black}, 1999 Netherlands MO, P4 of 5)} All entries of a \(8 \times 8\) matrix are positive integers. One may repeatedly transform the entries of the matrix according to the following rules:

\begin{enumerate}

    \item Multiply all entries in some row by 2.

    \item Subtract 1 from all entries in some column.

\end{enumerate}

Prove that it is possible to transform the given matrix into the zero matrix.

\textbf{99. (\color{red}d3\color{black}, 2001 Polish MO, Round 1, Q5)} Prove that for all integers \(n \geq 2\) and all prime numbers \(p\), the number \(n^{p^p} + p^p\) is composite.

\textbf{90. (\color{red}d3\color{black}, 2017 Singapore MO Open Round 2, Q3)} Find the smallest integer $n$ so that $\sqrt{\frac{1^2+2^2+3^2+ \cdots + n^2}{n}}$ is an integer.

\textbf{81. (\color{red}d3\color{black}, 2015 Singapore Maths Olympiad Junior R2, Q5)} Find all positive integers \(k\) such that \(k^k+1\) is divisible by \(30\). Justify your answer.

\textbf{73. (\color{red}d3\color{black}, 2011 BMO2, Q2 of 4)} Find all positive integers $x$ and $y$ such that $x+y+1$ divides $2xy$ and $x+y-1$ divides $x^2+y^2-1$.

\textbf{71. (\color{red}d3\color{black}, 1989 APMO, Q2 of 5)} Prove that the equation $[6(6a^2+3b^2+c^2)=5n^2]$ has no solutions in integers except $a=b=c=n=0$.

\textbf{61. (\color{red}d3\color{black}, Monday Maths Workshop Feb 2019, Q6 of 8)} Show that there are infinitely many integer solutions $(x, y)$ to the equation \[x^2-3y^2=1.\]

\textbf{50. (\color{red}d3\color{black}, 2014 BMO2, Q3 of 4)} Let $a_0 = 4$ and define a sequence of terms using the formula $an = a{n-1}^2 - a_{n-1}$ for each positive integer $n$.\
a) Prove that there are infinitely many prime numbers which are factors of at least one term of the sequence.\
b) Are there infinitely many prime numbers which are factors of no term in the sequence?

\textbf{42. (\color{red}d3\color{black}, 2008 Spanish MO, Q4 of 6)} Let $p$ and $q$ be two different prime numbers. Prove that there are two positive integers, $a$ and $b$, such that the arithmetic mean of the divisors of $n = p^a q^b$ is an integer.

\textbf{29. (\color{red}d3\color{black}, 2018 Putnam, B3)} Find all positive integers $n < 10^{100}$ for which simultaneously $n$ divides $2^n, n - 1$ divides $2^n - 1$ and $n - 2$ divides $2^n - 2$.

\textbf{21. (\color{red}d3\color{black}, 2009 Japanese MO, Final Round, Q2)} Let $N$ be a positive integer. Prove that if the sum of the elements in ${1,2,\dots,N}$ is even, then it is possible to paint each each element red or green so that the sum of the red numbers is equal to the sum of the green numbers.

\textbf{1367. (\color{red}d4\color{black}, 2021 CJMO, P2 of 5)} How many ways are there to permute the first $n$ positive integers such that in the
permutation, for each value of $k \leq n$, the first $k$ elements of the permutation have
distinct remainder mod $k$?

\textbf{1360. (\color{red}d4\color{black}, 2022 Canadian MO, P 2 of 5)} Let $d(k)$ denote the number of positive integer divisors of $k$. For example, $d(6) = 4$ since
6 has 4 positive divisors, namely, 1, 2, 3, and 6. Prove that for all positive integers $n$, $d(1) + d(3) + d(5) + \cdots· + d(2n - 1) \leq d(2) + d(4) + d(6) + \cdots + d(2n)$

\textbf{1340. (\color{red}d4\color{black}, Folklore)} Let $S(k)$ be the sum of the digits of $k$ in base $10$. Let $a_n = S(2^n)$. Does there exist a $N \in \mathbb{N}$ such that whenever $n > N$ we have $a_{n+1} > a_n$?

\textbf{1319. (\color{red}d4\color{black}, 2015 BMO1, P6 of 6)} Determine all functions \(f(n)\) from the positive integers to the positive integers which satisfy the following condition: whenever \(a\), \(b\) and \(c\) are positive integers such that \(1/a + 1/b = 1/c\), then \[1/f (a) + 1/f (b) = 1/f (c).\]

\textbf{1311. (\color{red}d4\color{black}, 2003 IMOSL, N1)} Let $m$ be a fixed integer greater than $1$. The sequence $x_0$, $x_1$, $x_2$, $\ldots$ is defined as follows:
\[x_i = \begin{cases}2^i&\text{if }0\leq i \leq m - 1;\\\sum_{j=1}^mx_{i-j}&\text{if }i\geq m.\end{cases}\]Find the greatest $k$ for which the sequence contains $k$ consecutive terms divisible by $m$ .

\textbf{1221. (\color{red}d4\color{black}, 1979 IMO, P1 of 6)} If $p$ and $q$ are natural numbers so that\[ \frac{p}{q}=1-\frac{1}{2}+\frac{1}{3}-\frac{1}{4}+ \ldots -\frac{1}{1318}+\frac{1}{1319}, \]prove that $p$ is divisible with $1979$.

\textbf{1214. (\color{red}d4\color{black}, 2007 Belarusian MO, P7 of 8)} Find all solutions in positive integers to \[n^5+n^4=7^m-1.\]

\textbf{1193. (\color{red}d4\color{black}, 1996 Spanish MO)} The natural numbers $a$ and $b$ are such that $ \frac{a+1}{b}+ \frac{b+1}{a}$ is an integer. Show that the greatest common divisor of a and b is not greater than $\sqrt{a+b}$.

\textbf{1185. (\color{red}d4\color{black}, 2011 CMO P 4 of 5)} Show that there exists a positive integer $N$ such that for all integers $a > N$, there exists
a contiguous substring of the decimal expansion of $a$ that is divisible by 2011. (For
instance, if $a = 153204$, then 15, 532, and 0 are all contiguous substrings of $a$. Note that
0 is divisible by 2011.)

\textbf{1157. (\color{red}d4\color{black}, 2015 CMO, P1 of 5)} Let $\mathbb N = \{1, 2, 3, \ldots \}$ be the set of positive integers. Find all functions $f$, defined on $\mathbb N$ and taking values in $\mathbb N$, such that $(n - 1)^2 < f(n)f(f(n)) < n^2 + n$ for every positive integer $n$.

\textbf{1137. (\color{red}d4\color{black}, 2000 IMOSL, N1)} Determine all positive integers $ n\geq 2$ that satisfy the following condition: for all $ a$ and $ b$ relatively prime to $ n$ we have\[a \equiv b \pmod n\qquad\text{if and only if}\qquad ab\equiv 1 \pmod n.\]

\textbf{1129. (\color{red}d4\color{black}, 2022 USAMO, P4 of 6)} Find all pairs of primes $(p, q)$ for which $p-q$ and $pq-q$ are both perfect squares.

\textbf{1122. (\color{red}d4\color{black}, 2021 BMO2, P1 of 4)} A positive integer $n$ is called \emph{good} if there is a set of divisors of $n$ whose members sum to $n$
and include $1$. Prove that every positive integer has a multiple which is good.

\textbf{1059. (\color{red}d4\color{black}, 2022 BMO2, P1 of 4)} For a given positive integer \(k\), we call an integer \(n\) a \(k\)-number if both of the following
conditions are satisfied:
\begin{enumerate}
    \item The integer \(n\) is the product of two positive integers which differ by \(k\).
    \item The integer \(n\) is \(k\) less than a square number.
\end{enumerate}
Find all \(k\) such that there are infinitely many \(k\)-numbers.

\textbf{1017. (\color{red}d4\color{black}, PST 1.10)} Show that if we take \(n+1\) numbers from the set \(\{1,2,3,\dots,2n\}\), there must exist one which is divisible by another.

\textbf{961. (\color{red}d4\color{black}, 2021 HMMT Feb Team P 3 of 10)} Let $m$ be a positive integer. Show that there exists a positive integer $n$ such that each of the
$2m + 1$ integers
\[
    2^n-m, 2^n-(m-1),\ldots,2^n+(m-1), 2^n+m
\]
is positive and composite.

\textbf{848. (\color{red}d4\color{black}, 2010 BMO2, P3 of 4)} The integer $x$ is at least $3$ and $n=x^6-1$. Let $p$ be a prime and $k$ be a positive integer such that $p^k$ is a factor of $n$. Show that $p^{3k}<8n$.

\textbf{842. (\color{red}d4\color{black}, 1991 IMO, P2 of 6)} Let $\,n > 6\,$ be an integer and $\,a_{1},a_{2},\cdots ,a_{k}\,$ be all the natural numbers less than $n$ and relatively prime to $n$. If \[a_{2} - a_{1} = a_{3} - a_{2} = \cdots = a_{k} - a_{k - 1} > 0,\] prove that $\,n\,$ must be either a prime number or a power of $\,2$.

\textbf{779. (\color{red}d4\color{black}, 2021 Irish MO, P3 of 10)} For each integer $n \geq 100$ we define $T(n)$ to be the number obtained from $n$ by moving the two leading digits to the end. For example, $T(12345)=34512$ and $T(100)=10$. Find all integers $n \geq 100$ for which:
$$
    n+T(n)=10 n.
$$

\textbf{751. (\color{red}d4\color{black}, 2021 EGMO P6 of 6)} Does there exist a nonnegative integer $a$ for which the equation
\[\left\lfloor\frac{m}{1}\right\rfloor + \left\lfloor\frac{m}{2}\right\rfloor + \left\lfloor\frac{m}{3}\right\rfloor + \cdots + \left\lfloor\frac{m}{m}\right\rfloor = n^2 + a\]has more than one million different solutions $(m, n)$ where $m$ and $n$ are positive integers?

The expression $\lfloor x\rfloor$ denotes the integer part (or floor) of the real number $x$. Thus $\lfloor\sqrt{2}\rfloor = 1, \lfloor\pi\rfloor =\lfloor 22/7 \rfloor = 3, \lfloor 42\rfloor = 42,$ and $\lfloor 0 \rfloor = 0$.

\textbf{582. (\color{red}d4\color{black}, 2020 USEMO, P1 of 6)} Which positive integers can be written in the form
$$
    \frac{\operatorname{lcm}(x, y)+\operatorname{lcm}(y, z)}{\operatorname{lcm}(x, z)}
$$
for positive integers $x, y, z$?

\textbf{527. (\color{red}d4\color{black}, 2019 Finnish MO, P4 of 5)} Define the sequence $a_n = n^n - (n-1)^{n+1}$ for positive integers $n$. Find all positive integers $m$ such that $a_n$ is eventually periodic modulo $m$.

\textbf{498. (\color{red}d4\color{black}, 2017 ELMO, P1 of 6)} Let $a_1,a_2,\dots, a_n$ be positive integers with product $P,$ where $n$ is an odd positive integer. Prove that$$\gcd(a_1^n+P,a_2^n+P,\dots, a_n^n+P)\le 2\gcd(a_1,\dots, a_n)^n.$$

\textbf{444. (\color{red}d4\color{black}, BMO2 1994, P4 of 4)} How many perfect squares are there mod $2^n$?

\textbf{401. (\color{red}d4\color{black}, 2020 NZ Squad Selection Test Rnd 2, P8 of 8)} Show that the equation $$a^2 = b^2 + bc + c^2$$ has infinitely many positive integer solutions such that $a$, $b$ and $c$ have no common factor greater than $1$.

\textbf{366. (\color{red}d4\color{black}, 2017 APMO, P1 of 5)} We call a 5-tuple of integers \textit{arrangeable} if its elements can be labelled $a$, $b$, $c$, $d$, $e$ in some order so that $a - b + c - d + e = 29$. Determine all 2017-tuples of integers $n_1, n_2, \dots, n_{2017}$ such that if we place them in a circle in clockwise order, then any 5-tuple of numbers in consecutive positions on the circle is arrangeable.

\textbf{338. (\color{red}d4\color{black}, 1998 BMO2, P3 of 4)} Suppose $x, y, z$ are positive integers satisfying the equation \begin{equation*}\frac{1}{x} - \frac{1}{y} = \frac{1}{z},\end{equation*} and let $h$ be the highest common factor of $x,y,z$.\\ Prove that $hxyz$ is a perfect square.\\ Prove also that $h(y-x)$ is a perfect square.

\textbf{316. (\color{red}d4\color{black}, 2015 IMOSL N1)} Determine all positive integers $M$ such that the sequence $a_0, a_1, a_2, \cdots$ defined by\[ a_0 = M + \frac{1}{2}   \qquad  \textrm{and} \qquad    a_{k+1} = a_k\lfloor a_k \rfloor   \quad \textrm{for} \, k = 0, 1, 2, \cdots \]contains at least one integer term.

\textbf{288. (\color{red}d4\color{black}, 2005 Iran)} Let $p, n \geq 2$ be positive integers with $p$ prime. Suppose that $n \mid p - 1$ and $p \mid n^{3} - 1.$ Prove that $4p - 3$ is a perfect square.

\textbf{232. (\color{red}d4\color{black}, 2019 New Zealand MO Round 2, Q4 of 5     )} Show that for all positive integers $k$, there exists a positive integer $n$ such that $n \times 2^k - 7$ is a perfect square.

\textbf{171. (\color{red}d4\color{black}, 2004 CMO, Q5 of 5)} Let \(T\) be the set of all positive integer divisors of \(2004^{100}\). What is the largest possible number of elements of a subset \(S\) of \(T\) such that no element in \(S\) divides any other element in \(S\)?

\textbf{156. (\color{red}d4\color{black}, 2016 BMO1, P6 of 6)} Consecutive positive integers \(m, m+1, m+2\) and \(m+3\) are divisible by consecutive odd positive integers \(n, n+2, n+4\) and \(n+6\) respectively. Determine the smallest possible \(m\) in terms of \(n\).

\textbf{135. (\color{red}d4\color{black}, 2005 Polish MO, Third Round, Day 1, P1 of 3)} Find all triples \(x,y,n\) of positive integers satisfying the equation \[(x-y)^n = xy.\]

\textbf{122. (\color{red}d4\color{black}, 2017 USAMO, P1 of 6)} Prove there are infinitely many pairs of coprime integers $a>1, b>1$ satisfying $a+b \mid a^b + b^a.$

\textbf{113. (\color{red}d4\color{black}, 2018 BMO2, P3)} It is well known that, for each positive integer \(n\),
\[1^3 + 2^3 + \dots + n^3 = \frac{n^2(n+1)^2}{4}\]
and so is a square. Determine whether or not there is a positive integer \(m\) such that
\[(m+1)^3 + (m+2)^3 + \dots + (2m)^3\]
is a square.

\textbf{104. (\color{red}d4\color{black}, 2018 Tournament of Towns, Senior Q3)} Prove that\\
\begin{enumerate}
    \item any integer of the form \(3k-2\), where \(k\) is an integer, can be represented as the sum of a perfect square and two perfect cubes of some integers.
    \item any integer can be represented as the sum of a perfect square and three perfect cubes of some integer.
\end{enumerate}

\textbf{82. (\color{red}d4\color{black}, 2016 NZ Camp Selection Problems, Q7)} Find all positive integers \(n\) for which the equation \[(x^2+y^2)^n = (xy)^{2016}\] has positive integer solutions.

\textbf{1. (\color{red}d4\color{black}, 2015 Romanian MoM, Q1)} Does there exist an infinite sequence of positive integers $a_1, a_2, a_3, \dots$ such that $a_m$ and $a_n$ are coprime if and only if $\lvert m - n \rvert = 1$?

\textbf{1305. (\color{red}d5\color{black}, 1995 Russia (ACPS 7.1.32))} The sequence \(a_1, a_2, \ldots\) of natural numbers satisfies \[\operatorname{GCD}(a_i,a_j) = \operatorname{GCD}(i,j)\] for all \(i \neq j\).  Prove that \(a_i = i\) for all \(i\).

\textbf{1276. (\color{red}d5\color{black}, 2010 Putnam, A4 of 6)} Prove that for each positive integer $n$, the number $10^{10^{10^n}}+10^{10^n}+10^n-1$ is not prime

\textbf{1262. (\color{red}d5\color{black}, 2007 IMOSL, N2)} Let $b,n > 1$ be integers. Suppose that for each $k > 1$ there exists an integer $a_k$ such that $b - a^n_k$ is divisible by $k$. Prove that $b = A^n$ for some integer $A$.

\textbf{1200. (\color{red}d5\color{black}, Folklore)} Prove that each non-negative integer can be represented in the form $a^2+b^2-c^2$, where $a,b,$ and $c$ are positive integers with $a<b<c$

\textbf{1186. (\color{red}d5\color{black}, 1984 IMO, P6 of 6)} Let $a,b,c,d$ be odd integers such that $0<a<b<c<d$ and $ad=bc$. Prove that if $a+d=2^k$ and $b+c=2^m$ for some integers $k$ and $m$, then $a=1$.

\textbf{1151. (\color{red}d5\color{black}, 2021 Indian Statistical Institute Entrance Examination, P3 of 8)} Prove that every positive rational number can be expressed uniquely as a finite sum of the form$$a_1+\frac{a_2}{2!}+\frac{a_3}{3!}+\dots+\frac{a_n}{n!},$$where $a_n$ are integers such that $0 \leq a_n \leq n-1$ for all $n > 1$.

\textbf{1110. (\color{red}d5\color{black}, 2016 IMOSL, N6)} Denote by $\mathbb{N}$ the set of all positive integers. Find all functions $f:\mathbb{N}\rightarrow \mathbb{N}$ such that for all positive integers $m$ and $n$, the integer $f(m)+f(n)-mn$ is nonzero and divides $mf(m)+nf(n)$.

\textbf{1095. (\color{red}d5\color{black}, 2001 Russian MO, Grade 11, Day 2, P2)} Let $a, b$ be distinct natural numbers such that $ab(a+b)$ is divisible by $a^2 + ab + b^2$. Show that $|a - b| > \sqrt[3]{ab}$.

\textbf{1067. (\color{red}d5\color{black}, Folklore)} Prove that $\phi(n)+\sigma(n) \geq 2n$ for all positive integers $n$.  Here, $\phi$ is Euler's totient function, and $\sigma(n)$ denotes the sum of the positive divisors of $n$.

\textbf{1048. (\color{red}d5\color{black}, 2020 Balkan MO, P2 of 4)} Denote $\mathbb{Z}_{>0}=\{1,2,3, \ldots\}$ the set of all positive integers. Determine all functions $f: \mathbb{Z}_{>0} \rightarrow \mathbb{Z}_{>0}$ such that, for each positive integer $n$:
\begin{enumerate} \item $\sum_{k=1}^{n} f(k)$ is a perfect square
    \item $f(n)$ divides $n^{3}$. \end{enumerate}

\textbf{919. (\color{red}d5\color{black}, 2018 Taiwan "IMOC", N2)} Find all positive integers $f: \mathbb{N} \to \mathbb{N}$ such that $$\operatorname{lcm}[f(x), y]\gcd(f(x), f(y))=f(x) f(f(y))$$ holds for all $x, y \in \mathbb{N}$.

\textbf{898. (\color{red}d5\color{black}, 2005 IMOSL, N1)} Determine all positive integers $n \geq 2$ that satisfy the following condition: For all integers $a, b$ relatively prime to $n$,
$$
    a \equiv b \pmod{n}\quad \text { if and only if } \quad a b \equiv 1\pmod{n}
$$

\textbf{815. (\color{red}d5\color{black}, 2019 USEMO, P4 of 6)} Prove that for any prime $p,$ there exists a positive integer $n$ such that \[1^n+2^{n-1}+3^{n-2}+\cdots+n^1\equiv 2020\pmod{p}.\]

\textbf{753. (\color{red}d5\color{black}, 2021 EGMO Q1)} The number 2021 is fantabulous. For any positive integer $m$, if any element of the set $\{m, 2m+1, 3m\}$ is fantabulous, then all the elements are fantabulous. Does it follow that the number $2021^{2021}$ is fantabulous?

\textbf{716. (\color{red}d5\color{black}, 2021 AMO P4)} Let $P(x)$ and $Q(x)$ be polynomials with integer coefficients such that the leading coefficient of $P(x) $ is $1$. Suppose that $P(n)^n$ divides $Q(n)^{n+1}$ for infinitely many positive integers $n$.

Prove that $P(n)$ divides $Q(n)$ for infinitely many positive integers $n$.

\textbf{710. (\color{red}d5\color{black}, 2006/7 BMO1, P6 of 6)} Let $n$ be an integer. Show that, if $2+2\sqrt{1+12n^2}$ is an integer, then it is a perfect square.

\textbf{703. (\color{red}d5\color{black}, 2019 Balkan MO, P1 of 4)} Let $\mathbb{P}$ be the set of all prime numbers. Find all functions $f:\mathbb{P}\rightarrow\mathbb{P}$ such that:
$$f(p)^{f(q)}+q^p=f(q)^{f(p)}+p^q$$holds for all $p,q\in\mathbb{P}$.

\textbf{641. (\color{red}d5\color{black}, Fermat's Christmas Theorem (adapted))} Let $a$ and $b$ be positive integers, and let $p \equiv 3 \pmod{4}$ be a prime. Suppose $p$ divides $n=a^2+b^2$. Prove that $n$ is a multiple of $p^2$.

\textbf{576. (\color{red}d5\color{black}, 2004 Croatian TST, P1 of 3)} Find all pairs $(x,y)$ of positive integers such that $x(x+y) = y^2 + 1$.

\textbf{464. (\color{red}d5\color{black}, 2018 USAMO P4 of 6)} Let $p$ be a prime, and let $a_1, \dots, a_p$ be integers. Show that there exists an integer $k$ such that the numbers $$a_1 + k, a_2 + 2k, \dots, a_p + pk$$ produce at least $\frac{1}{2} p$ distinct remainders upon division by $p.$

\textbf{436. (\color{red}d5\color{black}, 2016 IMOSL, N1)} For any positive integer $k$, denote the sum of the digits of $k$ in its decimal representation by $S(k)$. Find all polynomials $P(x)$ with integer coefficients such that for any positive integer $n \ge 2016$, the integer $P(n)$ is positive and \begin{equation*}S(P(n)) = P(S(n)).\end{equation*}

\textbf{414. (\color{red}d5\color{black}, 2014 IMOSL, N2)} Determine all pairs $(x, y)$ of positive integers such that $$\sqrt[3]{7x^2 - 13xy + 7y^2} = \vert x - y \vert + 1. $$

\textbf{394. (\color{red}d5\color{black}, 2008 CMO, P4 of 5)} Determine all functions $f$ defined on the natural numbers that take values among the natural numbers for which \begin{equation*} \left(f(n)\right)^p \equiv n \pmod{f(p)}\end{equation*} for all $n\in\mathbb{N}$ and all primes $p$.

\textbf{380. (\color{red}d5\color{black}, 1999 BMO2, P4 of 4)} Consider all numbers of the form $3n^2 + n + 1$, where $n$ is a positive integer. \begin{enumerate}\item[(i)]{How small can the sum of the digits (in base 10) of such a number be?}\item[(ii)]{Can such a number have the sum of its digits (in base 10) equal to 1999?}\end{enumerate}

\textbf{374. (\color{red}d5\color{black}, 2014 IMOSL, N1)} Let $n \ge 2$ be an integer, and let $A_n$ be the set \[A_n = \{2^n  - 2^k\mid k \in \mathbb{Z},\, 0 \le k < n\}.\] Determine the largest positive integer that cannot be written as the sum of one or more (not necessarily distinct) elements of $A_n$.

\textbf{347. (\color{red}d5\color{black}, 2020 AMO, Q8 of 8)} Prove that for each integer $k$ satisfying $2 \leq k \leq 100$, there are positive integers $b_2, b_3, \dots, b_{101}$ such that
\[b_2^2 + b_3^3 + \dots + b_k^k = b_{k+1}^{k+1} + b_{k+2}^{k+2} + \dots + b_{101}^{101}.\]

\textbf{310. (\color{red}d5\color{black}, 2001 Romania TST3, P3 of 3)} Let $p, q \in \mathbb{N}$ be coprime. A set $S$ of non-negative integers is called \emph{ideal} if \begin{enumerate}\item{$0 \in S$}\item{$n \in S \Rightarrow n+p, \, n+q \in S$.}\end{enumerate} How many ideal sets are there?

\textbf{298. (\color{red}d5\color{black}, 2019 Korea MO, Intermediate Q4 of 8)} For any positive integer $n$, the sequence of positive integers $a_1, a_2, \dots, a_n, \dots$ satisfies the following inequality:
$$\left ( a_1 + a_2 + \dots + a_n \right ) \left ( \frac{1}{a_1} + \frac{1}{a_2} + \dots + \frac{1}{a_n} \right ) \leq n^2 + 2019.$$
Show that this is a constant sequence.

\textbf{268. (\color{red}d5\color{black}, 2017 IMO, P1 of 6)} For each integer $a_0 > 1$, define the sequence $a_0, a_1, a_2, \dots$ by:
\begin{equation*}
    a_{n+1} =
    \begin{cases}
        \sqrt{a_n} & \text{if } \sqrt{a_n} \text{ is an integer,} \\
        a_n + 3    & \text{otherwise,}
    \end{cases}
    \quad \text{for each } n \geq 0
\end{equation*}
Determine all values of $a_0$ for which there is a number $A$ such that $a_n = A$ for infinitely many values of $n$.

\textbf{226. (\color{red}d5\color{black}, 2005 Canada MO P5 of 5)} Let's say that an ordered triple of integers \((a,b,c)\) is \textit{n-powerful} if \(a \leq b \leq c\), \(\gcd(a,b,c) = 1\), and \(a^n + b^n + c^n\) is divisible by \(a + b + c\). For example, \((1,2,2)\) is 5-powerful.\begin{enumerate}
    \item Determine all ordered triples (if any) which are \(n\)-powerful for all \(n \geq 1\).
    \item Determine all ordered triples (if any) which are 2004-powerful and 2005-powerful, but not 2007-powerful.
\end{enumerate}

\textbf{213. (\color{red}d5\color{black}, 2018 NZIMO Camp PS2, P5 of 6)} Determine for which \(n\) there exists \(n\) distinct integers such that the sum of their squares equals the sum of their cubes.

\textbf{205. (\color{red}d5\color{black}, 2015 Pan-African MO, P3 of 6)} Let $a{1}, a{1}, \dots, a{11}$ be integers. Prove that there are numbers $b{1}, b{2}, \dots, b{11},$ each $b{i}$ equal to $-1, 0, 1,$ but not all being $0$ such that the number $$N = a{1}b{1} + a{2}b{2} \dots + a{11}b_{11}$$ is divisible by $2015.$

\textbf{179. (\color{red}d5\color{black}, Brazilian MO 2013 Q2 )} Ankan and Bankan play the following game:



Given a fixed finite set of positive integers $\mathcal{A}$ known by both players, Ankan picks a number $a \in \mathcal{A}$ but doesn't reveal it to Bankan. Bankan then picks an arbitrary positive integer $b$ (not necessarily in $\mathcal{A}$). Ankan now reveals the number of divisors of the number $a \times b$.



Show that Bankan can choose $b$ in such a way that he can determine what $a$ is.

\textbf{172. (\color{red}d5\color{black}, 2019 AMO, Day 2 Q8)} Let $n = 16^{3^r} - 4^{3^r} + 1$ for some positive integer $r$. Show that $2^{n-1}-1$ is divisible by $n$.

\textbf{164. (\color{red}d5\color{black}, 2016 APMO, P2 of 5)} A positive integer is called \emph{fancy} if it can be expressed in the form \[2^{a_1} + 2^{a_2} + \cdots + 2^{a_{100}},\] where \(a_1 , a_2 , \dots, a_{100}\) are non-negative integers that are not necessarily distinct.



\makebox[1.5em]{} Find the smallest positive integer \(n\) such that no multiple of \(n\) is a fancy number.

\textbf{40. (\color{red}d5\color{black}, 2002 IMO (43rd), Q4)} The positive divisors of the integer $n > 1$ are  $d_1 < d_2 < \dots < d_k$, so that $d_1 = 1, d_k = n$. Let $d = d_1d_2 + d_2d3 + \dots + d{k-1}d_k$. Show that $d < n^2$ and find all $n$ for which $d$ divides $n^2$.

\textbf{18. (\color{red}d5\color{black}, 2015/16 BMO1, Q6)} A positive integer is called \emph{charming} if it is equal to 2 or is of the form $3^i5^j$ where $i$ and $j$ are non-negative integers. Prove that every positive integer can be written as a sum of different charming integers.

\textbf{13. (\color{red}d5\color{black}, 2018 EGMO, Q2)} Consider the set $A = \{1 + \frac 1k : k = 1, 2, 3, \dots \}$
\begin{enumerate}
    \item Prove that every integer $x \geq 2$ can be written as the product of one or more elements of $A$, which are not necessarily different.
    \item For every integer $x \geq 2$, let $f(x)$ denote the minimum integer such that $x$ can be written as the product of $f(x)$ elements of $A$, which are not necessarily different.  Prove that there exist infinitely many pairs $(x, y)$ of integers with $x \geq 2, y \geq 2$, and $f(xy) < f(x) + f(y)$. (Pairs $(x_1, y_1)$ and $(x_2, y_2)$ are different if $x_1 \neq x_2$ or $y_1 \neq y_2$.)
\end{enumerate}

\textbf{5. (\color{red}d5\color{black}, 2005 IMO, Q4)} Determine all positive integers relatively prime to all the terms of the infinite sequence \[a_n=2^n+3^n+6^n -1,\, n\geq 1.\]

\textbf{1369. (\color{red}d6\color{black}, 2006 IMOSL, N3)} We define a sequence $ \left(a_{1},a_{2},a_{3},\ldots \right)$ by
\[ a_{n} = \frac {1}{n}\left(\left\lfloor\frac {n}{1}\right\rfloor + \left\lfloor\frac {n}{2}\right\rfloor + \cdots + \left\lfloor\frac {n}{n}\right\rfloor\right), \] where $\lfloor x\rfloor$ denotes the integer part of $x$.

\begin{enumerate}
    \item Prove that $a_{n+1}>a_n$ infinitely often.
    \item Prove that $a_{n+1}<a_n$ infinitely often.
\end{enumerate}

\textbf{1034. (\color{red}d6\color{black}, 2021 China National HS ML (Exam 2 of Prelims), P3 of 4)} Let \(n \geq 4\) be an integer.  Prove that if \(n\) divides \(2^n-2\), then \(\frac{2^n-2}{n}\) is a composite number.

\textbf{1011. (\color{red}d6\color{black}, 2021 BMO2, P4 of 4)} Matthew writes down a sequence $a_1,a_2,a_3,\ldots$ of positive integers. Each $a_n$ is the smallest
positive integer, different from all previous terms in the sequence, such that the mean of the
terms $a_1,a_2,\ldots,a_n$ is an integer. Prove that the sequence defined by $a_i-i$ for $i=1,2,3,\ldots$ contains every integer exactly once.

\textbf{970. (\color{red}d6\color{black}, 2006 China TST, Day 1, P3 of 3)} Find all positive integer pairs $(a,n)$ such that \[ \frac{(a+1)^{n}-a^{n}}{n}\] is an integer.

\textbf{857. (\color{red}d6\color{black}, 2011 IMOSL, N2)} For any integer $d > 0,$ let $f(d)$ be the smallest possible integer that has exactly $d$ positive divisors (so for example we have $f(1)=1, f(5)=16,$ and $f(6)=12$). Prove that for every integer $k \geq 0$ the number $f\left(2^k\right)$ divides $f\left(2^{k+1}\right).$


\textbf{816. (\color{red}d6\color{black}, 2021 USAMO, P4 of 6)} A finite set $S$ of positive integers has the property that, for each $s \in S$, and each positive integer divisor $d$ of $s$, there exists a unique element $t \in S$ satisfying $\operatorname{gcd}(s, t)=d$. The elements $s$ and $t$ could be equal.
Given this information, find all possible values for the number of elements of $S$.

\textbf{739. (\color{red}d6\color{black}, 2021 CMO, P4 of 5)} A function $f$ from the positive integers to the positive integers is called Canadian if it satisfies
$\gcd(f(f(x)), f(x+y))=\gcd(x, y)$
for all pairs of positive integers $x$ and $y$.
Find all positive integers $m$ such that $f(m)=m$ for all Canadian functions $f$.

\textbf{724. (\color{red}d6\color{black}, 2002 Putnam, A5)} Define a sequence by $a_0=1$, together with the rules $a_{2n+1}=a_n$ and $a_{2n+2}=a_n+a_{n+1}$ for each integer $n\ge0$. Prove that every positive rational number appears in the set $ \left\{ \tfrac {a_{n-1}}{a_n}: n \ge 1 \right\} = \left\{ \tfrac {1}{1}, \tfrac {1}{2}, \tfrac {2}{1}, \tfrac {1}{3}, \tfrac {3}{2}, \cdots \right\} $.

\textbf{697. (\color{red}d6\color{black}, 2015 EGMO, P3 of 6)} Let $n, m$ be integers greater than $1,$ and let $a_{1}, a_{2}, \ldots, a_{m}$ be positive integers not greater than $n^{m}$. Prove that there exist positive integers $b_{1}, b_{2}, \ldots, b_{m}$ not greater than $n$, such that
$$
    \gcd\left(a_{1}+b_{1}, a_{2}+b_{2}, \ldots, a_{m}+b_{m}\right)<n
$$
where $\gcd\left(x_{1}, x_{2}, \ldots, x_{m}\right)$ denotes the greatest common divisor of $x_{1}, x_{2}, \ldots, x_{m}$

\textbf{696. (\color{red}d6\color{black}, 2009 IMO, P1 of 6)} Let $n$ be a positive integer and let $a_{1}, a_{2}, a_{3}, \ldots, a_{k}$ $(k \geq 2)$ be distinct integers in the set $1,2, \ldots, n$ such that $n$ divides $a_{i}\left(a_{i+1}-1\right)$ for $i=1,2, \ldots, k-1$. Prove that $n$ does not divide $a_{k}\left(a_{1}-1\right)$.

\textbf{669. (\color{red}d6\color{black}, 2015 IMOSL, N4)} Suppose that $a_0, a_1, \dots $ and $b_0, b_1, \dots$ are two sequences of positive integers such that $a_0, b_0 \ge 2$ and \[ a_{n+1} = \gcd{(a_n, b_n)} + 1, \qquad b_{n+1} = \operatorname{lcm}{(a_n, b_n)} - 1. \] Show that the sequence $a_n$ is eventually periodic; in other words, there exist integers $N \ge 0$ and $t > 0$ such that $a_{n+t} = a_n$ for all $n \ge N$.

\textbf{627. (\color{red}d6\color{black}, 2020 AUS → UNK F3, P1 of 3)} Consider a sequence $a_1, a_2, \dots $ of non-zero integers such that $a_{m + n} \mid a_{m} - a_{n}$ for all $m, n \in \mathbb{N}.$
\begin{enumerate}
    \item Show that the sequence can be unbounded.
    \item Do there necessarily exist $i \neq j$ such that $a_i = a_j$?
\end{enumerate}

\textbf{493. (\color{red}d6\color{black}, 2017 IMOSL N3)} Determine all integers $ n\geq 2$ having the following property: for any integers $a_1,a_2,\ldots, a_n$ whose sum is not divisible by $n$, there exists an index $1 \leq i \leq n$ such that none of the numbers $$a_i,a_i+a_{i+1},\ldots,a_i+a_{i+1}+\ldots+a_{i+n-1}$$is divisible by $n$. Here, we let $a_i=a_{i-n}$ when $i >n$.

\textbf{431. (\color{red}d6\color{black}, 2015 Balkan MO, Q4)} Prove that among $20$ consecutive positive integers there is an integer $d$ such that for every positive integer $n$ the following inequality holds
$$n \sqrt{d} \left\{n \sqrt {d} \right \} > \dfrac{5}{2}$$
where by $\left \{x \right \}$ denotes the fractional part of the real number $x$. The fractional part of the real number $x$ is defined as the difference between the largest integer that is less than or equal to $x$ to the actual number $x$.

\textbf{375. (\color{red}d6\color{black}, OMO Fall 2019, Q21)} Let $p$ and $q$ be prime numbers such that $(p-1)^{q-1}-1$ is a positive integer that divides $(2q)^{2p}-1$. Compute the sum of all possible values of $pq$.

\textbf{145. (\color{red}d6\color{black}, 2015 IMO Shortlist, N3)} Let $m$ and $n$ be positive integers such that $m>n$. Define $x_k=\frac{m+k}{n+k}$ for $k=1,2,\ldots,n+1$. Prove that if all the numbers $x_1,x_2,\ldots,x_{n+1}$ are integers, then $x_1x_2\ldots x_{n+1}-1$ is divisible by an odd prime.

\textbf{129. (\color{red}d6\color{black}, 2018 EGMO, P6 of 6     )} Prove that for every real number $t$ such that $0 < t < \tfrac{1}{2}$ there exists a positive integer $n$ with the following property: for every set $S$ of $n$ positive integers there exist two different elements $x$ and $y$ of $S$, and a non-negative integer $m$ (i.e. $m \ge 0 $), such that $ |x-my|\leq ty$ Determine whether for every real number $t$ such that $0 < t < \tfrac{1}{2} $ there exists an infinite set $S$ of positive integers such that $|x-my| > ty$for every pair of different elements $x$ and $y$ of $S$ and every positive integer $m$ (i.e. $m > 0$).

\textbf{100. (\color{red}d6\color{black}, 2016 IMO, Problem 4)} A set of positive integers is called \textit{fragrant} if it contains at least two elements and each of its elements has a prime factor in common with at least one of the other elements. Let \(P(n) = n^2 + n + 1.\) What is the least possible value of the positive integer \(b\) such that there exists a non-negative integer \(a\) for which the set
\[\{P\left(a+1\right), P\left(a+2\right), \dots, P\left(a+b\right)\}\]
is fragrant?

\textbf{45. (\color{red}d6\color{black}, 2017 IMOSL, N2)} Let $p \geq 2$ be a prime number. Eduardo and Fernando play the following game making moves alternately: in each move, the current player chooses an index $i$ in the set ${0,1,\dots,p-1}$ that was not chosen before by either of the two players and then chooses an element $a_i$ of the set ${0,1,2,3,4,5,6,7,8,9}$. Eduardo has the first move. The game ends after all the indices $i \in {0,1,\dots,p-1}$ have been chosen. Then the following number is computed: \begin{equation}M = a_0 + 10\cdot a1 + \dots + 10^{p-1} \cdot a{p-1} = \sum_{j=0}^{p-1} a_j \cdot 10^j.\end{equation} The goal of Eduardo is to make the number $M$ divisible by $p$, and the goal of Fernando is to prevent this. Prove that Eduardo has a winning strategy.

\textbf{33. (\color{red}d6\color{black}, 2013 IMO Shortlist, N3)} Prove that there exist infinitely many positive integers $n$ such that the largest prime divisor of $n^4 + n^2 + 1$ is equal to the largest prime divisor of $(n+1)^4 + (n+1)^2 + 1$.

\textbf{1383. (\color{red}d7\color{black}, 2016 IMOSL, C2)} Find all positive integers  for which all positive divisors of  can be put into the cells of a rectangular table under the following constraints:
\begin{itemize}
    \item each cell contains a distinct divisor;
    \item the sums of all rows are equal; and
    \item the sums of all columns are equal.
\end{itemize}

\textbf{1341. (\color{red}d7\color{black}, 2020 IMO, P5 of 6)} A deck of $n > 1$ cards is given. A positive integer is written on each card. The deck has the property that the arithmetic mean of the numbers on each pair of cards is also the geometric mean of the numbers on some collection of one or more cards.
For which $n$ does it follow that the numbers on the cards are all equal?

\textbf{1327. (\color{red}d7\color{black}, 2011 IMO, P5 of 6)} Let $f$ be a function from the set of integers to the set of positive integers. Suppose that, for any two integers $m$ and $n$, the difference $f(m) - f(n)$ is divisible by $f(m- n)$. Prove that, for all integers $m$ and $n$ with $f(m) \leq f(n)$, the number $f(n)$ is divisible by $f(m)$.

\textbf{1299. (\color{red}d7\color{black}, 2019 IMO, P4 of 6)} Find all pairs $(k,n)$ of positive integers such that \[ k!=(2^n-1)(2^n-2)(2^n-4)\cdots(2^n-2^{n-1}). \]

\textbf{1292. (\color{red}d7\color{black}, 2022 IMO, P5 of 6)} Find all triples $(a,b,p)$ of positive integers with $p$ prime and $$a^p=b!+p.$$

\textbf{1250. (\color{red}d7\color{black}, 2005 IMOSL, N6)} Let $a$, $b$ be positive integers such that $b^n+n$ is a multiple of $a^n+n$ for all positive integers $n$. Prove that $a=b$.

\textbf{1208. (\color{red}d7\color{black}, 2013 IMO, P1 of 6)} Assume that $k$ and $n$ are two positive integers. Prove that there exist positive integers $m_1 , \dots , m_k$ such that \[1+\frac{2^k-1}{n}=\left(1+\frac1{m_1}\right)\cdots \left(1+\frac1{m_k}\right).\]

\textbf{1181. (\color{red}d7\color{black}, 2022 DIMO, P5 of 6)} For each positive integer $n$, define the following set: \[Q_n=\{a^n+b^n+c^n-nabc\mid (a,b,c)\in \mathbb{Z}_{>0}^3\}\] Find the least positive integer $k$ for which $Q_k^{(c)}\subseteq Q_k$ for some integral constant $c\geq 2$.\\ For a set $S$, we define $S^{(c)}$ to be the set whose elements are all the $c$-th powers of the elements of $S$.

\textbf{1082. (\color{red}d7\color{black}, 2000 IMO, P5 of 6)} Does there exist a positive integer $ n$ such that $ n$ has exactly 2000 prime divisors and $ n$ divides $ 2^n + 1$?

\textbf{1062. (\color{red}d7\color{black}, 2021 ELMO, P2 of 6)} Let $n > 1$ be an integer and let $a_1, a_2, \ldots, a_n$ be integers such that $n \mid a_i-i$ for all integers $1 \leq i \leq n$. Prove there exists an infinite sequence $b_1,b_2, \ldots$ such that
\begin{itemize}
    \item $b_k\in\{a_1,a_2,\ldots, a_n\}$ for all positive integers $k$, and
    \item $\sum\limits_{k=1}^{\infty}\frac{b_k}{n^k}$ is an integer.
\end{itemize}

\textbf{1020. (\color{red}d7\color{black}, 2017 China TST Test 4, P4 of 6)} Given integers $d>1,m$, prove that there exist integers $k>l>0$, such that $$(2^{2^k}+d,2^{2^l}+d)>m.$$

\textbf{984. (\color{red}d7\color{black}, 2018 RMM, P4 of 6)} Let $a$, $b$, $c$, $d$ be positive integers such that $ad \neq bc$ and $\gcd(a,b,c,d) = 1$. Let $S$ be the set of values attained by $\gcd(an+b,cn+d)$ as $n$ runs through the positive integers. Show that $S$ is the set of all positive divisors of some positive integer.

\textbf{964. (\color{red}d7\color{black}, USEMO 2021, P2 of 6)} Find all integers $n \geq 1$ such that $2^n - 1$ has exactly $n$ positive integer divisors.

\textbf{914. (\color{red}d7\color{black}, 2018 Taiwan "IMOC", N1)} Find all functions $f: \mathbb{N} \rightarrow \mathbb{N}$ satisfying
$$
    x+f^{f(x)}(y) \mid 2(x+y)
$$
holds for all $x, y \in \mathbb{N}$

\textbf{865. (\color{red}d7\color{black}, 2016 BMO2, P4 of 4)} Suppose that $p$ is a prime number and that there are different positive integers $u$ and $v$ such that $p^2$ is the mean of $u^2$ and $v^2$. Prove that $2p-u-v$ is a square or twice a square.

\textbf{858. (\color{red}d7\color{black}, 2021 APMO, P2 of 5)} For a polynomial $P$ and a positive integer $n$, define $P_{n}$ as the number of positive integer pairs $(a, b)$ such that $a<b \leq n$ and $|P(a)|-|P(b)|$ is divisible by
$n$. Determine all polynomial $P$ with integer coefficients such that $P_{n} \leq 2021$ for all positive integers $n$.

\textbf{830. (\color{red}d7\color{black}, China TST 2021, Test 3, P2 of 6)} Given distinct positive integer $ a_1,a_2,…,a_{2020} $. For $ n \ge 2021 $, $a_n$ is the smallest number different from $a_1,a_2,…,a_{n-1}$ which doesn't divide $a_{n-2020}...a_{n-2}a_{n-1}$. Prove that every number large enough appears in the sequence.

\textbf{718. (\color{red}d7\color{black}, 2015 IMOSL, N6)} Let $\mathbb{Z}_{>0}$ denote the set of positive integers. Consider a function $f: \mathbb{Z}_{>0} \to \mathbb{Z}_{>0}$. For any $m, n \in \mathbb{Z}_{>0}$ we write $f^n(m) = \underbrace{f(f(\ldots f}_{n}(m)\ldots))$. Suppose that $f$ has the following two properties:

(i) if $m, n \in \mathbb{Z}_{>0}$, then $\frac{f^n(m) - m}{n} \in \mathbb{Z}_{>0}$;
(ii) The set $\mathbb{Z}_{>0} \setminus \{f(n) \mid n\in \mathbb{Z}_{>0}\}$ is finite.

Prove that the sequence $f(1) - 1, f(2) - 2, f(3) - 3, \ldots$ is periodic.

\textbf{711. (\color{red}d7\color{black}, 2019 IMOSL, N3)} We say that a set $S$ of integers is rootiful if, for any positive integer $n$ and any $a_0, a_1, \cdots, a_n \in S$, all integer roots of the polynomial $a_0+a_1x+\cdots+a_nx^n$ are also in $S$. Find all rootiful sets of integers that contain all numbers of the form $2^a - 2^b$ for positive integers $a$ and $b$.

\textbf{501. (\color{red}d7\color{black}, 2019 FKMO Day 1, P3 of 6)} For a positive integer $x$, define a sequence $a_0, a_1, a_2, \dots$ according to the following rules: \[a_0 = 1, a_1 = x+1 \text{ and } a_{n+2} = xa_{n+1} - a_n \text{ for all } n \geq 0.\] Prove that there exist infinitely many positive integerx $x$ such that this sequence does not contain a prime number.

\textbf{354. (\color{red}d7\color{black}, 2016 China TST Q6 of 6)} Let $m, n$ be naturals satisfying $n \geq m \geq 2$ and let $S$ be a set consisting of $n$ naturals. Prove that $S$ has at least $2^{n - m + 1}$ distinct subsets, each whose sum is divisible by $m.$

\textbf{256. (\color{red}d7\color{black}, 2019 AMOC School of Excellence N6)} Let \(a\) and \(b\) be two positive integers such that \[b+1 \mid a^2 + 1 \quad \text{ and } \quad a+1 \mid b^2 + 1.\] Prove that both \(a\) and \(b\) are odd.

\textbf{137. (\color{red}d7\color{black}, 2015 USAMO, P5 of 6)} Let $a$, $b$, $c$, $d$, $e$ be distinct positive integers such that $a^4+b^4=c^4+d^4=e^5$. Show that $ac+bd$ is a composite number.

\textbf{130. (\color{red}d7\color{black}, 2016 Balkan MO Shortlist N5       )} A positive integer is called downhill if the digits in its decimal representation form a nonstrictly decreasing sequence from left to right. Suppose that a polynomial $P(x)$ with rational coefficients takes on an integer value for each downhill positive integer $x$. Is it necessarily true that $P(x)$ takes on an integer value for each integer $x$?

\textbf{16. (\color{red}d7\color{black}, 2005 Korean MO (18th), Final Round, Q5)} Find all positive integers $m$ and $n$ such that both $3^m +1$ and $3^n +1$ are divisible by $mn$.

\textbf{1293. (\color{red}d8\color{black}, 2022 Iran TST P2 of 9)} For a positive integer $n$, let $\tau(n)$ and $\sigma(n)$ be the number of positive divisors of $n$ and the sum of positive divisors of $n$, respectively. let $a$ and $b$ be positive integers such that $\sigma(a^n)$ divides $\sigma(b^n)$ for all $n\in \mathbb{N}$. Prove that each prime factor of $\tau(a)$ divides $\tau(b)$.

\textbf{1272. (\color{red}d8\color{black}, 2021 RMM, P2 of 6)} Xenia and Sergey play the following game. Xenia thinks of a positive integer $N$ not exceeding $5000$. Then she fixes $20$ distinct positive integers $a_1, a_2, \cdots, a_{20}$ such that, for each $k = 1,2,\cdots,20$, the numbers $N$ and $a_k$ are congruent modulo $k$. By a move, Sergey tells Xenia a set $S$ of positive integers not exceeding $20$, and she tells him back the set $\{a_k : k \in S\}$ without spelling out which number corresponds to which index. How many moves does Sergey need to determine for sure the number Xenia thought of?

\textbf{1216. (\color{red}d8\color{black}, 2022 Israel Olympic Revenge, P2 of 4)} A triple $(a,b,c)$ of positive integers is called \textbf{strong} if the following holds: for every integer $m>1$, the number $a+b+c$ does not divide $a^m+b^m+c^m$. The \textbf{sum} of a strong triple $(a,b,c)$ is defined as $a+b+c$.

Prove that there exists an infinite set of strong triples with pairwise coprime sums.

\textbf{1202. (\color{red}d8\color{black}, 2022 CAMO, P5 of 6)} Prove or disprove the following assertion: for each positive integer $k$, we have \[\gcd\left((2k-1)^{2k-1}+(2k+1)^{2k+1},(2k-1)^{2k+1}+(2k+1)^{2k-1}\right)=4k^2\]

\textbf{1167. (\color{red}d8\color{black}, Vietnam TST)} Find all positive integers $a,b,c,d$ such that $a+b+d^2=4abc$.

\textbf{1139. (\color{red}d8\color{black}, 2013 IMOSL, N4)} Determine whether there exists an infinite sequence of nonzero digits $a_0, a_1, a_2, \dots $ and a positive integer $N$ such that for every integer $k>N$, the number $\overline{a_ka_{k-1}\cdots a_0}$ is a perfect square.

\textbf{1090. (\color{red}d8\color{black}, 2016 IMOSL, N5)} Let $a$ be a positive integer which is not a square number. Denote by $A$ the set of all positive integers $k$ such that
\begin{align} k&=\frac{x^2-a}{x^2-y^2}\end{align}
for some integers $x$ and $y$ with $x>\sqrt{a}$. Denote by $B$ the set of all positive integers $k$ so that (1) is satisfied for some integers $x$ and $y$ with $0\leq x<\sqrt{a}$. Prove that $A=B$.

\textbf{985. (\color{red}d8\color{black}, 2019 China TST Test 4, P6 of 6)} Given positive integer $n,k$ such that $2 \le n <2^k$. Prove that there exist a subset $A$ of $\{0,1,\cdots,n\}$ such that for any $x \neq y \in A$, ${y\choose x}$ is even, and$$|A| \ge \frac{{k\choose \lfloor \frac{k}{2} \rfloor}}{2^k} \cdot (n+1)$$

\textbf{928. (\color{red}d8\color{black}, 2013 USA TSTST, P8)} Define a function \(f : \mathbb{N} \to \mathbb{N}\) by \(f(1) = 1\), \(f(n+1) = f(n) + 2^{f(n)}\) for every positive integer \(n\). Prove that \(f(1), f(2), \dots, f(3^{2013})\) leave distinct remainders when divided by \(3^{2013}\).

\textbf{879. (\color{red}d8\color{black}, 2020 IMOSL, N5)} Determine all functions $f$ defined on the set of all positive integers and taking non-negative integer values, satisfying the three conditions:
(i) $f(n) \neq 0$ for at least one $n$;
(ii) $f(x y)=f(x)+f(y)$ for every positive integers $x$ and $y$;
(iii) there are infinitely many positive integers $n$ such that $f(k)=f(n-k)$ for all $k<n$.

\textbf{831. (\color{red}d8\color{black}, 2020 USA TST, P5 of 6)} Find all integers $n \ge 2$ for which there exists an integer $m$ and a polynomial $P(x)$ with integer coefficients satisfying the following three conditions:
\begin{itemize}
    \item $m > 1$ and $\gcd(m,n) = 1$;
    \item the numbers $P(0)$, $P^2(0)$, $\ldots$, $P^{m-1}(0)$ are not divisible by $n$; and
    \item $P^m(0)$ is divisible by $n$.
\end{itemize}
Here $P^k$ means $P$ applied $k$ times, so $P^1(0) = P(0)$, $P^2(0) = P(P(0))$, etc.

\textbf{768. (\color{red}d8\color{black}, 2011 IMOSL, N7)} Let $p$ be an odd prime number. For every integer $a,$ define the number $S_a = \sum^{p-1}_{j=1} \frac{a^j}{j}.$ Let $m,n \in \mathbb{Z},$ such that $S_3 + S_4 - 3S_2 = \frac{m}{n}.$ Prove that $p$ divides $m.$

\textbf{747. (\color{red}d8\color{black}, 2021 China National Olympiad, P6 of 6)} Find $f: \mathbb{Z}_{+} \rightarrow \mathbb{Z}_{+},$ such that for any $x, y \in \mathbb{Z}_{+}, f(f(x)+y) | x+f(y)$

\textbf{691. (\color{red}d8\color{black}, 2016 Serbia TST, P6 of 6)} Let $a_1, a_2, \dots, a_{2^{2016}}$ be positive integers not bigger than $2016$. We know that for each $n \leq 2^{2016}$, $a_1a_2 \dots a_{n} +1 $ is a perfect square. Prove that for some $i $, $a_i=1$.

\textbf{677. (\color{red}d8\color{black}, 2013 EGMO, P3 of 6)} Let $n$ be a positive integer.

(a) Prove that there exists a set $S$ of $6n$ pairwise different positive integers, such that the least common multiple of any two elements of $S$ is no larger than $32n^2$.

(b) Prove that every set $T$ of $6n$ pairwise different positive integers contains two elements the least common multiple of which is larger than $9n^2$.

\textbf{600. (\color{red}d8\color{black}, 2019 IMOSL, N6)} Let $H = \{ \lfloor i\sqrt{2}\rfloor : i \in \mathbb Z_{>0}\} = \{1,2,4,5,7,\dots \}$ and let $n$ be a positive integer. Prove that there exists a constant $C$ such that, if $A\subseteq \{1,2,\dots, n\}$ satisfies $|A| \ge C\sqrt{n}$, then there exist $a,b\in A$ such that $a-b\in H$. (Here $\mathbb Z_{>0}$ is the set of positive integers, and $\lfloor z\rfloor$ denotes the greatest integer less than or equal to $z$.)

\textbf{543. (\color{red}d8\color{black}, 2020 ELMO, P2 of 6)} Define the Fibonacci numbers by $F_1 = F_2 = 1$ and $F_n = F_{n - 1} + F_{n -2}$ for $n \geq 3.$ Let $k$ be a positive integer. Suppose that for every positive integer $m$ there exists a positive integer $n$ such that $m \mid F_n - k.$ Must $k$ be a Fibonacci number?

\textbf{536. (\color{red}d8\color{black}, PEN E8)} Show that for all integer $k>1$, there are infinitely many natural numbers $n$ such that $k \cdot 2^{2^n} + 1$ is composite.

\textbf{509. (\color{red}d8\color{black}, 2015 China TST 3, P3 of 3)} Let $a,b$ be two integers such that their gcd has at least two distinct prime factors. Let $S =  \{ x \mid x \in \mathbb{N}, x \equiv a \pmod b \} $ and call $ y \in S$ irreducible if it cannot be expressed as product of two or more elements of $S$ (not necessarily distinct). Show there exists $t$ such that any element of $S$ can be expressed as product of at most $t$ irreducible elements.

\textbf{404. (\color{red}d8\color{black}, PEN)} The integers are partitioned into finitely many (at least two) disjoint arithmetic progressions. Prove that two of these arithmetic progressions have the same common difference.

\textbf{382. (\color{red}d8\color{black}, 2015 Iran TST 1, Day 1, P3 of 3)} Let $ b_1<b_2<b_3<\dots $ be the sequence of all natural numbers which are sum of squares of two natural numbers.
Prove that there exists infinite natural numbers like $m$ which $b_{m+1}-b_m=2015$ .

\textbf{291. (\color{red}d8\color{black}, 2013 EGMO, Q3 of 6)} Let $n$ be a positive integer.

(a) Prove that there exists a set $S$ of $6n$ pairwise different positive integers, such that the least common multiple of any two elements of $S$ is no larger than $32n^2$.

(b) Prove that every set $T$ of $6n$ pairwise different positive integers contains two elements the least common multiple of which is larger than $9n^2$.

\textbf{278. (\color{red}d8\color{black}, 2008 IMO Q3)} Prove that there are infinitely many positive integers $n$ such that $n^{2} + 1$ has a prime divisor greater than $2n + \sqrt{2n}.$

\textbf{271. (\color{red}d8\color{black}, 2018 IMO SL N5)} Four positive integers $x,$ $y,$ $z$ and $t$ satisfy the relations $$xy - zt = x + y = z + t$$ Is it possible that both $xy$ and $zt$ are perfect squares?

\textbf{208. (\color{red}d8\color{black}, 1995 IMO SL N7)} Does there exist an integer $n > 1$ which satisfies the following condition? The set of positive integers can be partitioned into $n$ nonempty subsets, such that an arbitrary sum of $n - 1$ integers, one taken from each of any $n - 1$ of the subsets, lies in the remaining subset.

\textbf{180. (\color{red}d8\color{black}, 2016 RMMSL N2)} Let $p$ be a prime number. Prove that for all but finitely many primes $q,$ the sum
$\sum_{k=1}^{\lfloor{\frac{q}{p}}\rfloor}  {k^{p - 1}}$
Is not divisible by $q$.
\textbf{1350. (\color{red}d9\color{black}, 2021 IMOSL N6)} Determine all integers
$n \geq 2$ with the following property:
every $n$ pairwise distinct integers whose sum
is not divisible by $n$ can be arranged in
some order $a_1,a_2,\ldots, a_n$ so that $n$
divides $1\cdot a_1+2\cdot a_2+\cdots+n\cdot a_n$.

\textbf{1244. (\color{red}d9\color{black}, 2019 USA TST, P2 of 6)} Let $\mathbb{Z}/n\mathbb{Z}$ denote the set of integers considered modulo $n$ (hence $\mathbb{Z}/n\mathbb{Z}$ has $n$ elements). Find all positive integers $n$ for which there exists a bijective function $g: \mathbb{Z}/n\mathbb{Z} \to \mathbb{Z}/n\mathbb{Z}$, such that the $101$ functions \[g(x),\quad g(x)+x,\quad g(x)+2x,\quad \dots,\quad g(x)+100x\] are all bijections on $\mathbb{Z}/n\mathbb{Z}$.

\textbf{1140. (\color{red}d9\color{black}, 2021 ICMC Round 2, P2 of 4)} Let $p > 3$ be a prime number. A sequence of $p-1$ integers $a_1,a_2, \dots, a_{p-1}$ is called \emph{wonky} if they are distinct modulo \(p\) and $a_ia_{i+2} \not\equiv a_{i+1}^2 \pmod p$ for all \(i \in \{1, 2, \dots, p-1\}\), where \(a_p = a_1\) and \(a_{p+1} = a_2\). Does there always exist a wonky sequence such that $$a_1a_2, \qquad a_1a_2+a_2a_3, \qquad \dots, \qquad a_1a_2+\cdots +a_{p-1}a_1,$$are all distinct modulo $p$?

\textbf{853. (\color{red}d9\color{black}, 2020 IMOSL, N7)} Let $\mathcal{S}$ be a set consisting of $n \ge 3$ positive integers, none of which is a sum of two other distinct members of $\mathcal{S}$. Prove that the elements of $\mathcal{S}$ may be ordered as $a_1, a_2, \dots, a_n$ so that $a_i$ does not divide $a_{i - 1} + a_{i + 1}$ for all $i = 2, 3, \dots, n - 1$.

\textbf{818. (\color{red}d9\color{black}, 2012 USA TSTST, P3 of 9)} Let $\mathbb N$ be the set of positive integers. Let $f: \mathbb N \to \mathbb N$ be a function satisfying the following two conditions:
\smallbreak
(a) $f(m)$ and $f(n)$ are relatively prime whenever $m$ and $n$ are relatively prime.
\smallbreak
(b) $n \le f(n) \le n+2012$ for all $n$.
\bigbreak
Prove that for any natural number $n$ and any prime $p$, if $p$ divides $f(n)$ then $p$ divides $n$.

\textbf{804. (\color{red}d9\color{black}, 2018 IMOSL, N6)} Let $f : \{ 1, 2, 3, \dots \} \to \{ 2, 3, \dots \}$ be a function such that $f(m + n) | f(m) + f(n) $ for all pairs $m,n$ of positive integers. Prove that there exists a positive integer $c > 1$ which divides all values of $f$.

\textbf{762. (\color{red}d9\color{black}, 2015 IMOSL, N7)} Let $\mathbb{Z}_{>0}$ denote the set of positive integers. For any positive integer $k$, a function $f: \mathbb{Z}_{>0} \to \mathbb{Z}_{>0}$ is called $k$-good if $\gcd(f(m) + n, f(n) + m) \le k$ for all $m \neq n$. Find all $k$ such that there exists a $k$-good function.

\textbf{678. (\color{red}d9\color{black}, 2014 USA TST, P2 of 6)} Let $a_1,a_2,a_3,\ldots$ be a sequence of integers, with the property that every consecutive group of $a_i$'s averages to a perfect square. More precisely, for every positive integers $n$ and $k$, the quantity\[\frac{a_n+a_{n+1}+\cdots+a_{n+k-1}}{k}\]is always the square of an integer. Prove that the sequence must be constant (all $a_i$ are equal to the same perfect square).

\textbf{670. (\color{red}d9\color{black}, 2016 IMOSL, N8)} Find all polynomials $P(x)$ of odd degree $d$ and with integer coefficients satisfying the following property: for each positive integer $n$, there exists $n$ positive integers $x_1, x_2, \ldots, x_n$ such that $\frac12 < \frac{P(x_i)}{P(x_j)} < 2$ and $\frac{P(x_i)}{P(x_j)}$ is the $d$-th power of a rational number for every pair of indices $i$ and $j$ with $1 \leq i, j \leq n$.

\textbf{656. (\color{red}d9\color{black}, 2013 ELMO, P3 of 6)} Let $m_1,m_2,...,m_{2013} > 1$ be 2013 pairwise relatively prime positive integers and $A_1,A_2,...,A_{2013}$ be 2013 (possibly empty) sets with $A_i\subseteq \{1,2,...,m_i-1\}$ for $i=1,2,...,2013$. Prove that there is a positive integer $N$ such that
\[ N \le \left( 2\left\lvert A_1 \right\rvert + 1 \right)\left( 2\left\lvert A_2 \right\rvert + 1 \right)\cdots\left( 2\left\lvert A_{2013} \right\rvert + 1 \right) \]
and for each $i = 1, 2, ..., 2013$, there does not exist $a \in A_i$ such that $m_i$ divides $N-a$.

\textbf{621. (\color{red}d9\color{black}, 2019 IMOSL, N8)} Let $a$ and $b$ be two positive integers. Prove that the integer
\[a^2+\left\lceil\frac{4a^2}b\right\rceil\]is not a square. (Here $\lceil z\rceil$ denotes the least integer greater than or equal to $z$.)

\textbf{559. (\color{red}d9\color{black}, 2019 IMOSL, N7)} Prove that there is a constant $c>0$ and infinitely many positive integers $n$ with the following property: there are infinitely many positive integers that cannot be expressed as the sum of fewer than $cn\log(n)$ pairwise coprime $n$th powers.

\textbf{475. (\color{red}d9\color{black}, 2012 USAMO, Q3 of 3)} Determine which integers $n > 1$ have the property that there exists an infinite sequence $a_1, a_2, a_3, \ldots$ of nonzero integers such that the equality\[a_k+2a_{2k}+\ldots+na_{nk}=0\]holds for every positive integer $k$.

\textbf{461. (\color{red}d9\color{black}, 2010 IMOSL N6)} The rows and columns of a $2^n \times 2^n$ table are numbered from $0$ to $2^{n}-1.$ The cells of the table have been coloured with the following property being satisfied: for each $0 \leq i,j \leq 2^n - 1,$ the $j$-th cell in the $i$-th row and the $(i+j)$-th cell in the $j$-th row have the same colour. (The indices of the cells in a row are considered modulo $2^n$.) Prove that the maximal possible number of colours is $2^n$.

\textbf{355. (\color{red}d9\color{black}, 2012 ELMO Q6 of 6)} A diabolical combination lock has $n$ dials (each with $c$ possible states), where $n, c > 1.$ The dials are initially set to states $d_1, d_2, \dots, d_n,$ where $0 \leq d_i \leq c - 1$ for each $1 \leq i \leq n.$ Unfortunately, the current states of the dials (the $d_i$'s) are always concealed, and the initial settings of the dials are also unknown. On a given turn, one may advance each dial by an integer amount $c_i$ ($0 \leq c_i \leq c - 1$), so that every dial is now in a state $d_i ' \equiv d_i + c_i$ (mod $c$) with $0 \leq d_i ' \leq c - 1.$ After each turn, the lock opens if and only if all the dials are set to the zero state; otherwise, the lock selects a random integer $k$ and cyclically shifts the $d_i$'s by $k$ (so that for every $i,$ $d_i$ is replaced by $d_{i - k},$ where indices are taken modulo $n).$
\smallbreak
Show that the lock can always be opened, regardless of the choices of the initial configuration and the choices of $k$ (which may vary from turn to turn), if and only if $n$ and $c$ are powers of the same prime.

\textbf{348. (\color{red}d9\color{black}, 2019 USATST, P2 of 6)} Let $\mathbb{Z}/n\mathbb{Z}$ denote the set of integers considered modulo $n$ (hence $\mathbb{Z}/n\mathbb{Z}$ has $n$ elements). Find all positive integers $n$ for which there exists a bijective function $g: \mathbb{Z}/n\mathbb{Z} \to \mathbb{Z}/n\mathbb{Z}$, such that the 101 functions
\[g(x), \quad g(x) + x, \quad g(x) + 2x, \quad \dots, \quad g(x) + 100x\]are all bijections on $\mathbb{Z}/n\mathbb{Z}$.


\textbf{327. (\color{red}d9\color{black}, Chevalley's Theorem)} Let $P_1(x_1, x_2, ..., x_n), P_2(x_1, x_2, ..., x_n), ..., P_r(x_1, x_2, ..., x_n)$ be polynomials with integer coefficeints modulo $p$. Suppose further that $n$ is greater than the sum of the degrees. Prove that if $x_1 = 0, x_2 = 0, ..., x_n = 0$ is a root of all the polynomials modulo $p$, then there exists another common root modulo $p$.

\textbf{312. (\color{red}d9\color{black}, 2017 USA TST \#2, Q3 of 3)}
Prove that there are infinitely many triples $(a, b, p)$ of positive integers
with $p$ prime, $a < p$, and $b < p$, such that $(a + b)^p - a^p - b^p$ is a
multiple of $p^3$.

\textbf{292. (\color{red}d9\color{black}, 2003 IMO, Q6 of 6)} Let $p$ be a prime number. Prove that there exists a prime number $q$ such that for every integer $n,$ the number $n^{p} - p$ is not divisible by $q.$

\textbf{243. (\color{red}d9\color{black}, 2019 F3, P3 of 3)} Find all functions $f:\mathbb{N}\to\mathbb{N}$ such that for any positive integers $a$ and $b$, one of $f(a) + b$ and $f(b) + a$ divides the other (they may be equal).

\textbf{236. (\color{red}d9\color{black}, 2007 IMOSL N7)} For a prime $p$ and a given integer $n$ let $\nu_{p} (n)$ denote the exponent of $p$ in the prime factorisation of $n!.$ Given $d \in \mathbb{N}$ and ${p{1}, p{2}, \dots p{k}}$ a set of $k$ primes, show that there are infinitely many positive integers $n$ such that $d \mid \nu_{p{i}}(n)$ for all $1 \leq i \leq k.$

\textbf{235. (\color{red}d9\color{black}, 2015 IMO, P2 of 6)} Find all positive integers \((a,b,c)\) such that \[ab-c,\quad bc-a,\quad ca-b\] are all powers of \(2\).\\[10pt]



\textit{(A power of 2 is an integer of the form \(2^n\), where \(n\) is a positive integer.)}

\textbf{201. (\color{red}d9\color{black}, 2010 China TST 3 P3 of 3  )} Let $k > 1$ be an integer, set $n = 2^k + 1.$ Prove that for any positive integers $a_{1} < a_{2} < \dots < a_{n},$ the product $\displaystyle\prod_{1 \leq i < j \leq n}^{} (a_{i} + a_{j})$ has at least $k + 1$ distinct prime factors.

\textbf{200. (\color{red}d9\color{black}, 2009 IMO Q3)} Suppose that $s_1, s_2, s_3, \dots$ is a strictly increasing sequence of positive integers such that the subsequences $$s_{s_1}, s_{s_2}, s_{s_3}, \dots \quad \text{and} \quad s_{s_1 + 1}, s_{s_2 + 1}, s_{s_3 + 1}, \dots$$ are both arithmetic progressions. Prove that the sequence $s_1, s_2, s_3, \dots$ is itself an arithmetic progression.

\textbf{1168. (\color{red}d10\color{black}, USA TSTST 2021, P9 of 9)} Let $q=p^r$ for a prime number $p$ and positive integer $r$. Let $\zeta = e^{\frac{2\pi i}{q}}$. Find the least positive integer $n$ such that
\[\sum_{\substack{1\leq k\leq q\\ \gcd(k,p)=1}} \frac{1}{(1-\zeta^k)^n}\]is not an integer. (The sum is over all $1\leq k\leq q$ with $p$ not dividing $k$.)

\textbf{832. (\color{red}d10\color{black}, 2020 USAMO, P3 of 6)} Let $p$ be an odd prime. An integer $x$ is called a quadratic non-residue if $p$ does not divide $x-t^2$ for any integer $t$.
\smallbreak
Denote by $A$ the set of all integers $a$ such that $1\le a<p$, and both $a$ and $4-a$ are quadratic non-residues. Calculate the remainder when the product of the elements of $A$ is divided by $p$.

\textbf{748. (\color{red}d10\color{black}, 2016 IMO, P3 of 6)} Let $P=A_1A_2\cdots A_k$ be a convex polygon in the plane. The vertices $A_1, A_2, \ldots, A_k$ have integral coordinates and lie on a circle. Let $S$ be the area of $P$. An odd positive integer $n$ is given such that the squares of the side lengths of $P$ are integers divisible by $n$. Prove that $2S$ is an integer divisible by $n$.

\textbf{664. (\color{red}d10\color{black}, 2010 IMO, P3 of 6)} Find all functions $g:\mathbb{N}\rightarrow\mathbb{N}$ such that\[\left(g(m)+n\right)\left(g(n)+m\right)\]is a perfect square for all $m,n\in\mathbb{N}.$

\textbf{657. (\color{red}d10\color{black}, 2009 IMOSL, N6)} Let $k$ be a positive integer. Show that if there exists a sequence $a_0,a_1,\ldots$ of integers satisfying the condition\[a_n=\frac{a_{n-1}+n^k}{n}\text{ for all } n\geq 1,\]then $k-2$ is divisible by $3$.

\textbf{629. (\color{red}d10\color{black}, 2012 IMO, P6 of 6)} Find all positive integers $n$ for which there exist non-negative integers $a_1, a_2, \ldots, a_n$ such that
\[
    \frac{1}{2^{a_1}} + \frac{1}{2^{a_2}} + \cdots + \frac{1}{2^{a_n}} =
    \frac{1}{3^{a_1}} + \frac{2}{3^{a_2}} + \cdots + \frac{n}{3^{a_n}} = 1.
\]

\textbf{594. (\color{red}d10\color{black}, Folklore )} Let $x$ and $y$ be positive integers. Suppose $x \pmod{p} \leq y \pmod{p}$ for all primes $p$ (where we define $a \pmod{p}$ to be the remainder of $a$ upon division by $p$). Must $x = y$?

\textbf{468. (\color{red}d10\color{black}, 2003 IMOSL, N8)} Let $p$ be a prime number and let $A$ be a set of positive integers that satisfies the following conditions:\\ (i) the set of prime divisors of the elements in $A$ consists of $p-1$ elements;\\ (ii) for any nonempty subset of $A$, the product of its elements is not a perfect $p$-th power.\\ What is the largest possible number of elements in $A$ ?

\textbf{460. (\color{red}d10\color{black}, 2000 IMOSL N6, Generalised)} Determine whether there are infinitely many positive integers that cannot be represented as a sum of distinct 2000th powers.

\textbf{390. (\color{red}d10\color{black}, 2015 IMOSL N8)} For every positive integer $n$ with prime factorisation $\textstyle{n = \prod_{i = 1}^{k} p_i^{\alpha_i}}$, define \[\mho(n) = \sum_{i: \; p_i > 10^{100}} \alpha_i.\] That is, $\mho(n)$ is the number of prime factors of $n$ greater than $10^{100},$ counted with multiplicity.

\makebox[1.5em]{} Find all strictly increasing functions $f: \mathbb{Z} \to \mathbb{Z}$ such that \[\mho \left( f(a) - f(b) \right) \leq \mho (a - b)\] for all integers \(a\) and \(b\) with \(a > b\).

\textbf{369. (\color{red}d10\color{black}, 2017 IMO, P6 of 6)} An ordered pair $(x, y)$ of integers is called a \emph{primitive point} if the greatest common divisor of $x$ and $y$ is 1. Given a finite set $S$ of primitive points, prove that there exist a positive integer $n$ and integers $a_0$, $a_1$, $\dots$, $a_n$ such that, for each $(x, y)$ in $S$, we have: $$a_0 x^n + a_1x^{n - 1}y + a_2 x^{n - 2}y^2 + \dots + a_{n - 1}x y^{n - 1} + a_n y^n = 1.$$

\textbf{313. (\color{red}d10\color{black}, 2015 USA TSTST Q3 of 6)} Let $P$ be the set of all primes, and let $M$ be a non-empty subset of $P.$ Suppose that for any non-empty subset $p_1, p_2, \dots, p_k$ of $M,$ all prime factors of $p_1p_2\dots p_k + 1$ are also in $M.$ Prove that $M = P.$

\textbf{181. (\color{red}d10\color{black}, 2018 IMOSL N7)} Let $n \ge 2018$ be an integer, and let $a_1, a_2, \dots, a_n, b_1, b_2, \dots, b_n$ be pairwise distinct positive integers not exceeding $5n$. Suppose that the sequence

$$\frac{a_1}{b_1}, \frac{a_2}{b_2}, \dots, \frac{a_n}{b_n}$$

forms an arithmetic progression. Prove that the terms of the sequence are equal.

\textbf{160. (\color{red}d10\color{black}, 2019 USAMO, P3 of 6)} Let $K$ be the set of all positive integers that do not contain the digit $7$ in their base-$10$ representation. Find all polynomials $f$ with nonnegative integer coefficients such that $f(n)\in K$ whenever $n\in K$.

\textbf{146. (\color{red}d11\color{black}, 2012 IMOSL, N8)} Prove that for every prime $p>100$ and every integer $r$, there exist two integers $a$ and $b$ such that $p$ divides $a^2+b^5-r$.

\textbf{426. (\color{red}d12\color{black}, Bertrand's postulate)} For every $n \in \mathbb N$, prove there is some prime $p$ satisfying $n \leq p \leq 2n$.

\textbf{244. (\color{red}d13\color{black}, Lagrange's Four Square Theorem)} Prove that every positive integer is the sum of four perfect squares.

\textbf{738. (\color{red}dT\color{black}, 2021 AFMO, P1 of 4)} Tony Wang is on a diet plan which lasts $p-2$ days. He plans to eat exactly $2^1 - 1$ noodles on the first day, $2^2 - 1$ noodles on the second day, and so on, up to $2^{p-2} - 1$ noodles on the $(p-2)$th day.

However, Tony gets sick if the number of noodles he eats in a day is a multiple of $p$.

Given that $p$ is prime and that Tony doesn't get sick in the $p-2$ days of his diet, what are the possible values for $p$?

\end{document}