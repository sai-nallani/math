\documentclass{article}
\usepackage{mdframed}
\usepackage{amsmath}
\usepackage{amssymb}
\usepackage{amsthm}
\theoremstyle{definition}
\newtheorem{definition}{Definition}[section]
\newtheorem{proposition}{Proposition}[definition]
\DeclareMathOperator*{\Dom}{Dom}
\DeclareMathOperator*{\Rng}{Rng}
\DeclareMathOperator*{\Cod}{Cod}

\title{Chapter 2, Limits and Continuity Notes}
\begin{document}
\maketitle
\section{Limits}
\begin{definition}
    Let $f$ be a real-valued function. We say that the \textbf{limit} of $f(x)$ as $x$ approaches $a$ is $L$, or
    \begin{center}
        \[ \lim_{x \to a} f(x) \]
    \end{center}
    if, for all $\epsilon > 0$, there exists $\delta > 0$ such that if $x$ is within $\delta$
    of $a$ (with $x \neq a$), then $f(x)$ is within $\epsilon$ of $L$. We write this more
    precisely as
    \begin{center}
        \[ 0 < |x - a| < \delta \Rightarrow |f(x) - L| < \epsilon\],
    \end{center}
    where the "$\Rightarrow$" symbol means "implies":
    \begin{center}
        If $0 < |x - a| < \delta$, then $|f(x) - L| < \epsilon$.
    \end{center}
\end{definition}
\end{document}