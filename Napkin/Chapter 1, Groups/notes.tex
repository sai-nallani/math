\documentclass{article}

\author{Sai Nallani}
\title{Chapter 1: Groups}
\begin{document}
\maketitle
A \textbf{group} is a pair $G = (G, *)$ consisting of a set of elements G, and
a binary operation * on G, such that:
\begin{enumerate}
    \item G has an \textbf{identity element}, usually denoted $1_G$ or just
          1, with the property that
          \begin{center}
              $1_G * g = g * 1_G = g$ for all $g \in G$.
          \end{center}
    \item The operation is associative, meaning $(a * b) * c = a * (b * c)$
          for any $a, b, c \in G$. Consequently we generally don't write the
          parantheses.
    \item Each element $g \in G$ has an \textbf{inverse}, that is, an element
          $h \in G$ such that
          \begin{center}
              $g * h = h * g = 1_G$
          \end{center}
\end{enumerate}

For the symmetric group, the group elements are not the numbers, but the
functions/permutations themselves.

\textbf{Fact 1.2.3} Let $G$ be a group.
\begin{enumerate}
    \item The identity of a group is unique.
          Let 1 and $1'$ are identities, then $1 = 1 * 1' = 1'$.
    \item If $h$ and $h'$ are inverses to $g$, then $1_G = g * h \Rightarrow
              h' = (h' * g) * h = 1_g * h = h$.
    \item For any $g \in G$, $(g^{-1})^{-1} = g$
          Let $h = g^{-1}$. $gh = 1_G$, $hh^{-1} = 1_G$, therefore $g=h^{-1}=(g^{-1})^{-1}$.

\end{enumerate}
\end{document}