\documentclass{article}
\usepackage[utf8]{inputenc}
\usepackage{amsmath}

\title{Random Proofs}
\author{sxbuto}
\date{December 2022}

\begin{document}
\maketitle

\section{Principle of Inclusion and Exclusion}
\subsection{2 Sets}
We have two sets, $A$ and $B$. Let's show that
$|A \cup B| = |A| + |B| - |A \cap B|$. If we add $|A|$ and $|B|$,
we overcount the overlap $|A \cap B|$, therefore we subtract it.
\subsection{3 Sets}
We have three sets, $A$, $B$, and $C$. Let's find $|A \cup B \cup C|$.
$|A \cup B \cup C| \neq |A| + |B| + |C|$
as we overcounted overlaps of $A \cap B, B \cap C, C \cap A$.
Let's subtract those values.
$|A \cup B \cup C| \neq
    |A| + |B| + |C| - |A \cap B| - |B \cap C| - |C \cap A|$
Oh no! We undercounted $A \cap B \cap C$. Let's just add it!\\
\begin{center}
    $|A \cup B \cup C| =
        |A| + |B| + |C|
        - |A \cap B| - |B \cap C| - |C \cap A| + |A \cap B \cap C|$.
\end{center}
\subsection{$N$ sets}
We have $N$ sets, $S_1$, $S_2$, ..., $S_{N-1}$, and $S_N$. We wish to find
$|\bigcup\limits_{i=1}^{N}S_i|$. Let's first add all sizes:
$\sum\limits_{i=1}^{N}|S_i|$. Let's subtract each intersection of two sets.
$\sum\limits_{i=1}^{N}|S_i| - \sum\limits_{1\leq{i}<j\leq{n}}^{N}|S_i \cap S_j| +
    \sum\limits_{1\leq{i}<j<k\leq{n}}^{N}|S_i \cap S_j \cap S_k| + ... +
    (-1)^{n-1}|S_1 \cap S_2 \cap S_3 \cap ... \cap S_N|$.
\section{Derangements}
\subsection{Introduction}
Let's say we have the set $A = \{1,2,3,4,5,6\}$. How many
ways can we permute the numbers such that every number
in A isn't in its original position. An example of
this kind of permutation is $\{2,1,4,3,6,5\}$. How many
such arrangements, aka \textbf{derangements}, can we make?
\subsection{Answer}
We know there are $6!$ total permutations. The total
number of derangements is less than the total permutations.
Let's just exclude some types of arrangements. Let $N$ be
the number of permutations where there is atleast 1 number
in the original position. The total number of derangements
would be $6! - N$. Let $A_r$ be the set of permutations
such that the $r$th object is in the right position. Then,
$N = |\bigcup\limits_{r=1}^{N}A_r|$. For generality, let $n=6$ in
this scenario, but keep in mind it can be any natural number. By PIE,\\
\begin{center}
    \begin{large}
        1. $\sum\limits_{i=1}^{n}{|A_i|} = \binom{n}{1}(n-1)!$\\
        2.  $\sum\limits_{i<j}^{n}{|A_i \cap A_j|} = \binom{n}{2}(n-2)!$\\
        % \line(1, 0){300}\\
        3. $N = |\bigcup\limits_{r=1}^{n}A_r|
            = \sum\limits_{i=1}^{n}{|A_i|}
            - \sum\limits_{i<j}{|A_i \cap A_j|}
            + \sum\limits_{i<j<k}{|A_i \cap A_j \cap A_k|}
            - \sum\limits_{i<j<k<l}
            {|A_i\cap{A_j}\cap{A_k}\cap{A_l}|} + ...
            + (-1)^{n+1}|A_1 \cap A_2 \cap ... \cap A_n|
            = \sum\limits_{i=1}^{n}{\binom{n}{i}(-1)^{n+1}
            (n-i)!}=\sum\limits_{i=1}^{n}{(-1)^{n+1}\frac{n!}{i!}}$\\
        4. $6! - N = n! - \sum\limits_{i=1}^{n}{(-1)^{n+1}\frac{n!}{i!}}
            = n! - n!\sum\limits_{i=1}^{n}{(-1)^{n+1}\frac{1}{i!}}
            = n!(1-\sum\limits_{i=1}^{n}{(-1)^{n+1}\frac{1}{i!}})
            = n!\sum\limits_{i=0}^{n}{(-1)^{n+1}\frac{1}{i!}}$\\
        5. The number of derangements is
        $\boxed{n!\sum\limits_{i=0}^{n}{\frac{(-1)^{n}}{i!}}}$ or
        $\boxed{n![1-1+\frac{1}{2}-\frac{1}{6}+...+\frac{(-1)^n}{n!}]}$
    \end{large}
\end{center}
\end{document}